\documentclass[11pt,twoside,twocolumn]{book}

\RequirePackage[left=1in,
                right=.767in,
                top=1in,
                bottom=1in,
                headheight=14bp,
                headsep=9bp,
                columnsep=0.24in,
                footskip=14bp,
                heightrounded]{geometry}

\usepackage{amsmath}
\usepackage{algorithm}
\usepackage{algpseudocode}
\usepackage{bm}
\usepackage{calc}
\usepackage{natbib}
\usepackage{graphicx}
\usepackage{longtable}
\usepackage{caption}
\usepackage[]{titletoc}

%Do not allow a page break to result in a line appearing by itself 
% https://tex.stackexchange.com/questions/4152/how-do-i-prevent-widow-orphan-lines 
\usepackage[all]{nowidow}

\makeindex
\usepackage{setspace}
% uncomment to make double space 
%\doublespacing
\usepackage{etoolbox}
\usepackage{verbatim}

% set up the listings package for highlighting block definitions and input files
\usepackage{listings}
\usepackage{xcolor}
\lstset{
  basicstyle=\footnotesize\ttfamily\color{black},
  numbers=none,
  columns=flexible,
  backgroundcolor=\color{yellow!10},
%  frame=tlbr,
  moredelim=**[is][\color{red}]{@}{@},
}
\lstdefinestyle{blockdefinition}{
  moredelim=**[is][\color{blue}]{@}{@},
}
%usage: \lstinputlisting[style=blockdefinition]{./mf6ivar/tex/gwf-chd-dimensions.dat}
\lstdefinestyle{inputfile}{
  morecomment=[l]\#,
  backgroundcolor=\color{gray!10},
}
%usage: \lstinputlisting[style=modeloutput]{file.dat}
\lstdefinestyle{modeloutput}{
  backgroundcolor=\color{blue!20},
}

\usepackage[hidelinks]{hyperref}
\hypersetup{
    pdftitle={Basic Model Interface for MODFLOW 6},
    pdfauthor={MODFLOW 6 Development Team},
    pdfsubject={numerical simulation groundwater flow},
    pdfkeywords={groundwater, MODFLOW, simulation, basic model interface, CSDMS},
    pdflang={en-US},
    bookmarksnumbered=true,     
    bookmarksopen=true,         
    bookmarksopenlevel=1,       
    colorlinks=true,
    allcolors={blue},          
    pdfstartview=Fit,           
    pdfpagemode=UseOutlines,
    pdfpagelayout=TwoPageRight
}

\graphicspath{{./Figures/}}
\newcommand{\modflowversion}{mf6.1.1}
\newcommand{\modflowdate}{December 12, 2019}
\newcommand{\currentmodflowversion}{Version \modflowversion---\modflowdate}


\title{Basic Model Interface for MODFLOW 6}
\author{MODFLOW 6 Development Team}
\date{\currentmodflowversion}

\newcommand{\mli}[1]{\mathit{#1}}

\urlstyle{rm}

\newcommand{\programname}{MODFLOW 6}
\newcommand{\mf}{MODFLOW~6~}
\newcommand{\mfdot}{MODFLOW~6.~}
\newcommand{\mfcomma}{MODFLOW~6,~}
\newcommand{\mfpar}{(MODFLOW~6)~}
\usepackage{placeins}
\usepackage{float}
\floatstyle{plain}
\newfloat{exampleinput}{H}{exi}
\floatname{exampleinput}{}

\newcommand{\inreferences}{%
\renewcommand{\theequation}{R--\arabic{equation}}%
\setcounter{equation}{0}%
\renewcommand{\thefigure}{R--\arabic{figure}}%
\setcounter{figure}{0}%
\renewcommand{\thetable}{R--\arabic{table}}%
\setcounter{table}{0}%
\renewcommand{\thepage}{R--\arabic{page}}%
\setcounter{page}{1}%
}

\newcounter{appendixno}
\setcounter{appendixno}{0}
\newcommand{\inappendix}{%
\addtocounter{appendixno}{1}%
\renewcommand{\theequation}{\Alph{appendixno}--\arabic{equation}}%
\setcounter{equation}{0}%
\renewcommand{\thefigure}{\Alph{appendixno}--\arabic{figure}}%
\setcounter{figure}{0}%
\renewcommand{\thetable}{\Alph{appendixno}--\arabic{table}}%
\setcounter{table}{0}%
\renewcommand{\thepage}{\Alph{appendixno}--\arabic{page}}%
\setcounter{page}{1}%
}

\renewcommand{\thesection}{}
\renewcommand{\thesubsection}{}
\newcommand{\SECTION}{\section}
\newcommand{\REFSECTION}{\section}

\makeatletter
\renewcommand*\l@section{\@dottedtocline{1}{0em}{1.5em}}
\renewcommand\section{\@startsection {section}{1}{-1em}%
  {-3.5ex \@plus -1ex \@minus -.2ex}%
  {2.3ex \@plus.2ex}%
  {\normalfont\Large\bfseries}}
\def\sectionmark#1{%
      \markright {\MakeUppercase{#1}}}
\makeatother

\makeatletter
\patchcmd{\@verbatim}
  {\verbatim@font}
  {\verbatim@font\footnotesize}
  {}{}
\makeatother

\renewcommand\bibname{References Cited}

\begin{document}
%\makefrontcover

%\makefrontmatter

\onecolumn
\hbadness=10000
\setlength{\parindent}{1.5pc}

\maketitle
 
\tableofcontents
\listoffigures
\listoftables

\newpage

\input{./gwf/bcoptions.tex}

%Introduction for input instructions
\SECTION{Introduction}
The Initialize, Run, Finalize model and the Basic Model Interface (BMI) developed by the Community Surface Dynamics Modeling System group \citep{PECKHAM20133} has been implemented in \mf.

Add a more introduction text.

%Instructions for running a simulation
\SECTION{Running a Simulation}
\input{running_simulation.tex}

%General form of input instructions
\SECTION{Form of Input Instructions}
\mf differs from its predecessors in the form of the input.  Whereas previous MODFLOW versions read numerical values, arrays, and lists in a highly structured form, \mf reads information in the form of blocks and keywords.  \mf also reads arrays and lists of information, but these arrays and lists are tagged with identifying block names or keywords.  \mf will terminate with an error if it detects an unrecognized block or keyword.

\subsection{Block and Keyword Input} 

Input to \mf is provided within blocks.  A block is a section of an ASCII input file that begins with a line that has ``BEGIN'' followed by the name of the block and ends with a line the begins with ``END'' followed by the name of the block.  \mf will terminate with an error if blocks do not begin and end with the same name, or if a ``BEGIN'' or ``END'' line is missing.  Information within a block differs depending on the part of \mf that reads the block.  In general, keywords are used within blocks to turn options on or specify the type of information that follows the keyword.  If an unrecognized keyword is encountered in a block, \mf will terminate with an error.

The keyword approach is adopted in \mf to improve readability of the \mf input files, enhance discovery of errors in input files, and improve support for backward compatibility by allowing the program to expand in functionality while allowing previously developed models to be run with newer versions of the program.

Within these user instructions, keywords are shown in capital letters to differentiate them from other input that is provided by the user.  For example, ``BEGIN'' and ``END'' are recognized by \mf, and so they are capitalized.  Also, line indentation is used within these user instructions to help with readability of the blocks.  Typically, lines within a block are indented two spaces to accentuate that the lines are part of the block.  This indentation is not enforced by the program, but users are encouraged to use it within their own input files to improve readability.

Unless stated otherwise in this user guide, information contained within a block can be listed in any order.  If the same keyword is provided more than once, then the program will use the last information provided by that keyword.

Comment lines and blanks lines are also allowed within most blocks and within most input files.  Valid comment characters include ``\#'' ``!'', and ``//''.  Comments can also be placed at the end of some input lines, after the required information.  Comments are not allowed at the end of some lines if the program is required to read an arbitrary number of non-keyword items.  Comments included at the end of the line must be separated from the rest of the line by at least one space.

Unless otherwise noted in the input instructions, multiple blocks of the same name cannot be specified in a single input file.  The block order within the input file must follow the order presented in the input instructions.  Each input file typically begins with an OPTIONS block, which is generally not required, followed by one or more data blocks.

The following is an example of how the input instructions for a block are presented in this document.  
\begin{lstlisting}[style=blockdefinition]
BEGIN OPTIONS
  [AUXILIARY <auxiliary(naux)>]
  [PRINT_INPUT]
  [MAXIMUM_ITERATION <maxsfrit>]
END OPTIONS
\end{lstlisting}
This example shows the items that may be specified with this OPTIONS block.  Optional items are enclosed between ``['' and ``]'' symbols.  The ``\texttt{<}'' and ``\texttt{>}'' symbols indicate a variable that must be provided by the user.  In this case, \texttt{auxiliary} is an array of size \texttt{naux}.  Because there are bracket symbols around the entire item, the user it not required to specify anything for this item.  Likewise, the user may or may not invoke the ``\texttt{PRINT\_INPUT}'' option.  Lastly, the user can specify ``\texttt{MAXIMUM\_ITERATION}'' followed by a numeric value for ``\texttt{maxsfrit}''.  If the user does not specify an optional item, then a default condition will apply.  Behavior of the default condition is described in the input instructions for that item.

\vspace{6pt}\noindent A valid user input block for OPTIONS might be:

\begin{lstlisting}[style=inputfile]
#This is my options block
BEGIN OPTIONS
  AUXILIARY temperature salinity
  MAXIMUM_ITERATION 10
END OPTIONS
\end{lstlisting}

\noindent The following is another valid user input block for OPTIONS:

\begin{lstlisting}[style=inputfile]
#This is an alternative options block
BEGIN OPTIONS
  # Assign two auxiliary variables
  AUXILIARY temperature salinity
  # Specify the maximum iteration
  MAXIMUM_ITERATION 10
  #specify the print input option
  PRINT_INPUT
END OPTIONS
#done with the options block
\end{lstlisting}

\subsection{Specification of Block Information in OPEN/CLOSE File} 
For most blocks, information can be read from a separate text file.  In this case, all of the information for the block must reside in the text file.  The file name is specified using the OPEN/CLOSE keyword as the first and only entry in the block as follows:

\begin{lstlisting}[style=inputfile]
#This is an alternative options block
BEGIN OPTIONS
  OPEN/CLOSE myoptblock.txt
END OPTIONS
\end{lstlisting}

\noindent When MODFLOW encounters the OPEN/CLOSE keyword, the program opens the specified file on unit 99 and continues processing the information in the file as if it were within the block itself.  When the program reaches the end of the file, the file is closed, and the program returns to reading the original package file.  The next line after the OPEN/CLOSE line must end the block.

Some blocks do not support the OPEN/CLOSE capability.  A list of all of the blocks, organized by component and input file type, are listed in a table in appendix A.  This table also indicates the blocks that do not support the OPEN/CLOSE capability.

\subsection{File Name Input}
Some blocks may require that a file name be entered.  Although spaces within a file name are not generally recommended, they can be specified if the entire file name is enclosed within single quotes, which means that the file name itself cannot have a single quote within it.  On Windows computers, file names are not case sensitive, and thus, ``model.dis'' can be referenced within the input files as ``MODEL.DIS''.  On some other operating systems, however, file names are case sensitive and the case used in the input instructions must exactly reflect the case used to name the file.

\subsection{Lengths of Character Variables}
Character variables, which are used to store names of models, packages, observations and other objects, are limited in the number of characters that can be used. Table \ref{table:characterlength} lists the limit used for each type of character variable.

\FloatBarrier
\input{../Common/characterlengthtable.tex}
\FloatBarrier

\subsection{Integer and Floating Point Variables}
\mf uses integer and floating point variables throughout the program.  The sizes of these variables are defined in a single module within the program.  Information about the precision, range, and size of integers and floating point real variables is written to the top of the simulation list file: 

{\small
\begin{lstlisting}[style=modeloutput]
MODFLOW was compiled using uniform precision.

Real Variables
  KIND: 8
  TINY (smallest non-zero value):    2.225074-308
  HUGE (largest value):    1.797693+308
  PRECISION: 15
  BIT SIZE: 64

Integer Variables
  KIND: 4
  HUGE (largest value): 2147483647
  BIT SIZE: 32

Long Integer Variables
  KIND: 8
  HUGE (largest value): 9223372036854775807
  BIT SIZE: 64

Logical Variables
  KIND: 4
  BIT SIZE: 32
\end{lstlisting}
}

This information indicates that real variables have about 15 digits of precision.  The smallest positive non-zero value that can be stored is 2.2e-308.  The largest value that can be stored is 1.8e+308.  If the user enters a value in an input file that cannot be stored, such as 1.9335e-310 for example, then the program can produce unexpected results.  This does not affect an exact value of zero, which can be stored accurately.  Integer variables also have a maximum and minimum value, which is about 2 billion.  Values larger and smaller than this cannot be stored.  These numbers are rarely exceeded for most practical problems, but as the size of models (number of nodes) increase into the billions, then the program may need to be recompiled using a larger size for integer variables. Long integers are used to calculate the amount of memory allocated in the memory manager:

{\small
\begin{lstlisting}[style=modeloutput]
 MEMORY MANAGER TOTAL STORAGE BY DATA TYPE, IN MEGABYTES
 -------------------------------
                    ALLOCATED   
 DATA TYPE           MEMORY     
 -------------------------------
 Character        1.53300000E-03
 Logical          4.40000000E-05
 Integer          100.03799     
 Real             223.43994     
 -------------------------------
 Total            323.47951     
 -------------------------------
\end{lstlisting}
}

Currently, standard precision 4 byte logical variables are used throughout the program.


%Simulation name file
\newpage
\SECTION{Simulation Name File}
The simulation name file contains information about simulation options, simulation timing, models that are present in the simulation, how models exchange information, and how models are solved.

The present version of \mf uses the concept of a solution group.  For most simulations, a solution group will contain one solution and one model within that solution.  The solution group is designed, however, so that multiple solutions can be solved together in a picard iteration loop.  This might be used in the future to iteratively couple models that cannot be tightly coupled at the matrix level within a single numerical solution.  The solution group is flexible so that multiple solution groups can be included in a simulation.  More information on solution groups will be added to this document as new model types and exchanges are added that can take advantage of the concept.

The simulation name file is read from a file in the current working directory with the name ``mfsim.nam''.  Input within the simulation name file is provided through the following input blocks, which must be listed in the order shown below.  The options block itself is optional.  All other blocks are required.

\vspace{5mm}
\subsection{Structure of Blocks}
\lstinputlisting[style=blockdefinition]{./mf6ivar/tex/sim-nam-options.dat}
\lstinputlisting[style=blockdefinition]{./mf6ivar/tex/sim-nam-timing.dat}
\lstinputlisting[style=blockdefinition]{./mf6ivar/tex/sim-nam-models.dat}
\lstinputlisting[style=blockdefinition]{./mf6ivar/tex/sim-nam-exchanges.dat}
\lstinputlisting[style=blockdefinition]{./mf6ivar/tex/sim-nam-solutiongroup.dat}

\vspace{5mm}
\subsection{Explanation of Variables}
\begin{description}
% DO NOT MODIFY THIS FILE DIRECTLY.  IT IS CREATED BY mf6ivar.py 

\item \textbf{Block: OPTIONS}

\begin{description}
\item \texttt{CONTINUE}---keyword flag to indicate that the simulation should continue even if one or more solutions do not converge.

\item \texttt{NOCHECK}---keyword flag to indicate that the model input check routines should not be called prior to each time step. Checks are performed by default.

\item \texttt{memory\_print\_option}---is a flag that controls printing of detailed memory manager usage to the end of the simulation list file.  NONE means do not print detailed information. SUMMARY means print only the total memory for each simulation component. ALL means print information for each variable stored in the memory manager. NONE is default if MEMORY\_PRINT\_OPTION is not specified.

\item \texttt{maxerrors}---maximum number of errors that will be stored and printed.

\end{description}
\item \textbf{Block: TIMING}

\begin{description}
\item \texttt{tdis6}---is the name of the Temporal Discretization (TDIS) Input File.

\end{description}
\item \textbf{Block: MODELS}

\begin{description}
\item \texttt{mtype}---is the type of model to add to simulation.

\item \texttt{mfname}---is the file name of the model name file.

\item \texttt{mname}---is the user-assigned name of the model.  The model name cannot exceed 16 characters and must not have blanks within the name.  The model name is case insensitive; any lowercase letters are converted and stored as upper case letters.

\end{description}
\item \textbf{Block: EXCHANGES}

\begin{description}
\item \texttt{exgtype}---is the exchange type.

\item \texttt{exgfile}---is the input file for the exchange.

\item \texttt{exgmnamea}---is the name of the first model that is part of this exchange.

\item \texttt{exgmnameb}---is the name of the second model that is part of this exchange.

\end{description}
\item \textbf{Block: SOLUTIONGROUP}

\begin{description}
\item \texttt{group\_num}---is the group number of the solution group.  Solution groups must be numbered sequentially, starting with group number one.

\item \texttt{mxiter}---is the maximum number of outer iterations for this solution group.  The default value is 1.  If there is only one solution in the solution group, then MXITER must be 1.

\item \texttt{slntype}---is the type of solution.  The Integrated Model Solution (IMS6) is the only supported option in this version.

\item \texttt{slnfname}---name of file containing solution input.

\item \texttt{slnmnames}---is the array of model names to add to this solution.  The number of model names is determined by the number of model names the user provides on this line.

\end{description}


\end{description}

\begin{table}[h]
\caption{Model types available in Version \modflowversion}
\small
\begin{center}
\begin{tabular*}{\columnwidth}{l l}
\hline
\hline
Mtype & Type of Model \\
\hline
GWF6 & Groundwater Flow Model \\
GWT6 & Groundwater Transport Model \\
\hline 
\end{tabular*}
\label{table:mtype}
\end{center}
\normalsize
\end{table}

\begin{table}[h]
\caption{Exchange types available in Version \modflowversion}
\small
\begin{center}
\begin{tabular*}{\columnwidth}{l p{15cm}}
\hline
\hline
Exgtype & Type of Exchange \\
\hline
GWF6-GWF6 & Exchange between two Groundwater Flow Models.  Input for this file is described in a dedicated section in this guide. \\
GWF6-GWT6 & Exchange between a Groundwater Flow Model and a Groundwater Transport Model.  In the present version, a filename is required for this exchange and the file must exist, however, nothing is read from this file.  \\
\hline 
\end{tabular*}
\label{table:exgtype}
\end{center}
\normalsize
\end{table}

\vspace{5mm}
\subsection{Example Input File}
\lstinputlisting[style=inputfile]{./mf6ivar/examples/sim-nam-example.dat}



%Simulation timing
\newpage
\SECTION{Temporal Discretization (TDIS) Package}
\input{simulation_timing.tex}

%GWF Model Input Instructions
\newpage
\SECTION{Groundwater Flow (GWF) Model Input}
This section describes the data files for a \mf Groundwater Flow (GWF) Model.  A GWF Model is added to the simulation by including a GWF entry in the MODELS block of the simulation name file.

There are three types of spatial discretization approaches that can be used with the GWF Model.  Input for a GWF Model may be entered in a structured form, like for previous MODFLOW versions, in that users specify cells using their layer, row, and column indices.  Users may also work with a layered grid in which cells are defined using vertices.  In this case, users specify cells using the layer number and the cell number.  Lastly, GWF Models may be entered as fully unstructured models, in which cells are specified using only their cell number.  Once a spatial discretization approach has been selected, then all input with cell indices must be entered accordingly.

The GWF Model is designed to permit input to be gathered, as it is needed, from many different files.  Likewise, results from the model calculations can be written to a number of output files. The GWF Model Listing File is a key file to which the GWF model output is written.  As \mf runs, information about the GWF Model is written to the GWF Model Listing File, including much of the input data (as a record of the simulation) and calculated results.  Details about the files used by each package are provided in this section on the GWF Model Instructions.

\mf is further designed to allow the user to control the amount, type, and frequency of information to be output. Much of the output will be written to the Simulation and GWF Model Listing Files, but some model output can be written to other files.  The Listing Files can become very large for common models.  Text editors are useful for examining the Listing File. The GWF Model Listing File includes a summary of the input data read for all packages.  In addition, the GWF Model Listing File optionally contains calculated head controlled by time step, and the overall volumetric budget controlled by time step. The Listing Files also contain information about solver convergence and error messages.  Output to other files can include head and cell-by-cell flow terms for use in calculations external to the model or in user-supplied applications such as plotting programs.

The GWF Model reads a file called the Name File, which specifies most of the files that will be used in a simulation. Several files are always required whereas other files are optional depending on the simulation. The Output Control Package receives instructions from the user to control the amount and frequency of output.  Details about the Name File and the Output Control Package are described in this section.

\subsection{Information for Existing MODFLOW Users}
\mf contains most of the functionality of MODFLOW-2005, MODFLOW-NWT, MODFLOW-USG, and MODFLOW-LGR.  To the existing MODFLOW user, however, \mf will feel different from previous MODFLOW versions.  Some packages have been divided, renamed, or removed, and some capabilities, which previously caused confusion or were implemented due to computer memory limitations, are no longer supported (for example, ``quasi-3d confining units'' are not supported in the GWF Model).  The form of the input files for \mf is different from previous MODFLOW versions in that input files are now divided into blocks, and keywords are used to specify options and input variables.  Extensive testing was used as part of the development process to ensure that \mf simulation results are identical to the results from previous MODFLOW versions.  In some cases, it was not possible to exactly replicate the simulation results from previous MODFLOW versions.  In those cases, the differences could be explained by an option that is no longer supported, or because of slight differences in the underlying formulation.  

The following list has been updated from \cite{modflow6gwf}, and summarizes the major differences between the GWF Model in \mf and previous versions of MODFLOW.  This list is intended for those with a general understanding of the capabilities in previous versions of MODFLOW.

\begin{enumerate}

\item The GWF Model in \mf supports three alternative input packages for specifying the grid used to discretize the groundwater system.  
\begin{itemize}
\item The Discretization (DIS) Package defines a grid based on layers, rows, and columns.  In this report, this type of grid is referred to as a ``regular MODFLOW grid'' because it corresponds to traditional MODFLOW grids.  An interior cell in a regular MODFLOW grid is connected to four adjacent cells in the same layer, to one overlying cell, and to one underlying cell.
\item The Discretization by Vertices (DISV) Package defines a grid using a list of ($x$, $y$) vertex pairs and the number of layers.  A list of vertices is provided by the user to define a two-dimensional horizontal grid in plan view.  This list of vertices may define a regular MODFLOW grid, or they may define more complex grids, such as grids consisting of triangles, hexagons, or Voronoi polygons, for example.  This same two-dimensional horizontal grid applies to each layer in the model.  Cells defined using the DISV Package are referenced by layer number and by the cell number within the horizontal grid.  Within a layer, a cell may be horizontally connected to any number of surrounding cells in that layer.  In the vertical direction a cell can be connected to only one overlying cell and only one underlying cell.  Grids defined with the DISV Package are considered to be unstructured.
\item The unstructured Discretization (DISU) Package is the most flexible of the three packages and is patterned after the unstructured grid implemented in MODFLOW-USG.  For each cell, the user specifies a list of connected cells and the connection properties.  When the DISU Package is used, cells are referenced only by their cell number; unlike the MODFLOW-USG approach, there is no concept of a layer in the DISU Package in \mfcomma but cells may still overlie or underlie one another.  
\end{itemize}

\item For the three grid types supported in the GWF Model (DIS, DISV, and DISU), cells can be permanently excluded from the grid for the simulation.  Input values (such as hydraulic conductivity) are still required for these excluded cells, and the program will write special codes or zero values for output, but the program does not allocate memory or store values for excluded cells during run time.  In this case, the matrix equations are formulated for a reduced system in which only the included cells are numbered.  Users can also mark excluded cells as ``vertical pass-through cells,'' but this option is only available for DIS and DISV grids.  When these vertical pass-through cells are encountered, the program connects the cells overlying and underlying the pass-through cell.  This capability allows ``pinched'' cells to be removed from the solution.  These options to exclude cells or exclude them as pass-through cells are available through specification of the IDOMAIN array.

\item There is no longer a Basic Package input file.  Initial head values are specified using an Initial Conditions (IC) Package, and constant heads are specified using the Time Varying Specified Head (CHD) Package.  Cells that are permanently excluded from the simulation can be eliminated using the IDOMAIN capability entered through the DIS or DISV Packages.  For a cell that may transition from inactive (``dry'') to active (``wet'') during a simulation, the user can start the cell as inactive by assigning an initial head below the cell bottom.

\item The Newton-Raphson formulations and accompanying upstream weighting schemes implemented in MODFLOW-NWT and MODFLOW-USG for handling dry or nearly dry cells have been synthesized into a single formulation.  The Newton-Raphson formulation in the GWF Model for \mf remains an optional alternative to the standard formulation used in most previous MODFLOW versions. Much of the \cite{modflow6gwf} report is focused on systematically explaining standard and Newton-Raphson formulations for the GWF Model and its packages.

\item Information on temporal discretization, such as number of stress periods, period lengths, number of time steps, and time step multipliers, is specified at the simulation level, rather than for an individual model.  This information is provided in the Timing Module, which controls the temporal discretization and applies to all models within a simulation.  The Timing Module is part of the \mf framework and is described separately in \cite{modflow6framework}.

\item Aquifer properties used to calculate hydraulic conductance are specified in the Node Property Flow (NPF) Package.  In \mfcomma the NPF Package calculates intercell conductance values, manages cell wetting and drying, and adds Newton-Raphson terms for intercell flow expressions.  The NPF Package allows individual cells to be designated as confined or convertible; this was not an option in previous MODFLOW versions as the designation was by layer.  The NPF Package also has several options for simulating drainage problems and problems involving perched aquifers where an active cell overlies a partially saturated cell.  The default NPF Package behavior (in which none of these options are set) is the most stable for typical groundwater problems.  The default NPF Package behavior does not correspond to the default behavior for other MODFLOW internal flow packages.  The NPF Package does not support quasi-3D confining units.  The NPF Package replaces the Layer Property Flow (LPF), Block-Centered Flow (BCF), and Upstream Weighting (UPW) Packages from previous MODFLOW versions.  Capabilities of the Hydrogeologic Unit Flow (HUF) Package \citep{anderman2000modflow, anderman2003modflow} are not supported in the GWF Model of \mf.

\item Aquifer storage properties are specified in the Storage (STO) Package.  If the STO Package is excluded for a model, then the model represents steady-state conditions.  If the STO Package is included, users can specify steady-state or transient conditions by stress period as needed.  Compressible storage contributions are no longer approximated as zero for unconfined layers; contributions from pore drainage and compressible storage are separated in the model output.

\item The Horizontal Flow Barrier (HFB) Package \citep{hsieh1993hfb, modflow2005} in \mf allows barrier properties and locations to change by stress period.  The capability to change barriers by stress period was not supported in previous MODFLOW versions.

\item The GWF Model in \mf allows multiple stress packages of the same type to be specified for a single GWF Model.  This capability is also available in MODFLOW-CDSS \citep{banta2011modflow}.  Package entries written to the budget file and budget terms in the listing file are written separately for each package.

\item Input of boundary conditions for simulation in multiple stress periods is entered differently than for previous MODFLOW versions. Boundary conditions are specified for a stress period in a ``PERIOD'' block. These boundary conditions remain active at their specified values until a subsequent ``PERIOD'' block is encountered or the end of the simulation is reached.  Individual entries within the ``PERIOD'' block can be specified as a time-series entry.  Values for these variables, which may correspond to a well pumping rate or a drain conductance, for example, are interpolated from a time-series dataset, for each time step, using several different interpolation options.

\item The Flow and Head Boundary (FHB) Package \citep{leake1997documentation, modflow2005} is not supported in \mf; however, its capabilities can be replicated using the WEL Package, the CHD Package, and the new time-series capability.

\item There is one Evapotranspiration (EVT) Package for \mf. The \mf EVT Package contains the functionality of the MODFLOW-2005 EVT Package, the Segmented Evapotranspiration (ETS) Package \citep{modflowdrtpack}, and the Riparian Evapotranspiration (RIP-ET) Package \citep{modflowripetpack}.

\item A new Multi-Aquifer Well (MAW) Package replaces the Multi-Node Well (MNW1 and MNW2) Packages \citep{halford2002, konikow2009}. The new package does not contain all of the options available in MNW1 and MNW2, but it does contain the most commonly used ones.  It also has new capabilities for simulating flowing wells. The MAW Package is solved as part of the matrix solution and is tightly coupled with the GWF Model. This tight coupling with the GWF Model may substantially improve convergence for simulations of groundwater flow to multi-aquifer wells.

\item Most capabilities of the Stream (STR) and Streamflow Routing (SFR) Packages \citep{prudic1989str, modflowsfr1pack, modflowsfr2pack} are included in \mf as a new SFR Package.  The new SFR Package contains all of the functionality of the SFR Package in MODFLOW-2005 with the following exceptions: (a) the concept of a ``segment'' has been eliminated, (b) only rectangular cross sections are supported for stream reaches, and (c) unsaturated zone flow beneath stream reaches cannot be simulated.

\item A new Lake (LAK) Package replaces the existing MODFLOW Lake Packages \citep{modflowlak3pack}. In addition to being able to represent lakes that are incised into the model grid, the new LAK Package can also represent sub-grid scale lakes that are conceptualized as being on top of the model.  The status of a lake can change during the simulation between \texttt{ACTIVE}, \texttt{INACTIVE}, and \texttt{CONSTANT}.  The new package contains most of the capabilities available in previous LAK Packages, including the ability to apply recharge and evapotranspiration to underlying cells if the lake is dry.  The LAK Package documented here does not represent unsaturated zone flow beneath a lake or support for the coalescing lake option described in \cite{modflowlak3pack}. 

\item A new Unsaturated Zone Flow (UZF) Package, based on the one described by \cite{UZF}, is included in the GWF Model of \mfdot The new UZF Package allows the UZF capabilities to be applied to only selected cells of the GWF model. The new UZF Package also supports a multi-layer option, which allows for vertical heterogeneity in unsaturated zone properties.

\item A new Water Mover (MVR) Package is included in \mfdot  The MVR Package can be used to transfer water from individual ``provider'' features of selected packages (WEL, DRN, RIV, GHB, MAW, SFR, LAK, and UZF) to individual ''receiver'' features of the advanced packages (MAW, SFR, LAK, and UZF).  Simple rules are used to determine how much of the available water is moved from the provider to the receiver, which allows management controls to be represented. 

\item A new Skeletal Storage, Compaction, and Subsidence (CSUB) Package was added to \mf in version 6.1.0. The one-dimensional effective-stress based compaction theory implemented in the CSUB Package is documented in \cite{leake2007modflow}. The numerical approach used for delay interbeds in the CSUB package is documented in \cite{hoffmann2003modflow} and uses the same one-dimensional effective-stress based compaction theory as coarse-grained and fine-grained no-delay interbed sediments.

\item \mf contains a flexible new Observation (OBS) capability, which allows the user to define many different types of continuous-in-time observations.  The new OBS capability replaces the Observation Process \citep{hill2000modflow}, the Gage Package, and the HYDMOD capability \citep{hanson1999documentation} in previous MODFLOW versions.  Flow, head, and drawdown observations can be obtained for the GWF Model.  Flow and other package-specific observations, such as the head in a multi-aquifer well or lake stage, for example, can also be obtained.  These observed values can be used subsequently with a parameter estimation program or they can be used to make time-series plots of a wide range of simulated values.  The new OBS capability does not support specification of field-measured observations, calculation of residuals, or interpolation within a grid, as was supported in previous versions of the MODFLOW OBS Process.

\item The GWF Model described in this report does not support the following list of packages and capabilities.  Support for some of these capabilities may be added in future \mf versions.
  \begin{itemize}
    \item Drain with Return Flow Package \citep{modflowdrtpack},
    \item Reservoir Package \citep{fenske1996documentation},
    \item Seawater Intrusion Package \citep{bakker2013documentation},
    \item Surface-Water Routing Process \citep{hughes2012documentation},
    \item Connected Linear Network Process \citep{modflowusg},
    \item Parameter Value File \citep{modflow2005}, and
    \item Link to the MT3DMS Contaminant Transport Model \citep{zheng2001modflow}.  However, MT3D-USGS can read the head and budget files created by MODFLOW 6, but only if the GWF Model uses the DIS Package.  MT3D-USGS will not work with GWF output if the DISV or DISU Packages are used.
  \end{itemize}

\end{enumerate}

In addition to this list of major differences, there are other differences between \mf and previous MODFLOW versions in terms of the input and output files and the way users interact with the program.  These differences include:

\begin{enumerate}

\item The \mf program begins by reading a simulation name file.  The simulation name file must be named ``mfsim.nam.''

\item All real variables in \mf are declared as double precision floating point numbers.  Real variables written to binary output files are also written in double precision.

\item Unit numbers are no longer specified by the user.  Unit numbers are determined automatically by \mf based upon user-provided file names.

\item The GWF Model name file contains a list of packages that are active for the model.  Names for output files are not specified in the name file.  Names for output files, such as the head and budget files are specified in the OC Package.

\item The EXTERNAL option for reading arrays and lists is no longer supported; however, the OPEN/CLOSE option is still supported.  The SFAC option for lists is no longer supported; however, many packages allow for specification of an auxiliary variable which can serve as a multiplier on a column of values in the list.

\item The CHD Package contains new flexibility.  Cells can transition between constant-head cells and active cells during the simulation.  This was not allowed in previous MODFLOW versions.  Also, the CHD Packages no longer performs linear interpolation between a starting (shead) and ending head (ehead).  Only a single head value is provided for each constant-head cell.  The capability to linearly interpolate a head value for each time step within a stress period is available through the use of time series.

\item There are two different forms of input for the RCH and EVT Packages: array-based input and list-based input.  For models that use DIS Package, the RCH and EVT input can be provided as arrays, which is consistent with previous MODFLOW versions.  To use array input, the user must specify the READASARRAYS keyword in the options block.  The READASARRAYS option can also be used for models that use the DISV Package.  If the READASARRAYS option is not specified, then input to the RCH and EVT Packages is provided in list form.  List-based input is the only option supported when the DISU Package is used.

List-based input offers several advantages over the array-based input for specifying recharge and evapotranspiration.  First, multiple list entries can be specified for a single cell.  This makes it possible to divide a cell into multiple areas, and assign a different recharge or evapotranspiration rate for each area (perhaps based on land use or some other criteria).  In this case, the user would likely specify an auxiliary variable to serve as a multiplier.  This multiplier would be calculated by the user and provided in the input file as the fractional cell are for the individual recharge entries.  Another advantage to using list-based input for specifying recharge is that ``boundnames'' can be specified.  Boundnames work with the Observations capability and can be used to sum recharge or evapotranspiration rates for entries with the same boundname.  A disadvantage of the list-based input is that one cannot easily assign recharge or evapotranspiration rates to the entire model without specifying a list of model cells.  For this reason \mf also supports array-based input.

\item Calculation and reporting of drawdown for the model grid is no longer supported, as this calculation is easily performed as a postprocessing step.  Calculation of drawdown is supported as an observation type by the OBS Package; 
drawdown is calculated as the difference between the starting head specified in the IC Package and the calculated head.

\item There are differences in the output files created by \mfcomma such as:
\begin{itemize}

\item A separate listing file is written for the simulation.  This simulation listing file contains information about the simulation, including solver information.  Separate listing files are written for each GWF Model that is part of the simulation.

\item Unformatted head files written by \mf are consistent with those written by previous MODFLOW versions; however, all real values are written in double precision.

\item The budget file written by the GWF Model is always written in ``compact'' form (as opposed to full three-dimensional arrays) and uses new method codes, which allow model and package names to be written to the file.  Simulated intercell flows are always written in a compressed sparse row format, even for regular MODFLOW grids.

\item Information about the GWF Model grid is written to a separate file, called a ``binary grid file'' each time the model runs.  The binary grid file can be used by postprocessing programs for subsequent analysis.  The format of the binary grid file is described in a section on ``Binary Output Files.''

\end{itemize}


\end{enumerate}


\input{gwf/array_data.tex}

\subsection{Units of Length and Time}
The GWF Model formulates the groundwater flow equation without using prescribed length and time units. Any consistent units of length and time can be used when specifying the input data for a simulation. This capability gives a certain amount of freedom to the user, but care must be exercised to avoid mixing units.  The program cannot detect the use of inconsistent units.  For example, if hydraulic conductivity is entered in units of feet per day and pumpage as cubic meters per second, the program will run, but the results will be meaningless. Other processes generally are expected to work with consistent length and time units; however, other processes could conceivably place restrictions on which units are supported.

The user can set flags that specify the length and time units (see the input instructions for the Timing Module and Spatial Discretization Files), which may be useful in various parts of MODFLOW.  For example, the program will label the table of simulation time with time units if the time units are specified by the optional TIME\_UNITS label, which can be set in the TDIS Package.  If the time units are not specified, the program still runs, but the table of simulation time does not indicate the time units. An optional LENGTH\_UNITS label can be set in the Discretization Package. Situations in other processes may require that the length or time units be specified.  In such situations, the input instructions will state the requirements. Remember that specifying the unit flags does not enforce consistent use of units.  The user must insure that consistent units are used in all input data.

\subsection{Steady-State Simulations}
A steady-state simulation is represented by a single stress period having a single time step with the storage term set to zero. Setting the number and length of stress periods and time steps is the responsibility of the Timing Module of the \mf framework. The length of the stress period and time step will not affect the head solution because the time derivative is not calculated in a steady-state problem. Setting the storage term to zero is the responsibility of the Storage Package. Most other packages need not "know" that a simulation is steady state.

A GWF Model also can be mixed transient and steady state because each stress period can be designated transient or steady state.  Thus, a GWF Model can start with a steady-state stress period and continue with one or more transient stress periods.  The settings for controlling steady-state and transient options are in the Storage Package.  If the Storage Package is not specified for a GWF Model, then the storage terms are zero and the GWF Model will be steady state.

\subsection{Volumetric Budget}
A summary of all inflows (sources) and outflows (sinks) of water is called a water budget.  The water budget for the GWF Model is termed a volumetric budget because volumes of water and volumetric flow rates are involved; thus strictly speaking, a volumetric budget is not a mass balance, although this term has been used in other model reports.  \mf calculates a water budget for the overall model as a check on the acceptability of the solution, and to provide a summary of the sources and sinks of water to the flow system.  The water budget is printed to the GWF Model Listing File for selected time steps.

Numerical solution techniques for simultaneous equations do not always result in a correct answer; in particular, iterative solvers may stop iterating before a sufficiently close approximation to the solution is attained.  A water budget provides an indication of the overall acceptability of the solution.  The system of equations solved by the model actually consists of a flow continuity statement for each model cell.  Continuity should also exist for the total flows into and out of the model---that is, the difference between total inflow and total outflow should equal the total change in storage.  In the model program, the water budget is calculated independently of the equation solution process, and in this sense may provide independent evidence of a valid solution.

The total budget as printed in the output does not include internal flows between model cells---only flows into or out of the model as a whole. For example, flow to or from rivers, flow to or from constant-head cells, and flow to or from wells are all included in the overall budget terms.  Flow into and out of storage is also considered part of the overall budget inasmuch as accumulation in storage effectively removes water from the flow system and storage release effectively adds water to the flow---even though neither process, in itself, involves the transfer of water into or out of the ground-water regime.  Each hydrologic package calculates its own contribution to the budget.

For every time step, the budget subroutine of each hydrologic package calculates the rate of flow into and out of the system due to the process simulated by the package.  The inflows and outflows for each component of flow are stored separately.  Most packages deal with only one such component of flow.  In addition to flow, the volumes of water entering and leaving the model during the time step are calculated as the product of flow rate and time-step length.  Cumulative volumes, from the beginning of the simulation, are then calculated and stored.

The GWF Model uses the inflows, outflows, and cumulative volumes to write the budget to the Listing File at the times requested by the model user.  When a budget is written, the flow rates for the last time step and cumulative volumes from the beginning of simulation are written for each component of flow.  Inflows are written separately from outflows.  Following the convention indicated above, water entering storage is treated as an outflow (that is, as a loss of water from the flow system) while water released from storage is treated as an inflow (that is, a source of water to the flow system).  In addition, total inflow and total outflow are written, as well as the difference between total inflow and outflow.  The difference is then written as a percentage error, calculated using the formula:

\begin{equation}
D = \frac{100 (IN-OUT)}{(IN + OUT) / 2}
\end{equation}

\noindent where $D$ is the percentage error term, $IN$ is the total inflow to the system, and $OUT$ is the total outflow.

If the model equations are solved correctly, the percentage error should be small.  In general, flow rates may be taken as an indication of solution validity for the time step to which they apply, while cumulative volumes are an indication of validity for the entire simulation up to the time of the output.  The budget is written to the GWF Model Listing File at the end of each stress period whether requested or not.

\subsection{Cell-By-Cell Flows}
In some situations, calculating flow terms for various subregions of the model is useful.  To facilitate such calculations, provision has been made to save flow terms for individual cells in a separate binary file so they can be used in computations external to the model itself.  These individual cell flows are referred to here as ``cell-by-cell'' flow terms and are of four general types: (1) cell-by-cell stress flows, or flows into or from an individual cell caused by one of the external stresses represented in the model, such as evapotranspiration or recharge; (2) cell-by-cell storage terms, which give the rate of accumulation or depletion of storage in an individual cell; and (3) internal cell-by-cell flows, which are actually the flows across individual cell faces---that is, between adjacent model cells.  These four kinds of cell-by-cell flow terms are discussed further in subsequent paragraphs.  To save any of these cell-by-cell terms, two flags in the model input must be set.  The input to the Output Control file indicates the time steps for which cell-by-cell terms are to be saved. In addition, each hydrologic package includes an option called SAVE\_FLOWS that must be set if the cell-by-cell terms computed by that package are to be saved.  Thus, if the appropriate option in the Evapotranspiration Package input is set, cell-by-cell evapotranspiration terms will be saved for each time step for which the saving of cell-by-cell flow is requested through the Output Control Option.  Only flow values are saved in the cell-by-cell files; neither water volumes nor cumulative water volumes are included.  The flow dimensions are volume per unit time, where volume and time are in the same units used for all model input data.  The cell-by-cell flow values are stored in unformatted form to make the most efficient use of disk space; see the Budget File section toward the end of this user guide for information on how the data are written to a file.

The cell-by-cell storage term gives the net flow to or from storage in a variable-head cell.  The net storage for each cell in the grid is saved in transient simulations if the appropriate flags are set.  Withdrawal from storage in the cell is considered positive, whereas accumulation in storage is considered negative.

The cell-by-cell constant-head flow term gives the flow into or out of an individual constant-head cell (specified with the CHD Package).  This term is always associated with the constant-head cell itself, rather than with the surrounding cells that contribute or receive the flow.  A constant-head cell may be surrounded by as many as six adjacent variable-head cells for a regular grid or any number of cells for the other grid types.  The cell-by-cell calculation provides a single flow value for each constant-head cell, representing the algebraic sum of the flows between that cell and all of the adjacent variable-head cells.  A positive value indicates that the net flow is away from the constant-head cell (into the variable-head part of the grid); a negative value indicates that the net flow is into the constant-head cell.

The internal cell-by-cell flow values represent flows across the individual faces of a model cell.  Flows between cells are written in the compressed row storage format, whereby the flow between cell $n$ and each one of its connecting $m$ neighbor cells are contained in a single one-dimensional array.  Flows are positive for the cell in question.  Thus the flow reported for cell $n$ and its connection with cell $m$ is opposite in sign to the flow reported for cell $m$ and its connection with cell $n$.  These internal cell-by-cell flow values are useful in calculations of the groundwater flow into various subregions of the model, or in constructing flow vectors.

Cell-by-cell stress flows are flow rates into or out of the model, at a particular cell, owing to one particular external stress.  For example, the cell-by-cell evapotranspiration term for cell $n$ would give the flow out of the model by evapotranspiration from cell $n$.  Cell-by-cell stress flows are considered positive if flow is into the cell, and negative if out of the cell.

\newpage
\subsection{GWF Model Name File}
The GWF Model Name File specifies the options and packages that are active for a GWF model.  The Name File contains two blocks: OPTIONS  and PACKAGES. The length of each line must be 299 characters or less. The lines in each block can be in any order.  Files listed in the PACKAGES block must exist when the program starts. 

Comment lines are indicated when the first character in a line is one of the valid comment characters.  Commented lines can be located anywhere in the file. Any text characters can follow the comment character. Comment lines have no effect on the simulation; their purpose is to allow users to provide documentation about a particular simulation. 

\vspace{5mm}
\subsubsection{Structure of Blocks}
\lstinputlisting[style=blockdefinition]{./mf6ivar/tex/gwf-nam-options.dat}
\lstinputlisting[style=blockdefinition]{./mf6ivar/tex/gwf-nam-packages.dat}

\vspace{5mm}
\subsubsection{Explanation of Variables}
\begin{description}
\input{./mf6ivar/tex/gwf-nam-desc.tex}
\end{description}

\begin{table}[H]
\caption{Ftype values described in this report.  The \texttt{Pname} column indicates whether or not a package name can be provided in the name file}
\small
\begin{center}
\begin{tabular*}{\columnwidth}{l l l}
\hline
\hline
Ftype & Input File Description & \texttt{Pname}\\
\hline
DIS6 & Rectilinear Discretization Input File \\
DISV6 & Discretization by Vertices Input File \\
DISU6 & Unstructured Discretization Input File \\
IC6 & Initial Conditions Package \\
OC6 & Output Control Option \\
NPF6 & Node Property Flow Package \\ 
STO6 & Storage Package \\
CSUB6 & Compaction and Subsidence Package \\
HFB6 & Horizontal Flow Barrier Package\\
CHD6 & Time-Variant Specified Head Option & * \\
WEL6 & Well Package & * \\
DRN6 & Drain Package & * \\
RIV6 & River Package & * \\
GHB6 & General-Head Boundary Package & * \\
RCH6 & Recharge Package & * \\
EVT6 & Evapotranspiration Package & * \\
MAW6 & Multi-Aquifer Well Package & * \\
SFR6 & Streamflow Routing Package & * \\
LAK6 & Lake Package & * \\
UZF6 & Unsaturated Zone Flow Package & * \\
MVR6 & Water Mover Package \\
GNC6 & Ghost-Node Correction Package \\
OBS6 & Observations Option \\
\hline 
\end{tabular*}
\label{table:ftype}
\end{center}
\normalsize
\end{table}

\vspace{5mm}
\subsubsection{Example Input File}
\lstinputlisting[style=inputfile]{./mf6ivar/examples/gwf-nam-example.dat}



\newpage
\subsection{Structured Discretization (DIS) Input File}
\input{gwf/dis}

\newpage
\subsection{Discretization by Vertices (DISV) Input File}
\input{gwf/disv}

\newpage
\subsection{Unstructured Discretization (DISU) Input File}
\input{gwf/disu}

\newpage
\subsection{Initial Conditions (IC) Package}
Initial Conditions (IC) Package information is read from the file that is specified by ``IC6'' as the file type.  Only one IC Package can be specified for a GWT model. 

\vspace{5mm}
\subsubsection{Structure of Blocks}
%\lstinputlisting[style=blockdefinition]{./mf6ivar/tex/gwf-ic-options.dat}
\lstinputlisting[style=blockdefinition]{./mf6ivar/tex/gwt-ic-griddata.dat}

\vspace{5mm}
\subsubsection{Explanation of Variables}
\begin{description}
\input{./mf6ivar/tex/gwt-ic-desc.tex}
\end{description}

\vspace{5mm}
\subsubsection{Example Input File}
\lstinputlisting[style=inputfile]{./mf6ivar/examples/gwt-ic-example.dat}



\newpage
\subsection{Output Control (OC) Option}
Input to the Output Control Option of the Groundwater Transport Model is read from the file that is specified as type ``OC6'' in the Name File. If no ``OC6'' file is specified, default output control is used. The Output Control Option determines how and when concentrations are printed to the listing file and/or written to a separate binary output file.  Under the default, concentration and overall transport budget are written to the Listing File at the end of every stress period. The default printout format for concentrations is 10G11.4.  The concentrations and overall transport budget are also written to the list file if the simulation terminates prematurely due to failed convergence.

Output Control data must be specified using words.  The numeric codes supported in earlier MODFLOW versions can no longer be used.

For the PRINT and SAVE options of concentration, there is no option to specify individual layers.  Whenever the concentration array is printed or saved, all layers are printed or saved.

\vspace{5mm}
\subsubsection{Structure of Blocks}
\vspace{5mm}

\noindent \textit{FOR EACH SIMULATION}
\lstinputlisting[style=blockdefinition]{./mf6ivar/tex/gwt-oc-options.dat}
\vspace{5mm}
\noindent \textit{FOR ANY STRESS PERIOD}
\lstinputlisting[style=blockdefinition]{./mf6ivar/tex/gwt-oc-period.dat}

\vspace{5mm}
\subsubsection{Explanation of Variables}
\begin{description}
\input{./mf6ivar/tex/gwt-oc-desc.tex}
\end{description}

\vspace{5mm}
\subsubsection{Example Input File}
\lstinputlisting[style=inputfile]{./mf6ivar/examples/gwt-oc-example.dat}


\newpage
\subsection{Observation (OBS) Utility for a GWF Model}

GWF Model observations include the simulated groundwater head (\texttt{head}), calculated drawdown (\texttt{drawdown}) at a node, and the flow between two connected nodes (\texttt{flow-ja-face}). The data required for each GWF Model observation type is defined in table~\ref{table:gwfobstype}. For \texttt{flow-ja-face} observation types, negative and positive values represent a loss from and gain to the \texttt{cellid} specified for ID, respectively.

\subsubsection{Structure of Blocks}
\vspace{5mm}

\noindent \textit{FOR EACH SIMULATION}
\lstinputlisting[style=blockdefinition]{./mf6ivar/tex/utl-obs-options.dat}
\lstinputlisting[style=blockdefinition]{./mf6ivar/tex/utl-obs-continuous.dat}

\subsubsection{Explanation of Variables}
\begin{description}
\input{./mf6ivar/tex/utl-obs-desc.tex}
\end{description}


\begin{longtable}{p{2cm} p{2.75cm} p{2cm} p{1.25cm} p{7cm}}
\caption{Available GWF model observation types} \tabularnewline

\hline
\hline
\textbf{Model} & \textbf{Observation type} & \textbf{ID} & \textbf{ID2} & \textbf{Description} \\
\hline
\endhead

\hline
\endfoot

\input{../Common/gwf-obs.tex}
\label{table:gwfobstype}
\end{longtable}

\vspace{5mm}
\subsubsection{Example Observation Input File}

An example GWF Model observation file is shown below.

\lstinputlisting[style=inputfile]{./mf6ivar/examples/utl-obs-example-obs.dat}



\newpage
\subsection{Node Property Flow (NPF) Package}
\input{gwf/npf}

\newpage
\subsection{Horizontal Flow Barrier (HFB) Package}
\input{gwf/hfb}

\newpage
\subsection{Storage (STO) Package}
\input{gwf/sto}

\newpage
\subsection{Skeletal Storage, Compaction, and Subsidence (CSUB) Package}
Input to the Skeletal Storage, Compaction, and Subsidence (CSUB) Package is read from the file that has type ``CSUB6'' in the Name File.  If the CSUB Package is not included for a model, then storage changes resulting from compaction will not be calculated.  Only one CSUB Package can be specified for a GWF model. Only the first and last stress period can be specified to be STEADY-STATE in the STO Package when the CSUB Package is being used in the GWF model. Also the specific storage (SS) must be specified to be zero in the STO Package for every cell.

\vspace{5mm}
\subsubsection{Structure of Blocks}

\vspace{5mm}
\noindent \textit{FOR EACH SIMULATION}
\lstinputlisting[style=blockdefinition]{./mf6ivar/tex/gwf-csub-options.dat}
\lstinputlisting[style=blockdefinition]{./mf6ivar/tex/gwf-csub-dimensions.dat}
\lstinputlisting[style=blockdefinition]{./mf6ivar/tex/gwf-csub-griddata.dat}
\lstinputlisting[style=blockdefinition]{./mf6ivar/tex/gwf-csub-packagedata.dat}
\vspace{5mm}
\noindent \textit{FOR ANY STRESS PERIOD}
\lstinputlisting[style=blockdefinition]{./mf6ivar/tex/gwf-csub-period.dat}
\packageperioddescription

\vspace{5mm}
\subsubsection{Explanation of Variables}
\begin{description}
\input{./mf6ivar/tex/gwf-csub-desc.tex}
\end{description}

\vspace{5mm}
\subsubsection{Example Input File}
\lstinputlisting[style=inputfile]{./mf6ivar/examples/gwf-csub-example.dat}


\vspace{5mm}
\subsubsection{Available observation types}
Subsidence Package observations include all of the terms that contribute to the continuity equation for each GWF cell. The data required for each CSUB Package observation type is defined in table~\ref{table:gwf-csubobstype}. Negative and positive values for \texttt{CSUB} observations represent a loss from and gain to the GWF model, respectively.


\begin{longtable}{p{2cm} p{2.75cm} p{2cm} p{1.25cm} p{7cm}}
\caption{Available CSUB Package observation types} \tabularnewline

\hline
\hline
\textbf{Stress Package} & \textbf{Observation type} & \textbf{ID} & \textbf{ID2} & \textbf{Description} \\
\hline
\endfirsthead

\captionsetup{textformat=simple}
\caption*{\textbf{Table \arabic{table}.}{\quad}Available CSUB Package observation types.---Continued} \\

\hline
\hline
\textbf{Stress Package} & \textbf{Observation type} & \textbf{ID} & \textbf{ID2} & \textbf{Description} \\
\hline
\endhead

\hline
\endfoot

\input{../Common/gwf-csubobs.tex}
\label{table:gwf-csubobstype}
\end{longtable}

\vspace{5mm}
\subsubsection{Example Observation Input File}
\lstinputlisting[style=inputfile]{./mf6ivar/examples/gwf-csub-example-obs.dat}


\newpage
\subsection{Buoyancy (BUY) Package}
Input to the Buoyancy (BUY) Package is read from the file that has type ``BUY6'' in the Name File.  If the BUY Package is included for a model, then the model will use the variable-density form of Darcy's Law for all flow calculations using the approach described by \cite{langevin2020hydraulic}.  Only one BUY Package can be specified for a GWF model. The BUY Package can be coupled with one or more GWT Models so that fluid density is updated dynamically with one or more simulated concentration fields.

The BUY Package calculates fluid density using the following equation of state from \cite{langevin2008seawat}:

\begin{equation}
\label{eqn:volumeconservationdiscrete}
%\rho = \rho_0 + \sum_{i=1}^{NRHOSPECIES} \left ( C_i - C_{i,0} \right )
\rho = DENSEREF + \sum_{i=1}^{NRHOSPECIES} DRHODC_i \left ( CONCENTRATION_i - CRHOREF_i \right )
\end{equation}

\noindent where $\rho$ is the calculated density, $DENSEREF$ is the density of a reference fluid, typically taken to be freshwater at a temperature of 25 degrees Celsius; $NRHOSPECIES$ is the number of chemical species that contribute to the density calculation, $DRHODC_i$ is the parameter that describes how density changes as a function of concentration for chemical species $i$ (i.e. the slope of a line that relates density to concentration), $CONCENTRATION_i$ is the concentration of species $i$, and $CRHOREF_i$ is the concentration of species $i$ in the reference fluid, which is normally set to zero.

\subsubsection{Stress Packages}
For head-dependent stress packages, the BUY Package may require fluid density and elevation for each head-dependent boundary so that the model can use a variable-density form of Darcy's Law to calculate flow between the boundary and the aquifer.  By default, the boundary elevation is set equal to the cell elevation.  For water-table conditions, the cell elevation is calculated as bottom elevation plus half of saturation multiplied by the cell thickness.  If desired, the user can more precisely locate the boundary elevation by specifying an auxiliary variable with the name ``ELEVATION''.  The program will use the values in this column as the boundary elevation.  A situation where this may be required is for river or general-head boundaries that are conceptualized as being on top of a model cell.  In those cases, an ELEVATION column should be specified and the values set to the top of the cell or some other appropriate elevation that corresponds to where the boundary stage applies.

By default, the boundary density is set equal to DENSEREF, commonly specified as the density of freshwater; however, there are two other options for setting the density of a boundary package.  The first is to assign an auxiliary variable with the name ``DENSITY''.  If this auxiliary variable is detected, then the density value in this column will be assigned to the density for the boundary.  Alternatively, a density value can be calculated for each boundary using the density equation of state and one or more concentrations provided as auxiliary variables.  In this case, the user must assign one auxiliary variable for each AUXSPECIESNAME listed in the PACKAGEDATA block below.  Thus, there must be NRHOSPECIES auxiliary variables, each with the identical name as those specified in PACKAGEDATA.  The BUY Package will calculate the density for each boundary using these concentrations and the values specified for DENSEREF, DRHODC, and CRHOREF.  If the boundary package contains an auxiliary variable named DENSITY and also contains AUXSPECIESNAME auxiliary variables, then the boundary density value will be assigned to the one in the DENSITY auxiliary variable.

A GWT Model can be used to calculate concentrations for the advanced stress packages (LAK, SFR, MAW, and UZF) if corresponding advanced transport packages are specified (LKT, SFT, MWT, and UZT).  The advanced stress packages have an input option called FLOW\_PACKAGE\_AUXILIARY\_NAME.  When activated, this option will result in the simulated concentration for a lake or other feature being copied from the advanced transport package into the auxiliary variable for the corresponding GWF stress package.  This means that the density for a lake or stream, for example, can be dynamically updated during the simulation using concentrations from advanced transport packages that are fed into auxiliary variables in the advanced stress packages, and ultimately used by the BUY Package to calculate a fluid density using the equation of state.  This concept also applies when multiple GWT Models are used simultaneously to simulate multiple species.  In this case, multiple auxiliary variables are required for an advanced stress package, with each one representing a concentration from a different GWT Model.  

\begin{longtable}{p{3cm} p{12cm}}
\caption{Description of density terms for stress packages}
\tabularnewline
\hline
\hline
\textbf{Stress Package} & \textbf{Note} \\
\hline
\endhead
\hline
\endfoot
GHB & ELEVATION can be specified as an auxiliary variable.  A DENSITY auxiliary variable or one or more auxiliary variables for calculating density in the equation of state can be specified \\
RIV & ELEVATION can be specified as an auxiliary variable.  A DENSITY auxiliary variable or one or more auxiliary variables for calculating density in the equation of state can be specified \\
DRN & The drain formulation assumes that the drain boundary contains water of the same density as the discharging water; auxiliary variables have no affect on the drain calculation  \\
LAK & Elevation for each lake-aquifer connection is determined based on lake bottom and adjacent cell elevations. A DENSITY auxiliary variable or one or more auxiliary variables for calculating density in the equation of state can be specified \\
SFR & Elevation for each sfr-aquifer connection is determined based on stream bottom and adjacent cell elevations. A DENSITY auxiliary variable or one or more auxiliary variables for calculating density in the equation of state can be specified \\
MAW & Elevation for each maw-aquifer connection is determined based on cell elevation. A DENSITY auxiliary variable or one or more auxiliary variables for calculating density in the equation of state can be specified \\
UZF & No density terms implemented \\
\end{longtable}

\vspace{5mm}
\subsubsection{Structure of Blocks}

\vspace{5mm}
\noindent \textit{FOR EACH SIMULATION}
\lstinputlisting[style=blockdefinition]{./mf6ivar/tex/gwf-buy-options.dat}
\lstinputlisting[style=blockdefinition]{./mf6ivar/tex/gwf-buy-dimensions.dat}
\lstinputlisting[style=blockdefinition]{./mf6ivar/tex/gwf-buy-packagedata.dat}
%\vspace{5mm}
%\noindent \textit{FOR ANY STRESS PERIOD}
%\lstinputlisting[style=blockdefinition]{./mf6ivar/tex/gwf-buy-period.dat}

\vspace{5mm}
\subsubsection{Explanation of Variables}
\begin{description}
\input{./mf6ivar/tex/gwf-buy-desc.tex}
\end{description}

\vspace{5mm}
\subsubsection{Example Input File}
\lstinputlisting[style=inputfile]{./mf6ivar/examples/gwf-buy-example.dat}



\newpage
\subsection{Constant-Head (CHD) Package}
\input{gwf/chd}

\newpage
\subsection{Well (WEL) Package}
\input{gwf/wel}

\newpage
\subsection{Drain (DRN) Package}
\input{gwf/drn}

\newpage
\subsection{River (RIV) Package}
\input{gwf/riv}

\newpage
\subsection{General-Head Boundary (GHB) Package}
\input{gwf/ghb}

\newpage
\subsection{Recharge (RCH) Package -- List-Based Input}
\input{gwf/rch}

\newpage
\subsection{Recharge (RCH) Package -- Array-Based Input}
\input{gwf/rcha}

\newpage
\subsection{Evapotranspiration (EVT) Package -- List-Based Input}
\input{gwf/evt}

\newpage
\subsection{Evapotranspiration (EVT) Package -- Array-Based Input}
\input{gwf/evta}

\newpage
\subsection{Multi-Aquifer Well (MAW) Package}
\input{gwf/maw}

\newpage
\subsection{Streamflow Routing (SFR) Package}
\input{gwf/sfr}

\newpage
\subsection{Lake (LAK) Package}
\input{gwf/lak}

\newpage
\subsection{Unsaturated Zone Flow (UZF) Package}
\input{gwf/uzf}

\newpage
\subsection{Water Mover (MVR) Package}
\input{gwf/mvr}

\newpage
\subsection{Ghost-Node Correction (GNC) Package}
\input{gwf/gnc}

\newpage
\subsection{Groundwater Flow (GWF) Exchange}
\input{gwf/gwf-gwf}



%GWT Model Input Instructions
\newpage
\SECTION{Groundwater Transport (GWT) Model Input}
The GWT Model simulates three-dimensional transport of a single solute species in flowing groundwater.  The GWT Model solves the solute transport equation using numerical methods and a generalized control-volume finite-difference approach, which can be used with regular MODFLOW grids (DIS Package) or with unstructured grids (DISV and DISU Packages).  The GWT Model is designed to work with most of the new capabilities released with the GWF Model, including the Newton flow formulation, unstructured grids, advanced packages, and the movement of water between packages.  The GWF and GWT Models operate simultaneously during a \mf simulation to represent coupled groundwater flow and solute transport.  The GWT Model can also run separately from a GWF Model by reading the heads and flows saved by a previously run GWF Model.  The GWT model is also capable of working with the flows from another groundwater flow model, as long as the flows from that model can be written in the correct form to flow and head files.  

The purpose of the GWT Model is to calculate changes in solute concentration in both space and time.  Solute concentrations within an aquifer can change in response to multiple solute transport processes.  These processes include (1) advective transport of solute with flowing groundwater, (2) the combined hydrodynamic dispersion processes of velocity-dependent mechanical dispersion and chemical diffusion, (3) sorbtion of solutes by the aquifer matrix either by adsorption to individual solid grains or by absorbtion into solid grains, (4) transfer of solute into very low permeability aquifer material (called an immobile domain) where it can be stored and later released, (5) first- or zero-order solute decay or production in response to chemical or biological reactions, (6) mixing with fluids from groundwater sources and sinks, and (7) direct addition of solute mass.

With the present implementation, there can be multiple domains and multiple phases.  There is a single mobile domain, which normally consists of flowing groundwater, and there can be one or more immobile domains.  The GWT Model simulates the dissolved phase of chemical constituents in both the mobile and immobile domains.  The dissolved phase is also referred to in this report as the aqueous phase.  If sorbtion is represented, then the GWT Model also simulates the solid phase of the chemical constituent in both the mobile and immobile domains.  The dissolved and solid phases of the chemical constituent are tracked in the different domains by the GWT Model and can be reported as output as requested by the user.

This section describes the data files for a \mf Groundwater Transport (GWT) Model.  A GWT Model is added to the simulation by including a GWT entry in the MODELS block of the simulation name file.  There are three types of spatial discretization approaches that can be used with the GWT Model: DIS, DISV, and DISU.  The input instructions for these three packages are not described here in this section on GWT Model input; input instructions for these three packages are described in the section on GWF Model input.

The GWT Model is designed to permit input to be gathered, as it is needed, from many different files.  Likewise, results from the model calculations can be written to a number of output files. The GWT Model Listing File is a key file to which the GWT model output is written.  As \mf runs, information about the GWT Model is written to the GWT Model Listing File, including much of the input data (as a record of the simulation) and calculated results.  Details about the files used by each package are provided in this section on the GWT Model Instructions.

The GWT Model reads a file called the Name File, which specifies most of the files that will be used in a simulation. Several files are always required whereas other files are optional depending on the simulation. The Output Control Package receives instructions from the user to control the amount and frequency of output.  Details about the Name File and the Output Control Package are described in this section.

For the GWT Model, ``flows'' (unless stated otherwise) represent solute mass ``flow'' in mass per time, rather than groundwater flow.  

\subsection{Information for Existing Solute Transport Modelers}
The \mf GWT Model contains most of the functionality of MODFLOW-GWT, MT3DMS, MT3D-USGS and MODFLOW-USG.  The following list summarizes major differences between the GWT Model in \mf and previous MODFLOW-based solute transport programs.

\begin{enumerate}

\item The GWT Model simulates transport of a single chemical species; however, because \mf allows for multiple models of the same type to be included in a single simulation, multiple species can be represented by using multiple GWT Models.

\item There is no specialized flow and transport link file \citep{zheng2001modflow} used to pass the simulated groundwater flows to the transport model.  Instead, simulated flows from the GWF Model are passed in memory to the GWT Model while the program is running.  Alternatively, the GWT Model can read binary flow and head files saved by the GWF Model while it is running.  If the user intends to simulate transport through the advanced stress packages and Water Mover Package, then flows from these advanced packages must also be saved to binary files.  Names for these binary files are provided as input to the FMI Package.

\item The GWT Model is based on a generalized control-volume finite-difference method, which means that solute transport can be simulated using regular MODFLOW grids consisting of layers, rows, and columns, or solute transport can be simulated using unstructured grids.

\item Advection can be simulated using central-in-space weighting, upstream weighting, or an implicit second-order TVD scheme.  The GWT model does not have the Method of Characteristics (particle-based approaches) or an explicit TVD scheme.  Consequently, the GWT Model may require a higher level of spatial discretization than other transport models that use higher order terms for advection dominated systems.  This can be an important limitation for some problems, which require the preservation of sharp solute fronts. 

\item Variable-density flow and transport can be simulated by including a GWF Model and a GWT Model in the same \mf simulation.  The Buoyancy Package should be activated for the GWF Model so that fluid density is calculated as a function of simulated concentration.  If more than one chemical species is represented then the Buoyancy Package allows the simulated concentration for each of them to be used in the density equation of state.   \cite{langevin2020hydraulic} describe the hydraulic-head formation that is implemented in the Buoyancy Package for variable-density groundwater flow and present the results from \mf variable-density simulations.  The variable-density capabilities available in \mf replicate and extend the capabilities available in SEAWAT to include the Newton flow formulation and unstructured grids, for example.  

\item The GWT model includes the MST and IST Packages.  These two package collectively comprise the capabilities of the MT3DMS Reactions Package.

\item The MST Package contains the linear, Freundlich, and Langmuir isotherms for representing sorption.  The IST Packages contains only the linear isotherm for representation of sorption. 

\item The GWT model was designed so that the user can specify as many immobile domains and necessary to represent observed contaminant transport patterns and solute breakthrough curves.  The effects of an immobile domain are represented using the Immobile Storage and Transfer (IST) Package, and the user can specify as many IST Packages as necessary.  

\item Although there is GWF-GWF Exchange, a GWT-GWT Exchange has not yet been developed to connect multiple transport models, as might be done in a nested grid configuration.  

\item There is no option to automatically run the GWT Model to steady state using a single time step.  This is an option available in MT3DMS \citep{zheng2010supplemental}.  Steady state conditions must be determined by running the transport model under transient conditions until concentrations stabilize.

\item The GWT Model described in this report is capable of simulating solute transport in the advanced stress packages of \mfcomma including the Lake, Streamflow Routing, Multi-Aquifer Well and Unsaturated Zone Transport Packages.  The present implementation simulates solute advection between package features, such as between two stream reaches, but dispersive transport is not represented.  Likewise, solute transport between the advanced packages and the aquifer occurs only through advection.

\item The GWT Model has not yet been programmed to work with the Skeletal Storage, Compaction, and Subsidence (CSUB) Package for the GWF Model.  

\item There are many other differences between the \mf GWT Model and other solute transport models that work with MODFLOW, especially with regards to program design and input and output.  Descriptions for the GWT input and output are described here.

\end{enumerate}

\subsection{Units of Length and Time}
The GWF Model formulates the groundwater flow equation without using prescribed length and time units. Any consistent units of length and time can be used when specifying the input data for a simulation. This capability gives a certain amount of freedom to the user, but care must be exercised to avoid mixing units.  The program cannot detect the use of inconsistent units.

\subsection{Solute Mass Budget}
A summary of all inflow (sources) and outflow (sinks) of solute mass is called a mass budget.  \mf calculates a mass budget for the overall model as a check on the acceptability of the solution, and to provide a summary of the sources and sinks of mass to the flow system.  The solute mass budget is printed to the GWT Model Listing File for selected time steps.

\subsection{Time Stepping}

For the present implementation of the GWT Model, all terms in the solute transport equation are solved implicitly.  With the implicit approach applied to the transport equation, it is possible to take relatively large time steps and efficiently obtain a stable solution.  If the time steps are too large, however, accuracy of the model results will suffer, so there is usually some compromise required between the desired level of accuracy and length of the time step.  An assessment of accuracy can be performed by simply running simulations with shorter time steps and comparing results.

In \mf time step lengths are controlled by the user and specified in the Temporal Discretization (TDIS) input file.  When the flow model and transport model are included in the same simulation, then the length of the time step specified in TDIS is used for both models.  If the GWT Model runs in a separate simulation from the GWT Model, then the time steps used for the transport model can be different, and likely shorter, than the time steps used for the flow solution.  Instructions for specifying time steps are described in the TDIS section of this user guide; additional information on GWF and GWT configurations are in the Flow Model Interface section.  



\newpage
\subsection{GWT Model Name File}
The GWF Model Name File specifies the options and packages that are active for a GWF model.  The Name File contains two blocks: OPTIONS  and PACKAGES. The length of each line must be 299 characters or less. The lines in each block can be in any order.  Files listed in the PACKAGES block must exist when the program starts. 

Comment lines are indicated when the first character in a line is one of the valid comment characters.  Commented lines can be located anywhere in the file. Any text characters can follow the comment character. Comment lines have no effect on the simulation; their purpose is to allow users to provide documentation about a particular simulation. 

\vspace{5mm}
\subsubsection{Structure of Blocks}
\lstinputlisting[style=blockdefinition]{./mf6ivar/tex/gwf-nam-options.dat}
\lstinputlisting[style=blockdefinition]{./mf6ivar/tex/gwf-nam-packages.dat}

\vspace{5mm}
\subsubsection{Explanation of Variables}
\begin{description}
\input{./mf6ivar/tex/gwf-nam-desc.tex}
\end{description}

\begin{table}[H]
\caption{Ftype values described in this report.  The \texttt{Pname} column indicates whether or not a package name can be provided in the name file}
\small
\begin{center}
\begin{tabular*}{\columnwidth}{l l l}
\hline
\hline
Ftype & Input File Description & \texttt{Pname}\\
\hline
DIS6 & Rectilinear Discretization Input File \\
DISV6 & Discretization by Vertices Input File \\
DISU6 & Unstructured Discretization Input File \\
IC6 & Initial Conditions Package \\
OC6 & Output Control Option \\
NPF6 & Node Property Flow Package \\ 
STO6 & Storage Package \\
CSUB6 & Compaction and Subsidence Package \\
HFB6 & Horizontal Flow Barrier Package\\
CHD6 & Time-Variant Specified Head Option & * \\
WEL6 & Well Package & * \\
DRN6 & Drain Package & * \\
RIV6 & River Package & * \\
GHB6 & General-Head Boundary Package & * \\
RCH6 & Recharge Package & * \\
EVT6 & Evapotranspiration Package & * \\
MAW6 & Multi-Aquifer Well Package & * \\
SFR6 & Streamflow Routing Package & * \\
LAK6 & Lake Package & * \\
UZF6 & Unsaturated Zone Flow Package & * \\
MVR6 & Water Mover Package \\
GNC6 & Ghost-Node Correction Package \\
OBS6 & Observations Option \\
\hline 
\end{tabular*}
\label{table:ftype}
\end{center}
\normalsize
\end{table}

\vspace{5mm}
\subsubsection{Example Input File}
\lstinputlisting[style=inputfile]{./mf6ivar/examples/gwf-nam-example.dat}



%\newpage
%\subsection{Structured Discretization (DIS) Input File}
%\input{gwf/dis}

%\newpage
%\subsection{Discretization with Vertices (DISV) Input File}
%\input{gwf/disv}

%\newpage
%\subsection{Unstructured Discretization (DISU) Input File}
%\input{gwf/disu}

\newpage
\subsection{Initial Conditions (IC) Package}
Initial Conditions (IC) Package information is read from the file that is specified by ``IC6'' as the file type.  Only one IC Package can be specified for a GWT model. 

\vspace{5mm}
\subsubsection{Structure of Blocks}
%\lstinputlisting[style=blockdefinition]{./mf6ivar/tex/gwf-ic-options.dat}
\lstinputlisting[style=blockdefinition]{./mf6ivar/tex/gwt-ic-griddata.dat}

\vspace{5mm}
\subsubsection{Explanation of Variables}
\begin{description}
\input{./mf6ivar/tex/gwt-ic-desc.tex}
\end{description}

\vspace{5mm}
\subsubsection{Example Input File}
\lstinputlisting[style=inputfile]{./mf6ivar/examples/gwt-ic-example.dat}



\newpage
\subsection{Output Control (OC) Option}
Input to the Output Control Option of the Groundwater Transport Model is read from the file that is specified as type ``OC6'' in the Name File. If no ``OC6'' file is specified, default output control is used. The Output Control Option determines how and when concentrations are printed to the listing file and/or written to a separate binary output file.  Under the default, concentration and overall transport budget are written to the Listing File at the end of every stress period. The default printout format for concentrations is 10G11.4.  The concentrations and overall transport budget are also written to the list file if the simulation terminates prematurely due to failed convergence.

Output Control data must be specified using words.  The numeric codes supported in earlier MODFLOW versions can no longer be used.

For the PRINT and SAVE options of concentration, there is no option to specify individual layers.  Whenever the concentration array is printed or saved, all layers are printed or saved.

\vspace{5mm}
\subsubsection{Structure of Blocks}
\vspace{5mm}

\noindent \textit{FOR EACH SIMULATION}
\lstinputlisting[style=blockdefinition]{./mf6ivar/tex/gwt-oc-options.dat}
\vspace{5mm}
\noindent \textit{FOR ANY STRESS PERIOD}
\lstinputlisting[style=blockdefinition]{./mf6ivar/tex/gwt-oc-period.dat}

\vspace{5mm}
\subsubsection{Explanation of Variables}
\begin{description}
\input{./mf6ivar/tex/gwt-oc-desc.tex}
\end{description}

\vspace{5mm}
\subsubsection{Example Input File}
\lstinputlisting[style=inputfile]{./mf6ivar/examples/gwt-oc-example.dat}


\newpage
\subsection{Observation (OBS) Utility for a GWT Model}

GWT Model observations include the simulated groundwater concentration (\texttt{concentration}), and the mass flow, with units of mass per time, between two connected cells (\texttt{flow-ja-face}). The data required for each GWT Model observation type is defined in table~\ref{table:gwtobstype}. For \texttt{flow-ja-face} observation types, negative and positive values represent a loss from and gain to the \texttt{cellid} specified for ID, respectively.

\subsubsection{Structure of Blocks}
\vspace{5mm}

\noindent \textit{FOR EACH SIMULATION}
\lstinputlisting[style=blockdefinition]{./mf6ivar/tex/utl-obs-options.dat}
\lstinputlisting[style=blockdefinition]{./mf6ivar/tex/utl-obs-continuous.dat}

\subsubsection{Explanation of Variables}
\begin{description}
\input{./mf6ivar/tex/utl-obs-desc.tex}
\end{description}


\begin{longtable}{p{2cm} p{2.75cm} p{2cm} p{1.25cm} p{7cm}}
\caption{Available GWT model observation types} \tabularnewline

\hline
\hline
\textbf{Model} & \textbf{Observation type} & \textbf{ID} & \textbf{ID2} & \textbf{Description} \\
\hline
\endhead

\hline
\endfoot

\input{../Common/gwt-obs.tex}
\label{table:gwtobstype}
\end{longtable}

\vspace{5mm}
\subsubsection{Example Observation Input File}

An example GWT Model observation file is shown below.

\lstinputlisting[style=inputfile]{./mf6ivar/examples/utl-obs-gwt-example.dat}



\newpage
\subsection{Advection (ADV) Package}
Advection (ADV) Package information is read from the file that is specified by ``ADV6'' as the file type.  Only one ADV Package can be specified for a GWT model. 

\vspace{5mm}
\subsubsection{Structure of Blocks}
\lstinputlisting[style=blockdefinition]{./mf6ivar/tex/gwt-adv-options.dat}

\vspace{5mm}
\subsubsection{Explanation of Variables}
\begin{description}
\input{./mf6ivar/tex/gwt-adv-desc.tex}
\end{description}

\vspace{5mm}
\subsubsection{Example Input File}
\lstinputlisting[style=inputfile]{./mf6ivar/examples/gwt-adv-example.dat}



\newpage
\subsection{Dispersion (DSP) Package}
Dispersion (DSP) Package information is read from the file that is specified by ``DSP6'' as the file type.  Only one DSP Package can be specified for a GWT model.  The DSP Package is based on the mathematical formulation presented for the XT3D option of the NPF Package available to represent full three-dimensional anisotropy in groundwater flow.  XT3D can be computationally expensive and can be turned off to use a simplified and approximate form of the dispersion equations.  For most problems, however, XT3D will be required to accurately represent dispersion.

\vspace{5mm}
\subsubsection{Structure of Blocks}
\lstinputlisting[style=blockdefinition]{./mf6ivar/tex/gwt-dsp-options.dat}
\lstinputlisting[style=blockdefinition]{./mf6ivar/tex/gwt-dsp-griddata.dat}

\vspace{5mm}
\subsubsection{Explanation of Variables}
\begin{description}
\input{./mf6ivar/tex/gwt-dsp-desc.tex}
\end{description}

\vspace{5mm}
\subsubsection{Example Input File}
\lstinputlisting[style=inputfile]{./mf6ivar/examples/gwt-dsp-example.dat}



\newpage
\subsection{Source and Sink Mixing (SSM) Package}
Source and Sink Mixing (SSM) Package information is read from the file that is specified by ``SSM6'' as the file type.  Only one SSM Package can be specified for a GWT model.  

The SSM Package is used to add or remove solute mass from GWT model cells based on inflows and outflows from GWF stress packages.  If a GWF stress package provides flow into a model cell, that flow can be assigned a user-specified concentration.  This user-specified concentration must be entered as an auxiliary variable in the flow package.  In the SOURCES block below, the user provides the name of the package and the name of the auxiliary variable containing concentration values for each boundary.  As described below for srctype, there are multiple options for defining this behavior.

If the user does not enter a record for a GWF stress package in the SOURCES block, then inflow to the GWT model is assigned a concentration value of zero.  For negative flow rates in GWF stress packages (sinks), the sink concentration is set to the calculated cell concentration.

\vspace{5mm}
\subsubsection{Structure of Blocks}
\lstinputlisting[style=blockdefinition]{./mf6ivar/tex/gwt-ssm-options.dat}
\lstinputlisting[style=blockdefinition]{./mf6ivar/tex/gwt-ssm-sources.dat}

\vspace{5mm}
\subsubsection{Explanation of Variables}
\begin{description}
\input{./mf6ivar/tex/gwt-ssm-desc.tex}
\end{description}

\vspace{5mm}
\subsubsection{Example Input File}
\lstinputlisting[style=inputfile]{./mf6ivar/examples/gwt-ssm-example.dat}



\newpage
\subsection{Mobile Storage and Transfer (MST) Package}
Mobile Storage and Transfer (MST) Package information is read from the file that is specified by ``MST6'' as the file type.  Only one MST Package can be specified for a GWT model. 

\vspace{5mm}
\subsubsection{Structure of Blocks}
\lstinputlisting[style=blockdefinition]{./mf6ivar/tex/gwt-mst-options.dat}
\lstinputlisting[style=blockdefinition]{./mf6ivar/tex/gwt-mst-griddata.dat}

\vspace{5mm}
\subsubsection{Explanation of Variables}
\begin{description}
\input{./mf6ivar/tex/gwt-mst-desc.tex}
\end{description}

\vspace{5mm}
\subsubsection{Example Input File}
\lstinputlisting[style=inputfile]{./mf6ivar/examples/gwt-mst-example.dat}



\newpage
\subsection{Immobile Storage and Transfer (IST) Package}
Immobile Storage and Transfer (IMD) Package information is read from the file that is specified by ``IST6'' as the file type.  Any number of IST Packages can be specified for a single GWT model.  This allows the user to specify triple porosity systems, or systems with as many immobile domains as necessary. 

\vspace{5mm}
\subsubsection{Structure of Blocks}
\lstinputlisting[style=blockdefinition]{./mf6ivar/tex/gwt-ist-options.dat}
\lstinputlisting[style=blockdefinition]{./mf6ivar/tex/gwt-ist-griddata.dat}

\vspace{5mm}
\subsubsection{Explanation of Variables}
\begin{description}
\input{./mf6ivar/tex/gwt-ist-desc.tex}
\end{description}

\vspace{5mm}
\subsubsection{Example Input File}
\lstinputlisting[style=inputfile]{./mf6ivar/examples/gwt-ist-example.dat}



\newpage
\subsection{Constant Concentration (CNC) Package}
Constant Concentration (CNC) Package information is read from the file that is specified by ``CNC6'' as the file type.  Any number of CNC Packages can be specified for a single GWT model, but the same cell cannot be designated as a constant concentration by more than one CNC entry. 

\vspace{5mm}
\subsubsection{Structure of Blocks}
\vspace{5mm}

\noindent \textit{FOR EACH SIMULATION}
\lstinputlisting[style=blockdefinition]{./mf6ivar/tex/gwt-cnc-options.dat}
\lstinputlisting[style=blockdefinition]{./mf6ivar/tex/gwt-cnc-dimensions.dat}
\vspace{5mm}
\noindent \textit{FOR ANY STRESS PERIOD}
\lstinputlisting[style=blockdefinition]{./mf6ivar/tex/gwt-cnc-period.dat}
\packageperioddescription

\vspace{5mm}
\subsubsection{Explanation of Variables}
\begin{description}
\input{./mf6ivar/tex/gwt-cnc-desc.tex}
\end{description}

\vspace{5mm}
\subsubsection{Example Input File}
\lstinputlisting[style=inputfile]{./mf6ivar/examples/gwt-cnc-example.dat}

\vspace{5mm}
\subsubsection{Available observation types}
CNC Package observations are limited to the simulated constant concentration mass flow rate (\texttt{cnc}). The data required for the CNC Package observation type is defined in table~\ref{table:gwt-cncobstype}. Negative and positive values for an observation represent a loss from and gain to the GWT model, respectively.

\begin{longtable}{p{2cm} p{2.75cm} p{2cm} p{1.25cm} p{7cm}}
\caption{Available CNC Package observation types} \tabularnewline

\hline
\hline
\textbf{Model} & \textbf{Observation type} & \textbf{ID} & \textbf{ID2} & \textbf{Description} \\
\hline
\endhead

\hline
\endfoot

\input{../Common/gwt-cncobs.tex}
\label{table:gwt-cncobstype}
\end{longtable}

\vspace{5mm}
\subsubsection{Example Observation Input File}
\lstinputlisting[style=inputfile]{./mf6ivar/examples/gwt-cnc-example-obs.dat}


\newpage
\subsection{Mass Source Loading (SRC) Package}
Input to the Mass Source Loading (SRC) Package is read from the file that has type ``SRC6'' in the Name File.  Any number of SRC Packages can be specified for a single groundwater transport model.

\vspace{5mm}
\subsubsection{Structure of Blocks}
\vspace{5mm}

\noindent \textit{FOR EACH SIMULATION}
\lstinputlisting[style=blockdefinition]{./mf6ivar/tex/gwt-src-options.dat}
\lstinputlisting[style=blockdefinition]{./mf6ivar/tex/gwt-src-dimensions.dat}
\vspace{5mm}
\noindent \textit{FOR ANY STRESS PERIOD}
\lstinputlisting[style=blockdefinition]{./mf6ivar/tex/gwt-src-period.dat}
\packageperioddescription

\vspace{5mm}
\subsubsection{Explanation of Variables}
\begin{description}
\input{./mf6ivar/tex/gwt-src-desc.tex}
\end{description}

\vspace{5mm}
\subsubsection{Example Input File}
\lstinputlisting[style=inputfile]{./mf6ivar/examples/gwt-src-example.dat}

\vspace{5mm}
\subsubsection{Available observation types}
Mass Source Loading Package observations include the simulated source loading rates (\texttt{src}). The data required for each SRC Package observation type is defined in table~\ref{table:gwt-srcobstype}. The \texttt{src} observation is equal to the simulated mass source loading rate. Negative and positive values for an observation represent a loss from and gain to the GWT model, respectively.

\begin{longtable}{p{2cm} p{2.75cm} p{2cm} p{1.25cm} p{7cm}}
\caption{Available SRC Package observation types} \tabularnewline

\hline
\hline
\textbf{Stress Package} & \textbf{Observation type} & \textbf{ID} & \textbf{ID2} & \textbf{Description} \\
\hline
\endhead

\hline
\endfoot

\input{../Common/gwt-srcobs.tex}
\label{table:gwt-srcobstype}
\end{longtable}

\vspace{5mm}
\subsubsection{Example Observation Input File}
\lstinputlisting[style=inputfile]{./mf6ivar/examples/gwt-src-example-obs.dat}


\newpage
\subsection{Streamflow Transport (SFT) Package}
Streamflow Transport (SFT) Package information is read from the file that is specified by ``SFT6'' as the file type.  There can be as many SFT Packages as necessary for a GWT model. Each SFT Package is designed to work with flows from a corresponding GWF SFR Package. By default \mf uses the SFT package name to determine which SFR Package corresponds to the SFT Package.  Therefore, the package name of the SFT Package (as specified in the GWT name file) must match with the name of the corresponding SFR Package (as specified in the GWF name file).  Alternatively, the name of the flow package can be specified using the FLOW\_PACKAGE\_NAME keyword in the options block.  The GWT SFT Package cannot be used without a corresponding GWF SFR Package.

The SFT Package does not have a dimensions block; instead, dimensions for the SFT Package are set using the dimensions from the corresponding SFR Package.  For example, the SFR Package requires specification of the number of reaches (NREACHES).  SFT sets the number of reaches equal to NREACHES.  Therefore, the PACKAGEDATA block below must have NREACHES entries in it.

\vspace{5mm}
\subsubsection{Structure of Blocks}
\lstinputlisting[style=blockdefinition]{./mf6ivar/tex/gwt-sft-options.dat}
\lstinputlisting[style=blockdefinition]{./mf6ivar/tex/gwt-sft-packagedata.dat}
\lstinputlisting[style=blockdefinition]{./mf6ivar/tex/gwt-sft-period.dat}

\vspace{5mm}
\subsubsection{Explanation of Variables}
\begin{description}
\input{./mf6ivar/tex/gwt-sft-desc.tex}
\end{description}

\vspace{5mm}
\subsubsection{Example Input File}
\lstinputlisting[style=inputfile]{./mf6ivar/examples/gwt-sft-example.dat}

\vspace{5mm}
\subsubsection{Available observation types}
Streamflow Transport Package observations include reach concentration and all of the terms that contribute to the continuity equation for each reach. Additional SFT Package observations include mass flow rates for individual reaches, or groups of reaches. The data required for each SFT Package observation type is defined in table~\ref{table:gwt-sftobstype}. Negative and positive values for \texttt{sft} observations represent a loss from and gain to the GWT model, respectively. For all other flow terms, negative and positive values represent a loss from and gain from the SFT package, respectively.

\begin{longtable}{p{2cm} p{2.75cm} p{2cm} p{1.25cm} p{7cm}}
\caption{Available SFT Package observation types} \tabularnewline

\hline
\hline
\textbf{Stress Package} & \textbf{Observation type} & \textbf{ID} & \textbf{ID2} & \textbf{Description} \\
\hline
\endfirsthead

\captionsetup{textformat=simple}
\caption*{\textbf{Table \arabic{table}.}{\quad}Available SFT Package observation types.---Continued} \tabularnewline

\hline
\hline
\textbf{Stress Package} & \textbf{Observation type} & \textbf{ID} & \textbf{ID2} & \textbf{Description} \\
\hline
\endhead


\hline
\endfoot

\input{../Common/gwt-sftobs.tex}
\label{table:gwt-sftobstype}
\end{longtable}

\vspace{5mm}
\subsubsection{Example Observation Input File}
\lstinputlisting[style=inputfile]{./mf6ivar/examples/gwt-sft-example-obs.dat}




\newpage
\subsection{Lake Transport (LKT) Package}
Lake Transport (LKT) Package information is read from the file that is specified by ``LKT6'' as the file type.  There can be as many LKT Packages as necessary for a GWT model. Each LKT Package is designed to work with flows from a single corresponding GWF LAK Package. By default \mf uses the LKT package name to determine which LAK Package corresponds to the LKT Package.  Therefore, the package name of the LKT Package (as specified in the GWT name file) must match with the name of the corresponding LAK Package (as specified in the GWF name file).  Alternatively, the name of the flow package can be specified using the FLOW\_PACKAGE\_NAME keyword in the options block.  The GWT LKT Package cannot be used without a corresponding GWF LAK Package.

The LKT Package does not have a dimensions block; instead, dimensions for the LKT Package are set using the dimensions from the corresponding LAK Package.  For example, the LAK Package requires specification of the number of lakes (NLAKES).  LKT sets the number of lakes equal to NLAKES.  Therefore, the PACKAGEDATA block below must have NLAKES entries in it.

\vspace{5mm}
\subsubsection{Structure of Blocks}
\lstinputlisting[style=blockdefinition]{./mf6ivar/tex/gwt-lkt-options.dat}
\lstinputlisting[style=blockdefinition]{./mf6ivar/tex/gwt-lkt-packagedata.dat}
\lstinputlisting[style=blockdefinition]{./mf6ivar/tex/gwt-lkt-period.dat}

\vspace{5mm}
\subsubsection{Explanation of Variables}
\begin{description}
\input{./mf6ivar/tex/gwt-lkt-desc.tex}
\end{description}

\vspace{5mm}
\subsubsection{Example Input File}
\lstinputlisting[style=inputfile]{./mf6ivar/examples/gwt-lkt-example.dat}

\vspace{5mm}
\subsubsection{Available observation types}
Lake Transport Package observations include lake concentration and all of the terms that contribute to the continuity equation for each lake. Additional LKT Package observations include mass flow rates for individual outlets, lakes, or groups of lakes (\texttt{outlet}). The data required for each LKT Package observation type is defined in table~\ref{table:gwt-lktobstype}. Negative and positive values for \texttt{lkt} observations represent a loss from and gain to the GWT model, respectively. For all other flow terms, negative and positive values represent a loss from and gain from the LKT package, respectively.

\begin{longtable}{p{2cm} p{2.75cm} p{2cm} p{1.25cm} p{7cm}}
\caption{Available LKT Package observation types} \tabularnewline

\hline
\hline
\textbf{Stress Package} & \textbf{Observation type} & \textbf{ID} & \textbf{ID2} & \textbf{Description} \\
\hline
\endfirsthead

\captionsetup{textformat=simple}
\caption*{\textbf{Table \arabic{table}.}{\quad}Available LKT Package observation types.---Continued} \tabularnewline

\hline
\hline
\textbf{Stress Package} & \textbf{Observation type} & \textbf{ID} & \textbf{ID2} & \textbf{Description} \\
\hline
\endhead


\hline
\endfoot

\input{../Common/gwt-lktobs.tex}
\label{table:gwt-lktobstype}
\end{longtable}

\vspace{5mm}
\subsubsection{Example Observation Input File}
\lstinputlisting[style=inputfile]{./mf6ivar/examples/gwt-lkt-example-obs.dat}




\newpage
\subsection{Multi-Aquifer Well Transport (MWT) Package}
Multi-Aquifer Well Transport (MWT) Package information is read from the file that is specified by ``MWT6'' as the file type.  There can be as many MWT Packages as necessary for a GWT model. Each MWT Package is designed to work with flows from a corresponding GWF MAW Package. By default \mf uses the MWT package name to determine which MAW Package corresponds to the MWT Package.  Therefore, the package name of the MWT Package (as specified in the GWT name file) must match with the name of the corresponding MAW Package (as specified in the GWF name file).  Alternatively, the name of the flow package can be specified using the FLOW\_PACKAGE\_NAME keyword in the options block.  The GWT MWT Package cannot be used without a corresponding GWF MAW Package.

The MWT Package does not have a dimensions block; instead, dimensions for the MWT Package are set using the dimensions from the corresponding MAW Package.  For example, the MAW Package requires specification of the number of wells (NMAWWELLS).  MWT sets the number of wells equal to NMAWWELLS.  Therefore, the PACKAGEDATA block below must have NMAWWELLS entries in it.

\vspace{5mm}
\subsubsection{Structure of Blocks}
\lstinputlisting[style=blockdefinition]{./mf6ivar/tex/gwt-mwt-options.dat}
\lstinputlisting[style=blockdefinition]{./mf6ivar/tex/gwt-mwt-packagedata.dat}
\lstinputlisting[style=blockdefinition]{./mf6ivar/tex/gwt-mwt-period.dat}

\vspace{5mm}
\subsubsection{Explanation of Variables}
\begin{description}
\input{./mf6ivar/tex/gwt-mwt-desc.tex}
\end{description}

\vspace{5mm}
\subsubsection{Example Input File}
\lstinputlisting[style=inputfile]{./mf6ivar/examples/gwt-mwt-example.dat}

\vspace{5mm}
\subsubsection{Available observation types}
Multi-Aquifer Well Transport Package observations include well concentration and all of the terms that contribute to the continuity equation for each well. Additional MWT Package observations include mass flow rates for individual wells, or groups of wells; the well volume (\texttt{volume}); and the conductance for a well-aquifer connection conductance (\texttt{conductance}). The data required for each MWT Package observation type is defined in table~\ref{table:gwt-mwtobstype}. Negative and positive values for \texttt{mwt} observations represent a loss from and gain to the GWT model, respectively. For all other flow terms, negative and positive values represent a loss from and gain from the MWT package, respectively.

\begin{longtable}{p{2cm} p{2.75cm} p{2cm} p{1.25cm} p{7cm}}
\caption{Available MWT Package observation types} \tabularnewline

\hline
\hline
\textbf{Stress Package} & \textbf{Observation type} & \textbf{ID} & \textbf{ID2} & \textbf{Description} \\
\hline
\endfirsthead

\captionsetup{textformat=simple}
\caption*{\textbf{Table \arabic{table}.}{\quad}Available MWT Package observation types.---Continued} \tabularnewline

\hline
\hline
\textbf{Stress Package} & \textbf{Observation type} & \textbf{ID} & \textbf{ID2} & \textbf{Description} \\
\hline
\endhead


\hline
\endfoot

\input{../Common/gwt-mwtobs.tex}
\label{table:gwt-mwtobstype}
\end{longtable}

\vspace{5mm}
\subsubsection{Example Observation Input File}
\lstinputlisting[style=inputfile]{./mf6ivar/examples/gwt-mwt-example-obs.dat}




\newpage
\subsection{Unsaturated Zone Transport (UZT) Package}
Unsaturated Zone Transport (UZT) Package information is read from the file that is specified by ``UZT6'' as the file type.  There can be as many UZT Packages as necessary for a GWT model. Each UZT Package is designed to work with flows from a corresponding GWF UZF Package. By default \mf uses the UZT package name to determine which UZF Package corresponds to the UZT Package.  Therefore, the package name of the UZT Package (as specified in the GWT name file) must match with the name of the corresponding UZF Package (as specified in the GWF name file).  Alternatively, the name of the flow package can be specified using the FLOW\_PACKAGE\_NAME keyword in the options block.  The GWT UZT Package cannot be used without a corresponding GWF UZF Package.

The UZT Package does not have a dimensions block; instead, dimensions for the UZT Package are set using the dimensions from the corresponding UZF Package.  For example, the UZF Package requires specification of the number of cells (NUZFCELLS).  UZT sets the number of UZT cells equal to NUZFCELLS.  Therefore, the PACKAGEDATA block below must have NUZFCELLS entries in it.

\vspace{5mm}
\subsubsection{Structure of Blocks}
\lstinputlisting[style=blockdefinition]{./mf6ivar/tex/gwt-uzt-options.dat}
\lstinputlisting[style=blockdefinition]{./mf6ivar/tex/gwt-uzt-packagedata.dat}
\lstinputlisting[style=blockdefinition]{./mf6ivar/tex/gwt-uzt-period.dat}

\vspace{5mm}
\subsubsection{Explanation of Variables}
\begin{description}
\input{./mf6ivar/tex/gwt-uzt-desc.tex}
\end{description}

\vspace{5mm}
\subsubsection{Example Input File}
\lstinputlisting[style=inputfile]{./mf6ivar/examples/gwt-uzt-example.dat}

\vspace{5mm}
\subsubsection{Available observation types}
Unsaturated Zone Transport Package observations include UZF cell concentration and all of the terms that contribute to the continuity equation for each UZF cell. Additional UZT Package observations include mass flow rates for individual UZF cells, or groups of UZF cells. The data required for each UZT Package observation type is defined in table~\ref{table:gwt-uztobstype}. Negative and positive values for \texttt{uzt} observations represent a loss from and gain to the GWT model, respectively. For all other flow terms, negative and positive values represent a loss from and gain from the UZT package, respectively.

\begin{longtable}{p{2cm} p{2.75cm} p{2cm} p{1.25cm} p{7cm}}
\caption{Available UZT Package observation types} \tabularnewline

\hline
\hline
\textbf{Stress Package} & \textbf{Observation type} & \textbf{ID} & \textbf{ID2} & \textbf{Description} \\
\hline
\endfirsthead

\captionsetup{textformat=simple}
\caption*{\textbf{Table \arabic{table}.}{\quad}Available UZT Package observation types.---Continued} \tabularnewline

\hline
\hline
\textbf{Stress Package} & \textbf{Observation type} & \textbf{ID} & \textbf{ID2} & \textbf{Description} \\
\hline
\endhead


\hline
\endfoot

\input{../Common/gwt-uztobs.tex}
\label{table:gwt-uztobstype}
\end{longtable}

\vspace{5mm}
\subsubsection{Example Observation Input File}
\lstinputlisting[style=inputfile]{./mf6ivar/examples/gwt-uzt-example-obs.dat}




\newpage
\subsection{Flow Model Interface (FMI) Package}
Flow Model Interface (FMI) Package information is read from the file that is specified by ``FMI6'' as the file type.  The FMI Package is optional, but if provided, only one FMI Package can be specified for a GWT model.

For most simulations, the GWT Model needs groundwater flows for every cell in the model grid, for all boundary conditions, and for other terms, such as the flow of water in or out of storage.  The FMI Package is the interface between the GWT Model and simulated groundwater flows provided by a corresponding GWF Model that is running concurrently within the simulation or from binary budget files that were created from a previous GWF model run.  The following are several different FMI simulation cases:

\begin{itemize}

\item Flows are provided by a corresponding GWF Model running in the same simulation---in this case, all groundwater flows are calculated by the corresponding GWF Model and provided through FMI to the transport model.  This is a common use case in which the user wants to run the flow and transport models as part of a single simulation.  The GWF and GWT models must be part of a GWF-GWT Exchange that is listed in mfsim.nam.  If a GWF-GWT Exchange is specified by the user, then the user does not need to specify an FMI Package input file for the simulation, unless an FMI option is needed.  If a GWF-GWT Exchange is specified and the FMI Package is specified, then the PACKAGEDATA block below is not read or used.

\item There is no groundwater flow and the user is interested only in the effects of diffusion, sorption, and decay or production---in this case, FMI should not be provided in the GWT name file and the GWT model should not be listed in any GWF-GWT Exchanges in mfsim.nam.  In this case, all groundwater flows are assumed to be zero and cells are assumed to be fully saturated.  The SSM Package should not be activated in this case, because there can be no sources or sinks of water.  Likewise, none of the advanced transport packages (LKT, SFT, MWT, and UZT) should be specified in the GWT name file.  This type of model simulation without an FMI Package is included as an option to represent diffusion, sorption, and decay or growth in the absence of any groundwater flow.

\item Flows are provided from a previous GWF model simulation---in this case FMI should be provided in the GWT name file and the head and budget files should be listed in the FMI options block.  In this case, FMI reads the simulated head and flows from these files and makes them available to the transport model.  There are some additional considerations when the heads and flows are provided from binary files.

\begin{itemize}
\item The binary budget file must contain the simulated flows for all of the packages that were included in the GWF model run.  Saving of flows can be activated for all packages by specifying ``SAVE\_FLOWS'' as an option in the GWF name file.  The GWF Output Control Package must also have ``SAVE BUGET ALL'' specified.  The easiest way to ensure that all flows and heads are saved is to use the following simple form of a GWF Output Control file:

\begin{verbatim}
BEGIN OPTIONS
  HEAD FILEOUT mymodel.hds
  BUDGET FILEOUT mymodel.bud
END OPTIONS

BEGIN PERIOD 1
  SAVE HEAD ALL
  SAVE BUDGET ALL
END PERIOD
\end{verbatim}

\item The binary budget file must have the same number of budget terms listed for each time step.  This will always be the case when the binary budget file is created by \mfdot
\item The advanced flow packages (LAK, SFR, MAW, and UZF) all have options for saving a detailed budget file the describes all of the flows for each lake, reach, well, or UZF cell.  These budget files can also be used as input to FMI if a corresponding advanced transport package is needed, such as LKT, SFT, MWT, and UZT.  If the Water Mover Package is also specified for the GWF Model, then the the budget file for the Water Mover Package will also need to be specified as input to this FMI Package.
\item The binary heads file must have heads saved for all layers in the model.  This will always be the case when the binary head file is created by \mfdot  This was not always the case as previous MODFLOW versions allowed different save options for each layer.
\item If the binary budget and head files have more than one time step for a single stress period, then the budget and head information must be contained within the binary file for every time step in the simulation stress period.
\item The binary budget and head files must correspond in terms of information stored for each time step and stress period.
\item If the binary budget and head files have information provided for only the first time step of each stress period, then this information will be used for all time steps in the GWT model run for that stress period.  This makes it possible to provide flows, for example, from a steady state GWF stress period and have those flows used for all steps in the GWT simulation.  With this option, it is possible to have smaller time steps in the GWT model than the time steps used in the GWF model.  Note that this cannot be done when the GWF and GWT models are run in the same simulation, because in that case, both models are solved for each time step in the stress period, as listed in the TDIS Package.  This option for reading flows from a previous GWF simulation may offer an efficient alternative to running both models in the same simulation, but it comes at the cost of having potentially very large budget files.
\end{itemize}

\end{itemize}

\noindent Determination of which FMI use case to invoke requires careful consideration of the different advantages and disadvantages of each case.  For example, running GWT and GWF in the same simulation can often be faster because GWF flows are passed through memory to the GWT model instead of being written to files.  The disadvantage of this approach is that the same time step lengths must be used for both GWF and GWT.  Ultimately, it should be relatively straightforward to test different ways in which GWF and GWT interact and select the use case most appropriate for the particular problem. 

\vspace{5mm}
\subsubsection{Structure of Blocks}
\lstinputlisting[style=blockdefinition]{./mf6ivar/tex/gwt-fmi-options.dat}
\lstinputlisting[style=blockdefinition]{./mf6ivar/tex/gwt-fmi-packagedata.dat}

\vspace{5mm}
\subsubsection{Explanation of Variables}
\begin{description}
\input{./mf6ivar/tex/gwt-fmi-desc.tex}
\end{description}

\vspace{5mm}
\subsubsection{Example Input File}
\lstinputlisting[style=inputfile]{./mf6ivar/examples/gwt-fmi-example.dat}



\newpage
\subsection{Mover Transport (MVT) Package}
Mover Transport (MVT) Package information is read from the file that is specified by ``MVT6'' as the file type.  Only one MVT Package can be specified for a GWT model.  

The MVT Package is used to route solute mass according to flows from the GWF Water Mover (MVR) Package.  This MVT Package must be activated by the user if the MVR Package was active for the GWF Model.  Flows from the GWF MVR Package must be available to the GWT model either through activation of a GWF-GWT Exchange or through specification of ``GWFMOVER'' in the PACKAGEDATA block of the GWT FMI Package.  

\vspace{5mm}
\subsubsection{Structure of Blocks}
\lstinputlisting[style=blockdefinition]{./mf6ivar/tex/gwt-mvt-options.dat}

\vspace{5mm}
\subsubsection{Explanation of Variables}
\begin{description}
\input{./mf6ivar/tex/gwt-mvt-desc.tex}
\end{description}

\vspace{5mm}
\subsubsection{Example Input File}
\lstinputlisting[style=inputfile]{./mf6ivar/examples/gwt-mvt-example.dat}





%Sparse Matrix Solution (IMS)
\newpage
\SECTION{Iterative Model Solution}
\input{ims.tex}

%OBS Utility Input Instructions
\newpage
\SECTION{Observation (OBS) Utility}
For consistency with earlier versions of MODFLOW (specifically, MODFLOW-2000 and MODFLOW-2005), \programname{} supports an ``Observation'' utility. Unlike the earlier versions of MODFLOW, the Observation utility of \programname{} does not require input of ``observed'' values, which typically were field- or lab-measured values. The Observation utility described here provides options for extracting numeric values of interest generated in the course of a model run. The Observation utility does not calculate residual values (differences between observed and model-calculated values). Output generated by the Observation utility is designed to facilitate further processing. For convenience and for consistency with earlier terminology, individual entries of the Observation utility are referred to as ``observations.''

Input for the Observation utility is read from one or more input files, where each file is associated with a specific model or package. For extracting values simulated by a GWF model, input is read from a file that is specified as type ``OBS6'' in the Name File. For extracting model values associated with a package, input is read from a file designated by the keyword ``OBS6'' in the Options block of the package of interest. The structures of observation input files for models and packages do not differ. Where a file name (or path name) containing spaces is to be read, enclose the name in single quotation marks.

Each OBS6 file can contain an OPTIONS block and one or more CONTINUOUS blocks. Each OBS6 file must contain at least one block. If present, the OPTIONS block must appear first. The CONTINUOUS blocks can be listed in any order. Comments, indicated by the presence of the ``\#'' character in column 1, can appear anywhere in the file and are ignored. 

Observations are output at the end of each time step and represent the value used by \mf during the time step. When input to the OBS utility references a stress-package boundary (for packages other than the advanced stress packages) that is not defined for a stress period of interest, a special NODATA value, indicating that a simulated value is not available, is written to output. The NODATA value is $3.0 \times 10\textsuperscript{30}$. 

Output files to be generated by the Observation utility can be either text or binary. When a text file is used for output, the user can specify the number of digits of precision are to be used in writing values. For compatibility with common spreadsheet programs, text files are written in Comma-Separated Values (CSV) format. For this reason, text output files are commonly named with ``csv'' as the extension. By convention, binary output files are named with ``bsv'' (for ``binary simulated values'') as the extension.

%When a binary file is used, the user can specify whether floating-point numbers should be written in single or double precision.

%For CONTINUOUS observations, note that boundaries identified by ID (and ID2 where used) must be defined in the corresponding package input file in all stress periods of the simulation. This requirement may mean that in some PERIOD blocks, the user will need to include entries that have no affect on the model; for example one could include a well with a recharge rate of zero or a drain boundary with a conductance of zero. In some situations preparation of input can be simplified by splitting package input into multiple input files, so that boundaries included in CONTINUOUS observations are separated from other boundaries simulated by the same package type.

\subsection{Structure of Blocks}
\vspace{5mm}

\noindent \textit{FOR EACH SIMULATION}
\lstinputlisting[style=blockdefinition]{./mf6ivar/tex/utl-obs-options.dat}
\lstinputlisting[style=blockdefinition]{./mf6ivar/tex/utl-obs-continuous.dat}

\subsection{Explanation of Variables}
\begin{description}
% DO NOT MODIFY THIS FILE DIRECTLY.  IT IS CREATED BY mf6ivar.py 

\item \textbf{Block: OPTIONS}

\begin{description}
\item \texttt{digits}---Keyword and an integer digits specifier used for conversion of simulated values to text on output. The default is 5 digits. When simulated values are written to a file specified as file type DATA in the Name File, the digits specifier controls the number of significant digits with which simulated values are written to the output file. The digits specifier has no effect on the number of significant digits with which the simulation time is written for continuous observations.

\item \texttt{PRINT\_INPUT}---keyword to indicate that the list of observation information will be written to the listing file immediately after it is read.

\end{description}
\item \textbf{Block: CONTINUOUS}

\begin{description}
\item \texttt{FILEOUT}---keyword to specify that an output filename is expected next.

\item \texttt{obs\_output\_file\_name}---Name of a file to which simulated values corresponding to observations in the block are to be written. The file name can be an absolute or relative path name. A unique output file must be specified for each CONTINUOUS block. If the ``BINARY'' option is used, output is written in binary form. By convention, text output files have the extension ``csv'' (for ``Comma-Separated Values'') and binary output files have the extension ``bsv'' (for ``Binary Simulated Values'').

\item \texttt{BINARY}---an optional keyword used to indicate that the output file should be written in binary (unformatted) form.

\item \texttt{obsname}---string of 1 to 40 nonblank characters used to identify the observation. The identifier need not be unique; however, identification and post-processing of observations in the output files are facilitated if each observation is given a unique name.

\item \texttt{obstype}---a string of characters used to identify the observation type.

\item \texttt{id}---Text identifying cell where observation is located. For packages other than NPF, if boundary names are defined in the corresponding package input file, ID can be a boundary name. Otherwise ID is a cellid. If the model discretization is type DIS, cellid is three integers (layer, row, column). If the discretization is DISV, cellid is two integers (layer, cell number). If the discretization is DISU, cellid is one integer (node number).

\item \texttt{id2}---Text identifying cell adjacent to cell identified by ID. The form of ID2 is as described for ID. ID2 is used for intercell-flow observations of a GWF model, for three observation types of the LAK Package, for two observation types of the MAW Package, and one observation type of the UZF Package.

\end{description}


\end{description}


\subsection{Available Observation Types}

\subsubsection{GWF Observations}
Observations are available for GWF models, GWF-GWF exchanges, and all stress packages. Available observation types have been listed for each package that supports observations (tables~\ref{table:gwfobstype} to~\ref{table:gwf-gwfobstype}). All available observation types are repeated in Table~\ref{table:gwf-obstypetable} for convenience. 

The sign convention adopted for flow observations are identical to the conventions used in budgets contained in listing files and used in the cell-by-cell budget output. For flow-ja-face observation types, negative and positive values represent a loss from and gain to the cellid specified for ID, respectively. For standard stress packages (Package = CHD, DRN, EVT, GHB, RCH, RIV, and WEL), negative and positive values represent a loss from and gain to the GWF model, respectively. For advanced packages (Package = LAK, MAW, SFR, and UZF), negative and positive values for exchanges with the GWF model (Observation type = lak, maw, sfr, uzf-gwrch, uzf-gwd, uzf-gwd-to-mvr, and uzf-gwet) represent a loss from and gain to the GWF model, respectively. For other advanced stress package flow terms, negative and positive values represent a loss from and gain from the advanced package, respectively.

\FloatBarrier
\input{../Common/gwf-obstypetable}
\FloatBarrier


\subsubsection{GWT Observations}
Observations are available for GWT models and GWT stress packages. Available observation types have been listed for each package that supports observations (tables~\ref{table:gwtobstype} to~\ref{table:gwt-uztobstype}). All available observation types are repeated in Table~\ref{table:gwt-obstypetable} for convenience. 

The sign convention adopted for transport observations are identical to the conventions used in budgets contained in listing files and used in the cell-by-cell budget output. For flow-ja-face observation types, negative and positive values represent a loss from and gain to the cellid specified for ID, respectively. For standard stress packages, negative and positive values represent a loss from and gain to the GWT model, respectively. For advanced transport packages (Package = LKT, MWT, SFT, and UZT), negative and positive values for exchanges with the GWT model (Observation type = lkt, mwt, sft, and uzt) represent a loss from and gain to the GWT model, respectively. For other advanced stress package flow terms, negative and positive values represent a loss from and gain from the advanced package, respectively.

\FloatBarrier
\input{../Common/gwt-obstypetable}
\FloatBarrier




%Time-variable input
\newpage
\SECTION{Time-Variable Input}
\input{gwf/tsi}

%Binary files
\newpage
\SECTION{Description of Binary Output Files for the Groundwater Flow (GWF) and Groundwater Transport (GWT) Models }
Users can optionally write \mf~output to binary files.  There are several different types of binary output files.  The first type is new to MODFLOW and is called a binary grid file.  The binary grid file contains all of the information necessary for a post-processing program to quickly reconstruct the the model grid and understand how cells are connected within the grid.  The option to specify an IDOMAIN array for DIS and DISV grids may result in cells being connected across model layers.  For this reason, cell connectivity information is written to the binary grid file. The second type of binary file is one that contains simulated results, such as head.  Simulated flows are written to a third type of binary file, called a budget file.  The budget file contains simulated flows between connected cells and flows from stress packages.  Lastly, observations can also be written to binary output files.

All floating point variables are written to the binary output files as DOUBLE PRECISION Fortran variables. Integer variables are written to the output files as Fortran integer variables. Some variables are character strings and are indicated as so in the following descriptions.

The file formats for the binary files are described in the following sections. The frequency of output and the types of output files that are created is described in the Output Control Option and in the individual package input files.

\newpage
\subsection{Binary Grid File}
\mf~writes a binary grid file that can be used for post processing model results.  The file structure was designed to be self-documenting so that it can evolve if necessary.  The file name is assigned automatically by the program by adding ``.grb'' to the end of the discretization input file name.  The structure of the binary grid file depends on the type of discretization package that is used.  The following subsections summarize the binary grid file for the different grid types.  The red text is not written to the binary grid file, but is shown here to explain the file content.

\newpage
\subsubsection{DIS Grids}

\vspace{5mm}
\noindent Header 1: \texttt{`GRID DIS'}  {\color{red} \footnotesize{CHARACTER(LEN=50)}} \\
\noindent Header 2: \texttt{`VERSION 1'}  {\color{red} \footnotesize{CHARACTER(LEN=50)}} \\
\noindent Header 3: \texttt{`NTXT 16'} {\color{red} \footnotesize{CHARACTER(LEN=50)}}\\
\noindent Header 4: \texttt{`LENTXT 100'} {\color{red} \footnotesize{CHARACTER(LEN=50)}}\\

\vspace{5mm}
\noindent Read \texttt{NTXT} strings of size \texttt{LENTXT}. Set the number of data records (\texttt{NDAT}) equal to number of lines that do not begin with \#.  \\
\noindent Definition 0: \texttt{`\#Comment ...'} {\color{red} \footnotesize{CHARACTER(LEN=LENTXT)}, comments not presently written} \\
\noindent Definition 1: \texttt{`NCELLS INTEGER NDIM 0 \# ncells'} {\color{red} \footnotesize{CHARACTER(LEN=LENTXT)}} \\
\noindent Definition 2: \texttt{`NLAY INTEGER NDIM 0 \# nlay'} {\color{red} \footnotesize{CHARACTER(LEN=LENTXT)}} \\
\noindent Definition 3: \texttt{`NROW INTEGER NDIM 0 \# nrow'} {\color{red} \footnotesize{CHARACTER(LEN=LENTXT)}} \\
\noindent Definition 4: \texttt{`NCOL INTEGER NDIM 0 \# ncol'} {\color{red} \footnotesize{CHARACTER(LEN=LENTXT)}} \\
\noindent Definition 5: \texttt{`NJA INTEGER NDIM 0 \# nja'} {\color{red} \footnotesize{CHARACTER(LEN=LENTXT)}} \\
\noindent Definition 6: \texttt{`XORIGIN DOUBLE NDIM 0 \# xorigin'} {\color{red} \footnotesize{CHARACTER(LEN=LENTXT)}} \\
\noindent Definition 7: \texttt{`YORIGIN DOUBLE NDIM 0 \# yorigin'} {\color{red} \footnotesize{CHARACTER(LEN=LENTXT)}} \\
\noindent Definition 8: \texttt{`ANGROT DOUBLE NDIM 0 \# angrot'} {\color{red} \footnotesize{CHARACTER(LEN=LENTXT)}} \\
\noindent Definition 9: \texttt{`DELR DOUBLE NDIM 1 ncol'} {\color{red} \footnotesize{CHARACTER(LEN=LENTXT)}} \\
\noindent Definition 10: \texttt{`DELC DOUBLE NDIM 1 nrow'} {\color{red} \footnotesize{CHARACTER(LEN=LENTXT)}} \\
\noindent Definition 11: \texttt{`TOP DOUBLE NDIM 1 nrow*ncol'} {\color{red} \footnotesize{CHARACTER(LEN=LENTXT)}} \\
\noindent Definition 12: \texttt{`BOTM DOUBLE NDIM 1 ncells'} {\color{red} \footnotesize{CHARACTER(LEN=LENTXT)}} \\
\noindent Definition 13: \texttt{`IA INTEGER NDIM 1 ncells+1'} {\color{red} \footnotesize{CHARACTER(LEN=LENTXT)}} \\
\noindent Definition 14: \texttt{`JA INTEGER NDIM 1 nja'} {\color{red} \footnotesize{CHARACTER(LEN=LENTXT)}} \\
\noindent Definition 15: \texttt{`IDOMAIN INTEGER NDIM 1 ncells'} {\color{red} \footnotesize{CHARACTER(LEN=LENTXT)}} \\
\noindent Definition 16: \texttt{`ICELLTYPE INTEGER NDIM 1 ncells'} {\color{red} \footnotesize{CHARACTER(LEN=LENTXT)}} \\

\vspace{5mm}
\noindent Read \texttt{NDAT} data variables using the definitions defined above. \\
\noindent Record 1: \texttt{NCELLS} {\color{red} \footnotesize{INTEGER}} \\
\noindent Record 2: \texttt{NLAY} {\color{red} \footnotesize{INTEGER}} \\
\noindent Record 3: \texttt{NROW} {\color{red} \footnotesize{INTEGER}} \\
\noindent Record 4: \texttt{NCOL} {\color{red} \footnotesize{INTEGER}} \\
\noindent Record 5: \texttt{NJA} {\color{red} \footnotesize{INTEGER}} \\
\noindent Record 6: \texttt{XORIGIN} {\color{red} \footnotesize{DOUBLE}} \\
\noindent Record 7: \texttt{YORIGIN} {\color{red} \footnotesize{DOUBLE}} \\
\noindent Record 8: \texttt{ANGROT} {\color{red} \footnotesize{DOUBLE}} \\
\noindent Record 9: \texttt{DELR} {\color{red} \footnotesize{DOUBLE PRECISION ARRAY SIZE(NCOL)}} \\
\noindent Record 10: \texttt{DELC} {\color{red} \footnotesize{DOUBLE PRECISION ARRAY SIZE (NROW)}} \\
\noindent Record 11: \texttt{(TOP(J),J=1,NROW*NCOL)} {\color{red} \footnotesize{DOUBLE PRECISION ARRAY SIZE(NROW*NCOL)}} \\
\noindent Record 12: \texttt{(BOTM(J),J=1,NCELLS)} {\color{red} \footnotesize{DOUBLE PRECISION ARRAY SIZE(NCELLS)}} \\
\noindent Record 13: \texttt{(IA(J),J=1,NCELLS+1)} {\color{red} \footnotesize{INTEGER ARRAY SIZE(NCELLS+1)}} \\
\noindent Record 14: \texttt{(JA(J),J=1,NJA)} {\color{red} \footnotesize{INTEGER ARRAY SIZE(NJA)}} \\
\noindent Record 15: \texttt{(IDOMAIN(J),J=1,NCELLS)} {\color{red} \footnotesize{INTEGER ARRAY SIZE(NCELLS)}} \\
\noindent Record 16: \texttt{(ICELLTYPE(J),J=1,NCELLS)} {\color{red} \footnotesize{INTEGER ARRAY SIZE(NCELLS)}} \\

\newpage
\subsubsection{DISV Grids}

The binary grid file for DISV grids contains information on the vertices and which vertices comprise a cell.  The x, y coordinates for each vertex are stored in the VERTICES array.  The list of vertices that comprise all of the cells is stored in the JAVERT array.  The list of vertices for any cell can be found using the IAVERT array.  The following pseudocode shows how to loop through every cell in the DISV grid and obtain the cell vertices.  The list of vertices is ``closed'' for each cell in that the first listed vertex is equal to the last listed vertex.  

\begin{verbatim}
DO K = 1, NLAY
  DO N = 1, NCPL
    PRINT *, 'THIS IS CELL (LAYER, ICELL2D): ', K, N
    NVCELL = IAVERT(N+1) - IAVERT(N)
    PRINT*, 'NUMBER OF VERTICES FOR CELL IS', NVCELL
    DO IPOS = IAVERT(N), IAVERT(N + 1) - 1
      IVERT = JAVERT(IPOS)
      X = VERTICES(1,IVERT)
      Y = VERTICES(2,IVERT)
      PRINT *,'  VERTEX PAIR: ', X, Y
    ENDDO
  ENDDO
ENDDO
\end{verbatim}

The IA and JA arrays are also contained in the DISV binary grid file.  These arrays describe the cell connectivity.  Connections in the JA array correspond directly with the FLOW-JA-FACE record that is written to the budget file.

The content of the DISV binary grid file is as follows.

\vspace{5mm}
\noindent Header 1: \texttt{`GRID DISV'}  {\color{red} \footnotesize{CHARACTER(LEN=50)}} \\
\noindent Header 2: \texttt{`VERSION 1'}  {\color{red} \footnotesize{CHARACTER(LEN=50)}} \\
\noindent Header 3: \texttt{`NTXT 20'} {\color{red} \footnotesize{CHARACTER(LEN=50)}}\\
\noindent Header 4: \texttt{`LENTXT 100'} {\color{red} \footnotesize{CHARACTER(LEN=50)}}\\

\vspace{5mm}
\noindent Read \texttt{NTXT} strings of size \texttt{LENTXT}. Set the number of data records (\texttt{NDAT}) equal to number of lines that do not begin with \#.  \\
\noindent Definition 0: \texttt{`\#Comment ...'} {\color{red} \footnotesize{CHARACTER(LEN=LENTXT)}, comments not presently written} \\
\noindent Definition 1: \texttt{`NCELLS INTEGER NDIM 0 \# ncells'} {\color{red} \footnotesize{CHARACTER(LEN=LENTXT)}} \\
\noindent Definition 2: \texttt{`NLAY INTEGER NDIM 0 \# nlay'} {\color{red} \footnotesize{CHARACTER(LEN=LENTXT)}} \\
\noindent Definition 3: \texttt{`NCPL INTEGER NDIM 0 \# ncpl'} {\color{red} \footnotesize{CHARACTER(LEN=LENTXT)}} \\
\noindent Definition 4: \texttt{`NVERT INTEGER NDIM 0 \# nvert'} {\color{red} \footnotesize{CHARACTER(LEN=LENTXT)}} \\
\noindent Definition 5: \texttt{`NJAVERT INTEGER NDIM 0 \# njavert'} {\color{red} \footnotesize{CHARACTER(LEN=LENTXT)}} \\
\noindent Definition 6: \texttt{`NJA INTEGER NDIM 0 \# nja'} {\color{red} \footnotesize{CHARACTER(LEN=LENTXT)}} \\
\noindent Definition 7: \texttt{`XORIGIN DOUBLE NDIM 0 \# xorigin'} {\color{red} \footnotesize{CHARACTER(LEN=LENTXT)}} \\
\noindent Definition 8: \texttt{`YORIGIN DOUBLE NDIM 0 \# yorigin'} {\color{red} \footnotesize{CHARACTER(LEN=LENTXT)}} \\
\noindent Definition 9: \texttt{`ANGROT DOUBLE NDIM 0 \# angrot'} {\color{red} \footnotesize{CHARACTER(LEN=LENTXT)}} \\
\noindent Definition 10: \texttt{`TOP DOUBLE NDIM 1 ncpl'} {\color{red} \footnotesize{CHARACTER(LEN=LENTXT)}} \\
\noindent Definition 11: \texttt{`BOTM DOUBLE NDIM 1 ncells'} {\color{red} \footnotesize{CHARACTER(LEN=LENTXT)}} \\
\noindent Definition 12: \texttt{`VERTICES DOUBLE NDIM 2 2 nvert'} {\color{red} \footnotesize{CHARACTER(LEN=LENTXT)}} \\
\noindent Definition 13: \texttt{`CELLX DOUBLE NDIM 1 ncpl'} {\color{red} \footnotesize{CHARACTER(LEN=LENTXT)}} \\
\noindent Definition 14: \texttt{`CELLY DOUBLE NDIM 1 ncpl'} {\color{red} \footnotesize{CHARACTER(LEN=LENTXT)}} \\
\noindent Definition 15: \texttt{`IAVERT INTEGER NDIM 1 ncpl+1'} {\color{red} \footnotesize{CHARACTER(LEN=LENTXT)}} \\
\noindent Definition 16: \texttt{`JAVERT INTEGER NDIM 1 njavert'} {\color{red} \footnotesize{CHARACTER(LEN=LENTXT)}} \\
\noindent Definition 17: \texttt{`IA INTEGER NDIM 1 ncells+1'} {\color{red} \footnotesize{CHARACTER(LEN=LENTXT)}} \\
\noindent Definition 18: \texttt{`JA INTEGER NDIM 1 nja'} {\color{red} \footnotesize{CHARACTER(LEN=LENTXT)}} \\
\noindent Definition 19: \texttt{`IDOMAIN INTEGER NDIM 1 ncells'} {\color{red} \footnotesize{CHARACTER(LEN=LENTXT)}} \\
\noindent Definition 20: \texttt{`ICELLTYPE INTEGER NDIM 1 ncells'} {\color{red} \footnotesize{CHARACTER(LEN=LENTXT)}} \\

\vspace{5mm}
\noindent Read \texttt{NDAT} data variables using the definitions defined above. \\
\noindent Record 1: \texttt{NCELLS} {\color{red} \footnotesize{INTEGER}} \\
\noindent Record 2: \texttt{NLAY} {\color{red} \footnotesize{INTEGER}} \\
\noindent Record 3: \texttt{NCPL} {\color{red} \footnotesize{INTEGER}} \\
\noindent Record 4: \texttt{NVERT} {\color{red} \footnotesize{INTEGER}} \\
\noindent Record 5: \texttt{NJAVERT} {\color{red} \footnotesize{INTEGER}} \\
\noindent Record 6: \texttt{NJA} {\color{red} \footnotesize{INTEGER}} \\
\noindent Record 7: \texttt{XORIGIN} {\color{red} \footnotesize{DOUBLE}} \\
\noindent Record 8: \texttt{YORIGIN} {\color{red} \footnotesize{DOUBLE}} \\
\noindent Record 9: \texttt{ANGROT} {\color{red} \footnotesize{DOUBLE}} \\
\noindent Record 10: \texttt{(TOP(J),J=1,NCPL)} {\color{red} \footnotesize{DOUBLE PRECISION ARRAY SIZE(NCPL)}} \\
\noindent Record 11: \texttt{((BOTM(J),J=1,NCELLS)} {\color{red} \footnotesize{DOUBLE PRECISION ARRAY SIZE(NCELLS)}} \\
\noindent Record 12: \texttt{((VERTICES(J,K),J=1,2),K=1,NVERT)} {\color{red} \footnotesize{DOUBLE PRECISION ARRAY SIZE(2,NVERT)}} \\
\noindent Record 13: \texttt{(CELLX(J),J=1,NCPL)} {\color{red} \footnotesize{DOUBLE PRECISION ARRAY SIZE(NCPL)}}\\
\noindent Record 14: \texttt{(CELLY(J),J=1,NCPL)} {\color{red} \footnotesize{DOUBLE PRECISION ARRAY SIZE(NCPL)}} \\
\noindent Record 15: \texttt{(IAVERT(J),J=1,NCPL+1)} {\color{red} \footnotesize{INTEGER ARRAY SIZE(NCPL+1)}} \\
\noindent Record 16: \texttt{(JAVERT(J),J=1,NJAVERT)} {\color{red} \footnotesize{INTEGER ARRAY SIZE(NJAVERT)}} \\
\noindent Record 17: \texttt{(IA(J),J=1,NCELLS+1)} {\color{red} \footnotesize{INTEGER ARRAY SIZE(NCELLS+1)}} \\
\noindent Record 18: \texttt{(JA(J),J=1,NJA)} {\color{red} \footnotesize{INTEGER ARRAY SIZE(NJA)}} \\
\noindent Record 19: \texttt{(IDOMAIN(J),J=1,NCELLS)} {\color{red} \footnotesize{INTEGER ARRAY SIZE(NCELLS)}} \\
\noindent Record 20: \texttt{(ICELLTYPE(J),J=1,NCELLS)} {\color{red} \footnotesize{INTEGER ARRAY SIZE(NCELLS)}} \\

\newpage
\subsubsection{DISU Grids}

The binary grid file for DISU grids may contain information on the vertices and which vertices comprise a cell, but this depends on whether or not the user provided the information in the DISU Package.  This information is not required unless the XT3D or SAVE\_SPECIFIC\_DISCHARGE options are specified in the NPF Package.  If provided, the x, y coordinates for each vertex are stored in the VERTICES array.  The list of vertices that comprise all of the cells is stored in the JAVERT array.  The list of vertices for any cell can be found using the IAVERT array.  Pseudocode for looping through cells in the grid is listed above in the section on the binary grid file for the DISV Package.  As for the DISV binary grid file, the list of vertices is ``closed'' for each cell in that the first listed vertex is equal to the last listed vertex.

\vspace{5mm}
\noindent Header 1: \texttt{`GRID DISU'}  {\color{red} \footnotesize{CHARACTER(LEN=50)}} \\
\noindent Header 2: \texttt{`VERSION 1'}  {\color{red} \footnotesize{CHARACTER(LEN=50)}} \\
\noindent Header 3: \texttt{`NTXT 10'} or \texttt{`NTXT 15'} {\color{red} \footnotesize{CHARACTER(LEN=50)}}\\
\noindent Header 4: \texttt{`LENTXT 100'} {\color{red} \footnotesize{CHARACTER(LEN=50)}}\\

\vspace{5mm}
\noindent Read \texttt{NTXT} strings of size \texttt{LENTXT}. Set the number of data records (\texttt{NDAT}) equal to number of lines that do not begin with \#.  \\
\noindent Definition 0: \texttt{`\#Comment ...'} {\color{red} \footnotesize{CHARACTER(LEN=LENTXT)}, comments not presently written} \\
\noindent Definition 1: \texttt{`NODES INTEGER NDIM 0 \# nodes'} {\color{red} \footnotesize{CHARACTER(LEN=LENTXT)}} \\
\noindent Definition 2: \texttt{`NJA INTEGER NDIM 0 \# nja'} {\color{red} \footnotesize{CHARACTER(LEN=LENTXT)}} \\
\noindent Definition 3: \texttt{`XORIGIN DOUBLE NDIM 0 \# xorigin'} {\color{red} \footnotesize{CHARACTER(LEN=LENTXT)}} \\
\noindent Definition 4: \texttt{`YORIGIN DOUBLE NDIM 0 \# yorigin'} {\color{red} \footnotesize{CHARACTER(LEN=LENTXT)}} \\
\noindent Definition 5: \texttt{`ANGROT DOUBLE NDIM 0 \# angrot'} {\color{red} \footnotesize{CHARACTER(LEN=LENTXT)}} \\
\noindent Definition 6: \texttt{`TOP DOUBLE NDIM 1 nodes'} {\color{red} \footnotesize{CHARACTER(LEN=LENTXT)}} \\
\noindent Definition 7: \texttt{`BOT DOUBLE NDIM 1 nodes'} {\color{red} \footnotesize{CHARACTER(LEN=LENTXT)}} \\
\noindent Definition 8: \texttt{`IA INTEGER NDIM 1 ncells+1'} {\color{red} \footnotesize{CHARACTER(LEN=LENTXT)}} \\
\noindent Definition 9: \texttt{`JA INTEGER NDIM 1 nja'} {\color{red} \footnotesize{CHARACTER(LEN=LENTXT)}} \\
\noindent Definition 10: \texttt{`ICELLTYPE INTEGER NDIM 1 ncells'} {\color{red} \footnotesize{CHARACTER(LEN=LENTXT)}} \\

\vspace{5mm}
\noindent If vertices are provided in the DISU Package, then 5 additional definitions are included: \\
\noindent Definition 11: \texttt{`VERTICES DOUBLE NDIM 2 2 nvert'} {\color{red} \footnotesize{CHARACTER(LEN=LENTXT)}} \\
\noindent Definition 12: \texttt{`CELLX DOUBLE NDIM 1 nodes'} {\color{red} \footnotesize{CHARACTER(LEN=LENTXT)}} \\
\noindent Definition 13: \texttt{`CELLY DOUBLE NDIM 1 nodes'} {\color{red} \footnotesize{CHARACTER(LEN=LENTXT)}} \\
\noindent Definition 14: \texttt{`IAVERT INTEGER NDIM 1 nodes+1'} {\color{red} \footnotesize{CHARACTER(LEN=LENTXT)}} \\
\noindent Definition 15: \texttt{`JAVERT INTEGER NDIM 1 njavert'} {\color{red} \footnotesize{CHARACTER(LEN=LENTXT)}} \\

\vspace{5mm}
\noindent Read \texttt{NDAT} data variables using the definitions defined above. \\
\noindent Record 1: \texttt{NODES} {\color{red} \footnotesize{INTEGER}} \\
\noindent Record 2: \texttt{NJA} {\color{red} \footnotesize{INTEGER}} \\
\noindent Record 3: \texttt{XORIGIN} {\color{red} \footnotesize{DOUBLE}} \\
\noindent Record 4: \texttt{YORIGIN} {\color{red} \footnotesize{DOUBLE}} \\
\noindent Record 5: \texttt{ANGROT} {\color{red} \footnotesize{DOUBLE}} \\
\noindent Record 6: \texttt{(TOP(J),J=1,NODES)} {\color{red} \footnotesize{DOUBLE PRECISION ARRAY SIZE(NODES)}} \\
\noindent Record 7: \texttt{((BOT(J),J=1,NODES)} {\color{red} \footnotesize{DOUBLE PRECISION ARRAY SIZE(NODES)}} \\
\noindent Record 8: \texttt{(IA(J),J=1,NODES+1)} {\color{red} \footnotesize{INTEGER ARRAY SIZE(NODES+1)}} \\
\noindent Record 9: \texttt{(JA(J),J=1,NJA)} {\color{red} \footnotesize{INTEGER ARRAY SIZE(NJA)}} \\
\noindent Record 10: \texttt{(ICELLTYPE(J),J=1,NCELLS)} {\color{red} \footnotesize{INTEGER ARRAY SIZE(NCELLS)}} \\

\vspace{5mm}
\noindent If vertices are provided in the DISU Package, then 5 additional records are included: \\
\noindent Record 11: \texttt{((VERT(J,K),J=1,2),K=1,NVERT)} {\color{red} \footnotesize{DOUBLE PRECISION ARRAY SIZE(2,NVERT)}} \\
\noindent Record 12: \texttt{(CELLX(J),J=1,NODES)} {\color{red} \footnotesize{DOUBLE PRECISION ARRAY SIZE(NODES)}}\\
\noindent Record 13: \texttt{(CELLY(J),J=1,NODES)} {\color{red} \footnotesize{DOUBLE PRECISION ARRAY SIZE(NODES)}} \\
\noindent Record 14: \texttt{(IAVERT(J),J=1,NODES+1)} {\color{red} \footnotesize{INTEGER ARRAY SIZE(NODES+1)}} \\
\noindent Record 15: \texttt{(JAVERT(J),J=1,NJAVERT)} {\color{red} \footnotesize{INTEGER ARRAY SIZE(NJAVERT)}} \\


\newpage
\subsection{Dependent Variable File}
In the present \mf version, the \texttt{TEXT} value is specified as ``HEAD''.  Cells that have been assigned an IDOMAIN value of zero or less are assigned a head value of $1.0$ x $10^{30}$.  Cells that have converted to dry are assigned a dry value of $-1.0$ x $10^{30}$.  The large negative value allows the results from a previous simulation to be used as starting heads for a subsequent simulation.  Cells assigned a large negative value as an initial condition will start the simulation as dry.  Note that the dry value is not used if the Newton-Raphson Formulation is active.  In this case, a dry cell will have a calculated head value that is below or at the bottom of the cell.

\subsubsection{DIS Grids}
For each stress period, time step, and layer for which data are saved to the binary output file, the following two records are written:

\vspace{5mm}
\noindent Record 1: \texttt{KSTP,KPER,PERTIM,TOTIM,TEXT,NCOL,NROW,ILAY} \\
\noindent Record 2: \texttt{((DATA(J,I,ILAY),J=1,NCOL),I=1,NROW)} \\

\vspace{5mm}
\noindent where

\begin{description} \itemsep0pt \parskip0pt \parsep0pt
\item \texttt{KSTP} is the time step number;
\item \texttt{KPER} is the stress period number;
\item \texttt{PERTIM} is the time value for the current stress period; 
\item \texttt{TOTIM} is the total simulation time;
\item \texttt{TEXT} is a character string (character*16);
\item \texttt{NCOL} is the number of columns;
\item \texttt{NROW} is the number of rows;
\item \texttt{ILAY} is the layer number; and
\item \texttt{DATA} is the head data of size (NCOL,NROW,NLAY).
\end{description}

\subsubsection{DISV Grids}
For each stress period, time step, and layer for which data are saved to the binary output file, the following two records are written:

\vspace{5mm}
\noindent Record 1: \texttt{KSTP,KPER,PERTIM,TOTIM,TEXT,NCPL,1,ILAY} \\
\noindent Record 2: \texttt{(DATA(J,ILAY),J=1,NCPL)} \\

\vspace{5mm}
\noindent where

\begin{description} \itemsep0pt \parskip0pt \parsep0pt
\item \texttt{KSTP} is the time step number;
\item \texttt{KPER} is the stress period number;
\item \texttt{PERTIM} is the time value for the current stress period; 
\item \texttt{TOTIM} is the total simulation time;
\item \texttt{TEXT} is a character string (character*16);
\item \texttt{NCPL} is the number of cells per layer;
\item \texttt{ILAY} is the layer number; and
\item \texttt{DATA} is the head data of size (NCPL,NLAY).
\end{description}

\newpage
\subsubsection{DISU Grids}
For each stress period, time step, and layer for which data are saved to the binary output file, the following two records are written:

\vspace{5mm}
\noindent Record 1: \texttt{KSTP,KPER,PERTIM,TOTIM,TEXT,NODES,1,1} \\
\noindent Record 2: \texttt{(DATA(N),N=1,NODES)} \\

\vspace{5mm}
\noindent where

\begin{description} \itemsep0pt \parskip0pt \parsep0pt
\item \texttt{KSTP} is the time step number;
\item \texttt{KPER} is the stress period number;
\item \texttt{PERTIM} is the time value for the current stress period; 
\item \texttt{TOTIM} is the total simulation time;
\item \texttt{TEXT} is a character string (character*16);
\item \texttt{NODES} is the number cells in the model grid;
\item \texttt{DATA} is unstructured head data of size (NODES).
\end{description}

\newpage
\subsubsection{LAK, MAW, and SFR Packages}

\vspace{5mm}
For each stress period, time step, and layer for which data are saved to the binary output file, the following two records are written:

\vspace{5mm}
\noindent Record 1: \texttt{KSTP,KPER,PERTIM,TOTIM,TEXT,MAXBOUND,1,1} \\
\noindent Record 2: \texttt{(DATA(N),N=1,MAXBOUND)} \\

\vspace{5mm}
\noindent where

\begin{description} \itemsep0pt \parskip0pt \parsep0pt
\item \texttt{KSTP} is the time step number;
\item \texttt{KPER} is the stress period number;
\item \texttt{PERTIM} is the time value for the current stress period; 
\item \texttt{TOTIM} is the total simulation time;
\item \texttt{TEXT} is a character string (\texttt{character*16});
\item \texttt{MAXBOUND} is the number advanced boundary items in the package;
\item \texttt{DATA} is unstructured dependent variable data of size (\texttt{MAXBOUND}).
\end{description}


\newpage
\subsection{Groundwater Flow Model Budget File}
The budget file for the GWF Model contains intercell flows, flows due to changes in storage, flows from the stress packages and advanced stress packages, and exchange flows with another model.  The intent of budget file is to contain all flow to and from any cell in the model.  Users must activate saving of flow terms in the Output Control Package and in the individual packages.  

The format for the budget file is different from the formats for previous MODFLOW versions.  Specifically, intercell flows are written in a different manner using a compressed sparse row storage scheme.  The record structure for the stress packages is also different and uses a method code 6, to distinguish it from the five method codes available in previous MODFLOW versions.  The new code 6 indicates that additional text identifiers are present, that auxiliary variables may be present, and that two identifying integer numbers are contained in the list (one for the node number of the GWF Model cell, and the other for an identifier to where the flow is from).  

\subsubsection{Format of Budget File}
The generalized form of the budget file is described so that utilities may be created to read the budget file.  Additional information about the content and the form of the content for different grid types is described in subsequent sections.

\vspace{5mm}
\noindent Record 1: \texttt{KSTP,KPER,TEXT,NDIM1,NDIM2,-NDIM3} \\
\noindent Record 2: \texttt{IMETH,DELT,PERTIM,TOTIM} \\

\begin{description}
\item \texttt{IMETH}=1: \textit{Read 1D array of size NDIM1*NDIM2*NDIM3.}\\
Record 3: \texttt{(DATA(J),J=1,NDIM1*NDIM2*NDIM3)}

\item \texttt{IMETH}=6: \textit{Read text identifiers, auxiliary text labels, and list of information.}\\
Record 3: \texttt{TXT1ID1}\\
Record 4: \texttt{TXT2ID1}\\
Record 5: \texttt{TXT1ID2}\\
Record 6: \texttt{TXT2ID2}\\
Record 7: \texttt{NDAT}\\
Record 8: \texttt{(AUXTXT(N),N=1,NDAT-1)}\\
Record 9: \texttt{NLIST}\\
Record 10: \texttt{((ID1(N),ID2(N),(DATA2D(I,N),I=1,NDAT)),N=1,NLIST)}\\
\end{description}

\noindent where

\begin{description} \itemsep0pt \parskip0pt \parsep0pt
\item \texttt{KSTP} is the integer time step number;
\item \texttt{KPER} is the integer stress period number;
\item \texttt{TEXT} is a character string (character*16) indicating the flow type;
\item \texttt{PERTIM} is the double precision time value for the current stress period; 
\item \texttt{TOTIM} is the double precision total simulation time;
\item \texttt{NDIM1} is the integer size of first dimension; 
\item \texttt{NDIM2} is the integer size of second dimension;
\item \texttt{NDIM3} is the integer size of third dimension;
\item \texttt{IMETH} is an integer code that specifies the form of the remaining data;
\item \texttt{DELT} is the double precision length of the timestep;
\item \texttt{PERTIM} is the double precision time value for the current stress period;
\item \texttt{TOTIM} is the double precision total simulation time;
\item \texttt{DATA} is a double precision array of budget values;
\item \texttt{TXT1ID1} is a character string (character*16) containing the first text identifier for information in ID1;
\item \texttt{TXT2ID1} is a character string (character*16) containing the second text identifier for information in ID1;
\item \texttt{TXT1ID2} is a character string (character*16) containing the model name for information in ID2;
\item \texttt{TXT2ID2} is a character string (character*16) containing the package or model name for information in ID2;
\item \texttt{NDAT} is the number of columns in DATA2D, which is the number of auxiliary values plus 1;
\item \texttt{AUXTXT} is an array of size NDAT - 1 containing character*16 text names for each auxiliary variable;
\item \texttt{NLIST} is the size of the list;
\item \texttt{ID1} is the first identifying number;
\item \texttt{ID2} is the second identifying number, and
\item \texttt{DATA2D} is a double precision 2D array of size (NDAT,NLIST).  The first column in DATA2D is the budget term; any remaining columns are auxiliary variable values.
\end{description}

\subsubsection{Variations for Discretization Types}
The format for the GWF Model budget file is the same no matter what discretization package is used; however, the variables may have different meanings depending on the grid type and the TEXT identifier.  If the TEXT identifier in Record 1 is FLOW-JA-FACE and IMETH is 1, then the DATA array contains intercell flows and is of size NJA.  If the TEXT identifier in Record 1 is something other than FLOW-JA-FACE (STO-SS or STO-SY, for example), then the dimension variables in Record 1 (NDIM1, NDIM2, and NDIM3) provide information about the size of the grid (table \ref{table:ndim}).  

\begin{longtable}{p{3cm} p{3cm} p{3cm} p{3cm}}
\caption{Budget file variations that depend on discretization package type} 
\tabularnewline
\hline
\textbf{Grid or Flow Type} & \textbf{NDIM1} & \textbf{NDIM2} & \textbf{NDIM3} \\
\hline
\endhead
\hline
\endfoot
DIS & NCOL & NROW & NLAY \\
DISV & NCPL & 1 & NLAY \\
DISU & NODES & 1 & 1 \\
FLOW-JA-FACE, IMETH=1 & NJA & 1 & 1 \\
\label{table:ndim}
\end{longtable}

\subsubsection{Budget File Contents}

The type of information that is written to the budget file for a GWF Model depends on the packages used for the model and whether or not the save flags are set.  Table \ref{table:gwfbud} contains a list of the types of information that may be contained in a GWF Model budget file.  In all cases, the flows in table \ref{table:gwfbud} are flows to or a from a GWF Model cell.  As described in the next section, intercell flows are written as FLOW-JA-FACE using IMETH=1.  If the model has an active Storage Package, then STORAGE-SS and STORAGE-SY are written to the budget file using IMETH=1. If the model has an active Skeletal Storage, Compaction, and Subsidence Package, then CSUB-CGELASTIC and CSUB-WATERCOMP are written to the budget file using IMETH=1.

The remaining flow terms in table \ref{table:gwfbud} are all written using IMETH=6.  When IMETH=6 is used, the records contain additional text descriptors and two identifying numbers.  For all records in the GWF Model budget file, TXT1ID1 is the name of the GWF Model and TXT2ID1 is also the name of the GWF Model.  These text identifiers describe what is contained in ID1.  For the GWF Model budget file, ID1 is the cell or node number in the GWF Model grid.  The second set of text identifiers refer to the information in ID2.  Unless noted otherwise in the description in table \ref{table:gwfbud}, TXT1ID2 is the name of the GWF Model, TXT2ID2 is the name of the package, and ID2 is the bound number in the package; for example, this is the first constant head cell, second constant head cell, and so forth.  

\begin{longtable}{p{3.5cm} p{2cm} p{9cm}}
\caption{Types of information that may be contained in the GWF Model budget file} 
\tabularnewline
\hline
\textbf{Flow Type (TEXT)} & \textbf{Method Code (IMETH)} & \textbf{Description} \\
\hline
\endhead
\hline
\endfoot
\texttt{FLOW-JA-FACE} & 1 & intercell flow; array of size(NJA) \\
\texttt{STO-SS} & 1 & confined storage; array of size (NCELLS) \\
\texttt{STO-SY} & 1 & unconfined storage; array of size (NCELLS) \\
\texttt{CSUB-CGELASTIC} & 1 & coarse-grained elastic storage from CSUB Package; array of size (NCELLS) \\
\texttt{CSUB-WATERCOMP} & 1 & water compressibility from CSUB Package; array of size (NCELLS) \\
\texttt{CSUB-ELASTIC} & 6 & interbed elastic storage from CSUB package; list of size(NINTERBEDS) \\
\texttt{CSUB-INELASTIC} & 6 & interbed inelastic storage from CSUB package; list of size(NINTERBEDS) \\
\texttt{CHD} & 6 & constant head flow\\
\texttt{WEL} & 6 & well flow \\
\texttt{WEL-TO-MVR} & 6 & well flow that is routed to Mover Package \\
\texttt{DRN} & 6 & drain flow \\
\texttt{DRN-TO-MVR} & 6 & drain flow that is routed to Mover Package\\
\texttt{RIV} & 6 & river leakage \\
\texttt{RIV-TO-MVR} & 6 & river leakage that is routed to Mover Package\\
\texttt{GHB} & 6 & general-head boundary flow \\
\texttt{GHB-TO-MVR} & 6 & general-head boundary flow that is routed to Mover Package\\
\texttt{RCH} & 6 & recharge flow \\
\texttt{EVT} & 6 & evapotranspiration flow \\
\texttt{MAW} & 6 & multi-aquifer well flow; ID2 contains the well number \\
\texttt{LAK} & 6 & lake leakage; ID2 contains the lake number \\
\texttt{SFR} & 6 & stream leakage; ID2 contains the stream reach number \\
\texttt{UZF-GWRCH} & 6 & water table recharge from UZF Package \\
\texttt{UZF-GWET} & 6 & water table evapotranspiration from UZF Package  \\
\texttt{UZF-GWD} & 6 & groundwater discharge to land surface from UZF Package \\
\texttt{UZF-GWD-TO-MVR} & 6 & groundwater discharge to land surface from UZF Package that is routed to Mover Package\\
\texttt{FLOW-JA-FACE} & 6 & flow to or from a cell in another GWF Model; TXT1ID1 is the name of the GWF Model described by this budget file, TXT2ID1 is the name of the GWF-GWF Exchange, TXT1ID2 is the name of the connected GWF Model, TXT2ID2 is the name of the GWF-GWF Exchange, and ID2 is the cell or node number of the cell in the connected model \\
\texttt{DATA-SPDIS} & 6 & specific discharge at the cell center.  The x, y, and z components are stored in auxiliary variables called ``qx'', ``qy'', and ``qz'', respectively.   The flow value written for each cell is zero.  The ``DATA'' prefix on the text identifier can be used by post-processors to recognize that the record does not contain a cell flow budget term. \\
\texttt{DATA-SAT} & 6 & cell saturation.  The cell saturation is stored in an auxiliary variable called ``sat''.   The flow value written for each cell is zero.  The ``DATA'' prefix on the text identifier can be used by post-processors to recognize that the record does not contain a cell flow budget term.  The cell saturation can be used by post-processors to determine how much of the cell is saturated without having to know the value for ICELLTYPE or the value for head.  If a cell is marked as confined (ICELLTYPE=0) then saturation is always one.  If ICELLTYPE is one, then saturation ranges between zero and one.  For Newton GWF simulations, saturation is zero if the head is below the cell bottom.
\label{table:gwfbud}
\end{longtable}

\subsubsection{Intercell Flows}

\mf writes a special budget record for flow between connected cells. This record has a TEXT identifier equal to FLOW-JA-FACE. For this record, the total number of values is equal to NJA, which is the total number of connections.  For each cell, the number of connections is equal to the number of connections to adjacent cells plus one, to represent the cell itself. Therefore, this budget record corresponds to the JA array. A value of zero is written to the node positions in the FLOW-JA-FACE record.  The JA array that is written in the binary grid corresponds directly to the FLOW-JA-FACE record.

For regular MODFLOW grids, there are no longer records for FLOW RIGHT FACE, FLOW FRONT FACE,  and FLOW LOWER FACE.  Instead, intercell flows are written to the FLOW-JA-FACE record.  Writing FLOW-JA-FACE allows face flows to be specified in straightforward manner, particularly when the IDOMAIN capability is used to remove cells and specify vertical pass-through cells.  

The following pseudocode shows how to loop through and process intercell flows using the IA and JA arrays (which can be read from the binary grid file) and the FLOWJA array, which is written to the budget file.  For a cell (N) that has been eliminated with IDOMAIN, the value for IA(N) and IA(N+1) will be equal, indicating that there are no connections or flows for that cell.

\begin{verbatim}
DO N = 1, NCELLS
  PRINT *, 'THIS IS CELL: ', N
  NCON = IA(N+1) - IA(N) - 1
  IF(NCON<0) NCON=0
  PRINT*, 'NUMBER OF CONNECTED CELLS IS ', NCON
  DO IPOS = IA(N) + 1, IA(N + 1) - 1
    M = JA(IPOS)
    Q = FLOWJA(IPOS)
    PRINT *,'  N M Q: ', N,M,Q
  ENDDO
ENDDO
\end{verbatim}
 
\newpage
\subsubsection{CSUB Package}

\vspace{5mm}
For each stress period, time step, and compaction data type that is saved to the CSUB Package binary output files as \texttt{IMETH=1} budget file type. The compaction data that are written to the CSUB Package binary files are summarized in Tables~\ref{table:binarycsub}.

\begin{longtable}{p{3.5cm} p{2cm} p{9cm}}
	\caption{Data written to the CSUB Package compaction binary output files} 
	\tabularnewline
		\hline
		\textbf{Flow Type (TEXT)} & \textbf{Method Code (IMETH)} & \textbf{Description} \\
		\hline
	\endhead
		\hline
	\endfoot
	\texttt{CSUB-COMPACTION} & 1 & total compaction for cell; array of size (NCELLS) \\
	\texttt{CSUB-INELASTIC} & 1 & inelastic compaction for cell; array of size (NCELLS) \\
	\texttt{CSUB-ELASTIC} & 1 & elastic compaction for cell; array of size (NCELLS) \\
	\texttt{CSUB-INTERBED} & 1 & interbed compaction for cell; array of size (NCELLS) \\
	\texttt{CSUB-COARSE} & 1 & coarse-grained compaction for cell; array of size (NCELLS) \\
	\texttt{CSUB-ZDISPLACE} & 1 & z-displacement for cell; z-displacement of the upper most model cells represents subsidance at land-surface; array of size (NCELLS) \\
	\label{table:binarycsub}
\end{longtable}

\subsubsection{LAK, MAW, SFR, and UZF Packages}

\vspace{5mm}
For each stress period, time step, and data type that is saved to the LAK, MAW, SFR, and UZF Packages binary output files as \texttt{IMETH=6} budget file type. For all advanced packages, \texttt{NDIM1} is equal to the number of nodes, \texttt{NDIM2} is equal to 1, and \texttt{NDIM3} is equal to -1. The data that are written to the LAK, MAW, SFR, and UZF Package binary files are summarized in Tables~\ref{table:binarylak}
 to~\ref{table:binaryuzf}, respectively.


% lake package binary budget output
\begin{longtable}{p{3.5cm} p{2cm} p{3.5cm} p{6.5cm}}
\caption{Data written to the LAK Package binary output file. Flow terms are listed in the order they are written to the LAK Package binary output file} \tabularnewline
\hline
\hline
\textbf{Flow term} & \textbf{IMETH} & \textbf{NDAT / NLIST} & \textbf{Description} \\
\hline
\endfirsthead

\hline
\hline
\textbf{Flow term} & \textbf{IMETH} & \textbf{NDAT / NLIST} & \textbf{Description} \\
\hline
\endhead

\hline
\endfoot

\texttt{FLOW-JA-FACE} & 6 & 1 / \texttt{2*nlen} & Connection flow from lake (\texttt{ID1}) to lake through a lake outlet to another lake (\texttt{ID2}). \texttt{nlen} is calculated as the sum of lake outlets that are connected to another lake (\texttt{lakeout} for a lake outlet is not equal to 0). \\
\texttt{GWF} & 6 & 2 / \texttt{maxbound} & Calculated flow from lake (\texttt{ID1}) to GWF cell (\texttt{ID2}). The lake connection-aquifer flow area (\texttt{FLOW-AREA}) is saved as an auxiliary data item for this flow term. \\
\texttt{EXT-INFLOW} & 6 & 1 / \texttt{nlakes} & Specified inflow to reach. The lake number is written to (\texttt{ID1}) and (\texttt{ID2}). \\
\texttt{RUNOFF} & 6 & 1 / \texttt{nlakes} & Specified runoff to reach. The lake number is written to (\texttt{ID1}) and (\texttt{ID2}). \\
\texttt{RAINFALL} & 6 & 1 / \texttt{nlakes} & Specified rainfall on reach. The lake number is written to (\texttt{ID1}) and (\texttt{ID2}). \\
\texttt{EVAPORATION} & 6 & 1 / \texttt{nlakes} & Calculated evaporation from lake. The lake number is written to (\texttt{ID1}) and (\texttt{ID2}). \\
\texttt{WITHDRAWAL} & 6 & 1 / \texttt{nlakes} & Specified withdrawal from lake. The lake number is written to (\texttt{ID1}) and (\texttt{ID2}). \\
\texttt{STORAGE} & 6 & 2 / \texttt{nlakes} & Calculated flow from storage for lake. The lake number is written to (\texttt{ID1}) and (\texttt{ID2}). The lake volume (\texttt{VOLUME}) is saved as an auxiliary data item for this flow term. \\
\texttt{CONSTANT} & 6 & 1 / \texttt{nlakes} & Calculated flow to maintain constant stage for lake. The lake number is written to (\texttt{ID1}) and (\texttt{ID2}). \\
\texttt{EXT-OUTFLOW} & 6 & 1 / \texttt{nlakes} & Calculated outflow to external boundaries (is nonzero for lakes with outlets not connected to another lake). The lake number is written to (\texttt{ID1}) and (\texttt{ID2}). \\
\texttt{FROM-MVR} & 6 & 1 / \texttt{nlakes} & Calculated flow to lake from the MVR Package. Only saved if MVR Package is used in the LAK Package. The lake number is written to (\texttt{ID1}) and (\texttt{ID2}). \\
\texttt{TO-MVR} & 6 & 1 / \texttt{noutlets} & Calculated flow from a lake outlet to the MVR Package. Only saved if MVR Package is used in the LAK Package. The lake number \texttt{LAKEIN} for the connected outlet is written to (\texttt{ID1}) and (\texttt{ID2}). \\
\texttt{AUXILIARY} & 6 & \texttt{naux}+1 / \texttt{nlakes} & Auxiliary variables, if specified in the LAK Package, are saved to this flow term. The first entry of the \texttt{DATA2D} column has a value of zero. The lake number is written to (\texttt{ID1}) and (\texttt{ID2}).
\label{table:binarylak}
\end{longtable}


% multi-aquifer well package binary budget output
\newpage
\begin{longtable}{p{3.5cm} p{2cm} p{3.5cm} p{6.5cm}}
\caption{Data written to the MAW Package binary output file. Flow terms are listed in the order they are written to the MAW Package binary output file} \tabularnewline
\hline
\hline
\textbf{Flow term} & \textbf{IMETH} & \textbf{NDAT / NLIST} & \textbf{Description} \\
\hline
\endfirsthead

\hline
\hline
\textbf{Flow term} & \textbf{IMETH} & \textbf{NDAT / NLIST} & \textbf{Description} \\
\hline
\endhead

\hline
\endfoot

\texttt{GWF} & 6 & 2 / \texttt{maxbound} & Calculated flow from multi-aquifer well (\texttt{ID1}) to GWF cell (\texttt{ID2}). The multi-aquifer well-aquifer flow area (\texttt{FLOW-AREA}) is saved as an auxiliary data item for this flow term.\\
\texttt{RATE} & 6 & 1 / \texttt{nmawwells} & Calculated pumping rate from the multi-aquifer well. The multi-aquifer well number is written to (\texttt{ID1}) and (\texttt{ID2}). \\
\texttt{FW-RATE} & 6 & 1 / \texttt{nmawwells} & calculated flowing well discharge rate from the multi-aquifer well. Only saved if \texttt{FLOWING\_WELLS} is specified in the OPTIONS block. The multi-aquifer well number is written to (\texttt{ID1}) and (\texttt{ID2}). \\
\texttt{STORAGE} & 6 & 2 / \texttt{nmawwells} & Calculated flow from storage for multi-aquifer well. Only saved if the \texttt{NO\_WELL\_STORAGE} is not specified in the OPTIONS block. The multi-aquifer well number is written to (\texttt{ID1}) and (\texttt{ID2}). The multi-aquifer well volume (\texttt{VOLUME}) is saved as an auxiliary data item for this flow term. \\
\texttt{CONSTANT} & 6 & 1 / \texttt{nmawwells} & Calculated flow to maintain constant head in multi-aquifer well. The multi-aquifer well number is written to (\texttt{ID1}) and (\texttt{ID2}). \\
\texttt{FROM-MVR} & 6 & 1 / \texttt{nmawwells} & Calculated flow to lake from the MVR Package. Only saved if MVR Package is used in the MAW Package. The lake number is written to (\texttt{ID1}) and (\texttt{ID2}). \\
\texttt{RATE-TO-MVR} & 6 & 1 / \texttt{nmawwells} & Calculated pumping rate from the multi-aquifer well to the MVR Package. Only saved if MVR Package is used in the MAW Package. The multi-aquifer well number is written to (\texttt{ID1}) and (\texttt{ID2}). \\
\texttt{FW-RATE-TO-MVR} & 6 & 1 / \texttt{nmawwells} & Calculated flowing well flow from a multi-aquifer well to the MVR Package. Only saved if MVR Package is used in the MAW Package and the \texttt{FLOWING\_WELLS} is specified in the OPTIONS block. The multi-aquifer well number is written to (\texttt{ID1}) and (\texttt{ID2}). \\
\texttt{AUXILIARY} & 6 & \texttt{naux}+1 / \texttt{nmawwells} & Auxiliary variables, if specified in the MAW Package, are saved to this flow term. The first entry of the \texttt{DATA2D} column has a value of zero. The multi-aquifer well number is written to (\texttt{ID1}) and (\texttt{ID2}).
\label{table:binarymaw}
\end{longtable}


% streamflow routing package binary budget output
\newpage
\begin{longtable}{p{3.5cm} p{2cm} p{3.5cm} p{6.5cm}}
\caption{Data written to the SFR Package binary output file. Flow terms are listed in the order they are written to the SFR Package binary output file} \tabularnewline
\hline
\hline
\textbf{Flow term} & \textbf{IMETH} & \textbf{NDAT / NLIST} & \textbf{Description} \\
\hline
\endfirsthead

\hline
\hline
\textbf{Flow term} & \textbf{IMETH} & \textbf{NDAT / NLIST} & \textbf{Description} \\
\hline
\endhead

\hline
\endfoot

\texttt{FLOW-JA-FACE} & 6 & 2 / $\sum_{n=1}^{\texttt{maxbound}} \texttt{nconn}_n$  & Connection flow from reach (\texttt{ID1}) to unmanaged and managed (tributaries) connections (\texttt{ID2}). The cross-sectional flow area (\texttt{FLOW-AREA}) is saved as an auxiliary data item for this flow term. \\
\texttt{GWF} & 6 & 2 / \texttt{maxbound} & Calculated flow from reach (\texttt{ID1}) to GWF cell (\texttt{ID2}). The reach-aquifer flow area (\texttt{FLOW-AREA}) is saved as an auxiliary data item for this flow term.\\
\texttt{EXT-INFLOW} & 6 & 1 / \texttt{maxbound} & Specified inflow to reach. The reach number is written to (\texttt{ID1}) and (\texttt{ID2}). \\
\texttt{RUNOFF} & 6 & 1 / \texttt{maxbound} & Specified runoff to reach. The reach number is written to (\texttt{ID1}) and (\texttt{ID2}). \\
\texttt{RAIN} & 6 & 1 / \texttt{maxbound} & Specified rainfall on reach. The reach number is written to (\texttt{ID1}) and (\texttt{ID2}). \\
\texttt{EVAPORATION} & 6 & 1 / \texttt{maxbound} & Calculated evaporation from reach. The reach number is written to (\texttt{ID1}) and (\texttt{ID2}). \\
\texttt{EXT-OUTFLOW} & 6 & 1 / \texttt{maxbound} & Calculated outflow to external boundaries (is nonzero for reaches with no downstream connections). The reach number is written to (\texttt{ID1}) and (\texttt{ID2}). \\
\texttt{STORAGE} & 6 & 2 / \texttt{maxbound} & Calculated storage changes for each reach.  This value is always zero for the present implementation.  The water volume in the reach (\texttt{VOLUME}) is saved as an auxiliary data item for this flow term.  The reach number is written to (\texttt{ID1}) and (\texttt{ID2}). \\
\texttt{FROM-MVR} & 6 & 1 / \texttt{maxbound} & Calculated flow to reach from the MVR Package. Only saved if MVR Package is used in the SFR Package. The reach number is written to (\texttt{ID1}) and (\texttt{ID2}). \\
\texttt{TO-MVR} & 6 & 1 / \texttt{maxbound} & Calculated flow from reach to the MVR Package. Only saved if MVR Package is used in the SFR Package. The reach number is written to (\texttt{ID1}) and (\texttt{ID2}). \\
\texttt{AUXILIARY} & 6 & \texttt{naux}+1 / \texttt{maxbound} & Auxiliary variables, if specified in the SFR Package, are saved to this flow term. The first entry of the \texttt{DATA2D} column has a value of zero.  The reach number is written to (\texttt{ID1}) and (\texttt{ID2}). 
\label{table:binarysfr}
\end{longtable}


% unsaturated zone package binary budget output
\newpage
\begin{longtable}{p{3.5cm} p{2cm} p{3.5cm} p{6.5cm}}
\caption{Data written to the UZF Package binary output file. Flow terms are listed in the order they are written to the UZF Package binary output file} \tabularnewline
\hline
\hline
\textbf{Flow term} & \textbf{IMETH} & \textbf{NDAT / NLIST} & \textbf{Description} \\
\hline
\endfirsthead

\hline
\hline
\textbf{Flow term} & \textbf{IMETH} & \textbf{NDAT / NLIST} & \textbf{Description} \\
\hline
\endhead

\hline
\endfoot

\texttt{FLOW-JA-FACE} & 6 & 1 / \texttt{2*nlen} & Connection flow from UZF cell (\texttt{ID1}) to a connected UZF cell (\texttt{ID2}). \texttt{nlen} is calculated as the number of uzf cells with \texttt{vertcon} values greater than 0.\\
\texttt{GWF} & 6 & 2 / \texttt{maxbound} & Calculated flow from UZF cell (\texttt{ID1}) to GWF cell (\texttt{ID2}). The UZF cell-aquifer flow area (\texttt{FLOW-AREA}) is saved as an auxiliary data item for this flow term.\\
\texttt{INFILTRATION} & 6 & 1 / \texttt{maxbound} & Specified infiltration to UZF cell. The UZF cell number is written to (\texttt{ID1}) and (\texttt{ID2}). \\
\texttt{REJ-INF} & 6 & 1 / \texttt{maxbound} & Calculated rejected infiltration from the UZF cell. The UZF cell number is written to (\texttt{ID1}) and (\texttt{ID2}). \\
\texttt{UZET} & 6 & 1 / \texttt{maxbound} & Calculated evaporation from the UZF cell. The UZF cell number is written to (\texttt{ID1}) and (\texttt{ID2}). \\
\texttt{STORAGE} & 6 & 2 / \texttt{maxbound} & Calculated flow from storage for the UZF cell. The UZF cell number is written to (\texttt{ID1}) and (\texttt{ID2}). \\
\texttt{FROM-MVR} & 6 & 1 / \texttt{maxbound} & Calculated flow to the UZF cell from the MVR Package. Only saved if MVR Package is used in the UZF Package. The UZF cell number is written to (\texttt{ID1}) and (\texttt{ID2}). \\
\texttt{REJ-INF-TO-MVR} & 6 & 1 / \texttt{maxbound} & Calculated rejected infiltration flow from the UZF cell to the MVR Package. Only saved if MVR Package is used in the UZF Package. The UZF cell number is written to (\texttt{ID1}) and (\texttt{ID2}). \\
\texttt{AUXILIARY} & 6 & \texttt{naux}+1 / \texttt{maxbound} & Auxiliary variables, if specified in the UZF Package, are saved to this flow term. The first entry of the \texttt{DATA2D} column has a value of zero.  The UZF cell number is written to (\texttt{ID1}) and (\texttt{ID2}). 
\label{table:binaryuzf}
\end{longtable}


\newpage
\subsection{Observation Output File}

When the BINARY option is used to open an observation output file (see section ``Observation (OBS) Utility''), the output file has the following form. Record 1 has a length of 100 bytes.

\vspace{5mm}
\noindent Record 1: \texttt{TYPE, PRECISION, LENOBSNAME} \textit{(Record 1 includes 85 blanks following LENOBSNAME.)} \\
\noindent Record 2: \texttt{NOBS} \\
\noindent Record 3: \texttt{OBSNAME(1),  OBSNAME(2), ..., OBSNAME(NOBS)} \\

\vspace{12pt}
\noindent \textbf{Repeat for each time step.}

\vspace{12pt}
\noindent Record 4: \texttt{TIME, SIMVALUE(1), SIMVALUE(2), ..., SIMVALUE(NOBS)} \\
 
\vspace{12pt}
\noindent where

\begin{description} \itemsep0pt \parskip0pt \parsep0pt
\item \texttt{TYPE} (bytes 1--4 of Record 1) is ``cont `` ---  ``cont'' indicates the file contains continuous observations ;
\item \texttt{PRECISION} (bytes 6--11 of Record 1) will always be ``double'' to indicate that floating-point values are written in double precision (8 bytes);
\item \texttt{LENOBSNAME} (bytes 12--15 of Record 1) is an integer indicating the number of characters used to store each observation name in following records (in the initial release of MODFLOW~6, LENOBSNAME equals 40);
\item \texttt{NOBS} (4-byte integer) is the number of observations recorded in the file;
\item \texttt{OBSNAME} (LENOBSNAME bytes) is an observation name;
\item \texttt{TIME} (floating-point) is the simulation time; and
\item \texttt{SIMVALUE} (floating-point) is the simulated value.
\end{description}


\newpage
\ifx\usgsdirector\undefined
\addcontentsline{toc}{section}{\hspace{1.5em}\bibname}
\else
\inreferences
\REFSECTION
\fi
\bibliography{../MODFLOW6References}
\bibliographystyle{../usgs.bst}


\newpage
\inappendix
\SECTION{Appendix A. List of Blocks}
This appendix describes changes introduced into MODFLOW 6 in previous releases. These changes may substantially affect users.

\begin{itemize}
	\item Version mf6.1.0
	
	\underline{NEW FUNCTIONALITY}
	\begin{itemize}
		\item Added the Skeletal Storage, Compaction, and Subsidence (CSUB) Package. The one-dimensional effective-stress based compaction theory implemented in the CSUB Package is documented in Leake and Galloway (2007). The numerical approach used for delay interbeds in the CSUB package is documented in Hoffmann and others (2003) and uses the same one-dimensional effective-stress based compaction theory as coarse-grained and fine-grained no-delay interbed sediments. A number of example problems that use the CSUB Package are documented in the ``MODFLOW 6 CSUB Package Example Problems'' pdf document included in this and subsequent releases.
	\end{itemize}
	
	\underline{BASIC FUNCTIONALITY}
	\begin{itemize}
		\item Added an error check to the DISU Package that ensures that an underlying cell has a top elevation that is less than or equal to the bottom of an overlying cell.  An underlying cell is one in which the IHC value for the connection is zero and the connecting node number is greater than the cell node number.
		\item Added restricted IDOMAIN support for DISU grids.  Users can specify an optional IDOMAIN in the DISU Package input file.  IDOMAIN values must be zero or one.  Vertical pass-through cells (specified with an IDOMAIN value of -1 in the DIS or DISV Package input files) are not supported for DISU.   
		\item NPF Package will now write a message to the GWF Model list file to indicate when the SAVE\_SPECIFIC\_DISCHARGE option is invoked.
		\item Added two new options to the NPF Package.  The K22OVERK option allows the user to enter the anisotropy ratio for K22.  If activated, the K22 values entered by the user in the NPF input file will be multiplied by the K values entered in the NPF input file.  The K33OVERK option allows the user to enter the anisotropy ratio for K33.  If activated, the K33 values entered by the user in the NPF input file will be multiplied by the K values entered in the NPF input file.  With this K33OVERK option, for example, the user can specify a value of 0.1 for K33 and have all K33 values be one tenth of the values specified for K.  The program will terminate with an error if these options are invoked, but arrays for K22 and/or K33 are not provided in the NPF input file.
		\item Added new MAXERRORS option to mfsim.nam.  If specified, the maximum number of errors stored and printed will be limited to this number.  This can prevent a situation where memory will run out when there are an excessive number of errors.
		\item Refactored many parts of the code to remove unused variables, conform to stricter FORTRAN standard checks, and allow for new development efforts to be included in the code base.
	\end{itemize}
	
	\underline{STRESS PACKAGES}
	\begin{itemize}
		\item There was an error in the calculation of the segmented evapotranspiration rate for the case where the rate did not decrease with depth.  There was another error in which PETM0 was being used as the evapotranspiration rate at the surface instead of the proportion of the evapotranspiration rate at the surface.
	\end{itemize}
	
	\underline{ADVANCED STRESS PACKAGES}
	\begin{itemize}
		\item Corrected the way auxiliary variables are handled for the advanced packages.  In some cases, values for auxiliary variables were not being correctly written to the GWF Model budget file or to the advanced package budget file.  A consistent approach for updating and saving auxiliary variables was implemented for the MAW, SFR, LAK, and UZF Packages.
		\item The user guide was updated to include a missing laksetting that was omitted from the PERIOD block.  The laksetting description now includes an INFLOW option; a description for INFLOW is also now included.
		\item The LAK package was incorrectly making an error check against NOUTLETS instead of NLAKES.
		\item For the advanced stress packages, values assigned to the auxiliary variables were not written correctly to the GWF Model budget file, but the values were correct in the advanced package budget file.  Program was modified so that auxiliary variables are correctly written to the GWF Model budget file.
		\item Corrected several error messages issued by the SFR Package that were not formatted correctly.  
		\item Fixed a bug in which the lake stage stable would sometimes result in touching numbers.  This only occurred for negative lake stages.
		\item The UZF Package was built on the UZFKinematicType, which used an array of structures.  A large array like this, can cause memory problems.  The UZFKinematicType was replaced with a new UzfCellGroupType, which is a structure of arrays and is much more memory efficient.  The underlying UZF algorithm did not change.
	\end{itemize}
	
	\underline{SOLUTION}
	\begin{itemize}
		\item Add ALL and FIRST options to optional NO\_PTC optional keyword in OPTIONS block. If NO\_PTC option is FIRST, PTC is disabled for the first stress period but is applied in all subsequent steady-state stress periods. If NO\_PTC option is ALL, PTC is disabled for all steady-state stress periods. If the NO\_PTC options is not defined, PTC is disabled for all steady-state stress periods (this is consistent with the behaviour of the NO\_PTC option in previous versions).
	\end{itemize}
	
	\item Version mf6.0.4--Feb. 27, 2019
	
	\underline{BASIC FUNCTIONALITY}
	\begin{itemize}
		\item Addressed issue with pointing contiguous pointer vectors/arrays to non-contiguous pointer vectors/arrays that caused code compilation failure with gfortran-8. A consequence of addressing this issue is that all pointer vectors/arrays that are allocated or pointed to using the memory manager must be defined to be contiguous.
		\item Corrected a problem with the reading of grid data from a binary file, in which the program was reading a binary header for each row of data.
		\item Added a new error check for very small time steps.  If the value of the starting time is equal to the ending time (starting time plus the time step length), then the time step is too small to be differentiated by the program based on the precision of floating point numbers.  The program will terminate with an error in this case.  The program will also terminate if the storage package with a transient stress period has a time step length of zero.
		\item The observation package was modified to use non-advancing output instead of fixed length strings when writing ascii output. The previous use of fixed length strings resulted in truncation of ascii observation output when the product of user-specified \texttt{digits} + 7 and the number of observations exceeded 5000.
		\item Corrected an error in the GWF-GWF Exchange module that caused the specific discharge values in the child model to be calculated incorrectly.  The calculation was incorrect because the face normal for the child model was pointing toward the center of the cell instead of outward.
		\item Minor refactoring to improve code clarity.
	\end{itemize}
	
	\underline{STRESS PACKAGES}
	\begin{itemize}
		\item Minor refactoring to improve code clarity.
	\end{itemize}
	
	\underline{ADVANCED STRESS PACKAGES}
	\begin{itemize}
		\item Modified the Multi-Aquifer Well (MAW) Package so that the HEAD\_LIMIT and RATE\_SCALING options work for injection wells.  Prior to this change, these options only worked for extraction wells.  These options can be used to reduce or even shut off well injection as the head in the well rises above user-specified levels.
		\item Added stage and residual convergence checks to the SFR package to make sure that stage and upstream flow changes between successive outer iterations are less than OUTER\_HCLOSE and OUTER\_RCLOSEBND, respectively. This addition is expected to be useful for steady-state simulations with complicated networks and simple reaches.
		\item Modified the final convergence check for the LAK package to use OUTER\_HCLOSE when evaluating lake stage changes between successive outer iterations.
		\item Modified the final convergence check for the UZF package to use OUTER\_RCLOSEBND when evaluating rejected infiltration, groundwater recharge, and groundwater seepage changes between successive outer iterations.
		\item Minor refactoring to improve code clarity.
	\end{itemize}
	
	\underline{SOLUTION}
	\begin{itemize}
		\item Modified pseudo-transient continuation (PTC) approach to use PTC for steady-state stress period for models using the Newton-Raphson formulation for problems with and without the storage (STO) package. Previously, PTC was only used with problems that did not include the STO package (this was not the intended behavior of PTC).
		\item Added NO\_PTC option to disable PTC for problems where PTC degrades/prevents model convergence. Option only applies to steady-state stress periods for models using the Newton-Raphson formulation. For many problems, PTC can significantly improve convergence behavior for steady-state simulations, and for this reason it is active by default.  In some cases, however, PTC can worsen the convergence behavior, especially when the initial conditions are similar to the solution.  When the initial conditions are similar to, or exactly the same as, the solution and convergence is slow, then this NO\_PTC option should be used to deactivate PTC.  This NO\_PTC option should also be used in order to compare convergence behavior with other MODFLOW versions, as PTC is only available in MODFLOW 6. 
		\item Small improvements to PTC to reduce the initial PTCDEL value loaded on the diagonal. This reduces the number of iterations required to achieve convergence for steady-state stress periods for most problems.
		\item Added OUTER\_RCLOSEBND variable that is used when performing final convergence checks on model packages that solve a separate equation not solved by the IMS linear solver. This value represents the maximum allowable residual at any single model package element between successive outer iterations. An example of a model package that would use OUTER\_RCLOSEBND to evaluate convergence is the SFR package which solves a continuity equation for each reach.
		\item Minor refactoring to improve code clarity.
	\end{itemize}
	
	\item Version mf6.0.3--Aug. 9, 2018
	
	\underline{BASIC FUNCTIONALITY}
	\begin{itemize}
		\item Fixed issues with observations specified using boundnames that are enclosed in quotes. Previously, the closing quote was retained on a boundname enclosed in quotes and resulted in an error (the erroneous observation boundname could not be found in the package).
	\end{itemize}
	
	\underline{STRESS PACKAGES}
	\begin{itemize}
		\item If the AUXMULTNAME keyword was used in combination with time series, then the multiplier was erroneously applied to all time series, and not just the time series in the column to be scaled.  
		\item For the array-based recharge and evapotranspiration packages, the IRCH and IEVT variables (if specified) must be specified as the first variable listed in the PERIOD block.  A check was added so that the program will terminate with an error if IRCH or IEVT is not the first variable listed in the PERIOD block.
		\item For the standard boundary packages, the ``to mover'' term (such as DRN-TO-MVR) written to the GWF Model budget was incorrect.  The budget terms were incorrect because the accumulator variables were not initialized to zero. 
		\item For regular MODFLOW grids, the recharge and evapotranspiration arrays of size (NCOL, NROW) were being echoed to the listing file (if requested by the user) of size (NCOL * NROW). 
	\end{itemize}
	
	\underline{ADVANCED STRESS PACKAGES}
	\begin{itemize}
		\item Fixed spelling of the THIEM keyword in the source code and in the input instructions of the MAW Package.
		\item Fixed an issue with the SFR package when the specified evaporation exceeds the sum of specified and calculated reach inflows, rainfall, and specified runoff. In this case, evaporation is set equal to the sum of specified and calculated reach inflows, rainfall, and specified runoff. Also if a negative runoff is specified and this value exceeds specified and calculated reach inflows, and rainfall then runoff is set to the sum of reach inflows and evaporation is set to zero.
		\item Fixed an issue in the MAW package budget information written to the listing file and MAW cell-by-cell budget file when a previously active well is inactivated. The ratesim variable was not being reset to zero for these wells and the simulated rate from the last stress period when the well was active was being reported.
		\item Program now terminates with an error if the OUTLETS block is present in the LAK package file and NOUTLETS is not specified or specified to be zero in the DIMENSIONS block.  Previously, this did not cause an error condition in the LAK package but would result in a segmentation fault error in the MVR package if LAK package OUTLETS are specified as providers.
		\item Program now terminates with an error when a DIVERSION block is present in a SFR package file but no diversions (all ndiv values are 0) are specified in the PACKAGEDATA block. 
	\end{itemize}
	
	\underline{SOLUTION}
	\begin{itemize}
		\item Fixed bug related to not allocating the preconditioner work array if a non-zero drop tolerance is specified but the number of levels is not specified or specified to be zero. In the case where the number of levels is not specified or specified to be zero the preconditioner work array is dimensioned to the product of the number of cells (NEQ) and the maximum number of connections for any cell.
		\item Updated linear solver output so number of levels and drop tolerance are output if either are specified to be greater than zero. 
	\end{itemize}
	
	\item Version mf6.0.2--Feb 23, 2018
	
	\underline{BASIC FUNCTIONALITY}
	\begin{itemize}
		\item Added a new option, called SAVE\_SPECIFIC\_DISCHARGE to the Node Property Flow Package.  When invoked, $x$, $y$, and $z$ specific discharge components are calculated for the center of each model cell and written to the binary budget file.
		\item For binary input of grid data, such as initial heads, the array reading utility was not reading a header record consisting of KSTP, KPER, PERTIM, TOTIM, TEXT, NLAY, NROW, NCOL.  This meant that a binary head file written by MODFLOW could not be used as input for a subsequent simulation.  For binary input, the array reading utility now reads a header record before reading the array values.
		\item The NOGRB option in the discretization packages was not working.  This option will now prevent the binary grid file from being written.
		\item Removed the PRIVATE attribute for two methods of the discretization packages so that the program works as intended with the latest Intel Fortran release.
		\item Switched to using a long integer for the memory manager so that memory usage is calculated correctly for large models.
	\end{itemize}
	
	\underline{STRESS PACKAGES}
	\begin{itemize}
		\item If a steady-state stress period followed a transient stress period, the storage terms written to the budget file were not being reset to zero.  The program now initializes these budget values to zero for steady-state periods before they are written.
	\end{itemize}
	
	\underline{ADVANCED STRESS PACKAGES}
	\begin{itemize}
		\item The STATUS INACTIVE option was not working correctly for the MAW Package.
		\item Modified the MAW connection conductance calculation so that a linear relation between the water level in a cell and saturation is used for the standard formulation. In the previous version, the same quadratic saturation function was being used for the standard and Newton-Raphson formulation to calculate the MAW connection conductance. 
		\item Modified the MAW Package so that the top and bottom of the screen for a connection are reset to the top and bottom of the cell, respectively, for SPECIFIED, THEIM, SKIN, and CUMULATIVE conductance equations (CONDEQN). Also, the program will now terminate with an error if a MAW well using SPECIFIED, THEIM, SKIN, or CUMULATIVE conductance equations has more than one connection to a single GWF cell. 
		\item Modified the MAW package so that the well bottom (BOTTOM) is reset to the cell bottom in the lowermost GWF cell connection in cases where the specified well bottom is above the bottom of this GWF cell.
	\end{itemize}
	
	\underline{SOLUTION}
	\begin{itemize}
		\item Prior to applying pseudo transient continuation terms, the Iterative Model Solution confirms that the L2-norm exceeds the previous L2-norm.  If it doesn't then pseudo transient continuation is turned off.  This fixes a rare situation in which convergence could not be achieved for consecutive steady state solutions with the same or similar answers. 
	\end{itemize}
	
	
	\item Version mf6.0.1--Sep 28, 2017
	
	\underline{BASIC FUNCTIONALITY}
	\begin{itemize}
		\item There is no requirement that FTYPE entries in the GWF name file should be upper case; however, an upper case convention was being enforced.  FTYPE entries can now be specified using any case.
		\item Tab characters within model input files were not being skipped correctly.  This has been fixed.
		\item The program was updated to use the ``approved for release'' disclaimer.  The previous version was still using a ``preliminary software'' disclaimer.
		\item The source code for time series and time array series was refactored.  Included in the refactoring was a correction to time array series to allow the time array to change from one stress period to the next.  The source file TimeSeriesGroupList.f90 was renamed to TimeSeriesFileList.f90.
	\end{itemize}
	
	\underline{STRESS PACKAGES}
	\begin{itemize}
		\item Fixed inconsistency with CHD package observation name in code (\texttt{chd-flow}) and name in the input-output document (\texttt{chd}). Using name defined in input-output document (\texttt{chd}).
		\item The cell area was not being used in the calculation of recharge and evapotranspiration when list input was used with time series.
		\item The AUXMULTNAME option was not being applied for recharge and evapotranspiration when the READASARRAYS option was used.
		\item The program was not terminating with an error if a PERIOD block was encountered with an iper value equal to the previous iper value.  Program now terminates with an error.
	\end{itemize}
	
	\underline{ADVANCED STRESS PACKAGES}
	\begin{itemize}
		\item Fixed incorrect sign for SFR package exchange with GWF model (\texttt{sfr}).
		\item Added option to specify \texttt{none} as the \texttt{bedleak} for a lake-\texttt{GWF} connection in lake (LAK) package. This option makes the lake-\texttt{GWF} connection conductance solely a function of aquifer properties in the connected \texttt{GWF} cell and lakebed sediments are assumed to be absent for this connection.
		\item Fixed bug in lake (LAK) and multi-aquifer well (MAW) packages that only reset steady-state flag if lake and/or multi-aquifer data are read for a stress period (in the pak\_rp() routines). Using pointer to GWF iss variable in the LAK package and resetting the MAW steady state flag in maw\_rp() routine every stress period, regardless of whether MAW data are specified for a stress period.
		\item Added a convergence check routine to the GWF Mover Package that requires at least two outer iterations if there are any active movers.  Because mover rates are lagged by one outer iteration, at least two outer iterations are required for some problems.
		\item Changed the behavior of the LAK Package so that recharge and evapotranspiration are applied to a vertically connected GWF model cell if the lake status is INACTIVE.  Prior to this change, recharge and evapotranspiration were only applied to an underlying GWF model cell if the lake was dry.
	\end{itemize}
	
	\underline{SOLUTION}
	\begin{itemize}
		\item Fixed bug in IMS that allowed convergence when outer iteration HCLOSE value was satisfied but the model did not converge during the inner iterations.
		\item Added STRICT rclose\_option that uses a infinity-Norm RCLOSE criteria but requires HCLOSE and RCLOSE be satisfied on the first inner iteration of an outer iteration. The STRICT option is identical to the closure criteria approach use in the PCG Package in MODFLOW-2005.
	\end{itemize}
	
	\underline{EXCHANGES}
	\begin{itemize}
		\item Use of an OPEN/CLOSE file was not being allowed for the OPTIONS and DIMENSIONS blocks of the GWF6-GWF6 exchange input file.  OPEN/CLOSE input is now allowed for both of these blocks.
	\end{itemize}
	
	\item
	Version mf6.0.0---August 10, 2017
	
	\underline{BASIC FUNCTIONALITY}
	\begin{itemize}
		\item Removed support for the SINGLE observation type.  All observations must be CONTINUOUS, which means observation values are written for every time step. 
		\item Added support for a no-data value (3.0E30), which can be used as a placeholder in a time-series file containing multiple time series. Use of the no-data value facilitates combining separate time series into a single file when the time series contain records for differing simulation times.
		\item Model names specified in the simulation name file cannot have spaces in them.  A check was implemented to terminate with an error if the model name contains spaces.  Model names cannot exceed 16 characters.  Trailing spaces are allowed.
		\item The name and version of the compiler used to make the run file is now written to the terminal and to the simulation list file.
		\item Many of the Fortran source files were modified and reformatted.  Unused variables were removed.
	\end{itemize}
	
	\underline{ADVANCED STRESS PACKAGES}
	\begin{itemize}
		\item Updated MAW package so that well connection conductance calculations correctly account for THICKSTRT in the NPF package for layers that use THICKSTRT (and are confined).
		\item Added \texttt{CUMULATIVE} \texttt{coneqn} (conductance) option to MAW package.
		\item Fixed bug in LAK package weir lake outlet calculation.
		\item Fixed bug in LAK package when internal outlets were specified and combined with the MVR package that was also moving water internally in the same LAK package.
		\item Updated the table created when PRINT\_FLOWS is specified in the LAK package OPTIONS block to include internal flow terms if NOUTLETS is greater than 0. 
		\item Renamed Lake Tables DIMENSIONS block NENTRIES to NROW and added NCOL to DIMENSIONS block.
		\item Eliminated MAXIMUM\_OUTLET\_DEPTH = 10 [L] as default behavior for MANNING and WEIR LAK package lake outlet types. The maximum depth threshold was used in MODFLOW-2005 lake package because a table was used to calculate lake outflows to SFR. Can still use maximum depth threshold in develop versions of MODFLOW 6 by specifying MAXIMUM\_OUTLET\_DEPTH in the options block with a value.
		\item Removed MULTILAYER option for UZF package---this option didn't actually do anything.
		\item Added the requirement that the UZF number be specified as the first value on each line in the PACKAGEDATA block.
		\item Renamed MAXBOUND in the DIMENSIONS block of the SFR Package to be NREACHES.
		\item Implemented a check in the SFR Package to make sure that information is specified in the PACKAGEDATA block for every reach.  Program terminates with an error if information for a reach is not found.
	\end{itemize}
	
	\item
	Version mf6beta0.9.03---June 23, 2017
	
	\underline{BASIC FUNCTIONALITY}
	\begin{itemize}
		\item Renamed all FTYPE keywords to version 6.  They were named with an 8.  So, for example, the GHB Package is now activated in the GWF name file using ``GHB6'' instead of ``GHB8''.
		\item Keywords in the simulation name file must now be specified as TDIS6, GWF6, and GWF6-GWF6 to be consistent.
		\item The DIS Package had grid offsets (XOFFSET and YOFFSET) that could be specified as options.  These offsets were relative to the upper-left corner of the model grid.  The default value for YOFFSET was set to the sum of DELR so that (0, 0) would correspond to the lower-left corner of the model grid.  These options have been removed and replaced with XORIGIN and YORIGIN, which is the coordinate of the lower-left corner of the model grid.  The default value is zero for XORIGIN and YORIGIN.
		\item Can now specify XORIGIN, YORIGIN, and ANGROT as options for the DISV and DISU packages.  These values are written to the binary grid file, which can be used by post-processors to locate the model grid in space.  These options have no affect on the simulation results.  The default value is 0.0 if not specified.
		\item Added a new option to the TDIS input file called START\_DATE\_TIME.  This is a 30 character string that represents the simulation starting date and time, preferably in the format described at https://www.w3.org/TR/NOTE-datetime.  The value provided by the user has no affect on the simulation, but if it is provided, the value is written to the simulation list file.
		\item Changed default behavior for how memory usage is written to the end of the simulation list file.  Added new MEMORY\_PRINT\_OPTION to simulation options to control how memory usage is written.
		\item Corrections were made to the memory manager to ensure that all memory is deallocated at the end of a simulation.
	\end{itemize}
	
	\underline{INTERNAL FLOW PACKAGES}
	\begin{itemize}
		\item Changed the way hydraulic conductivity is specified in the NPF Package.  Users no longer specify HK, VK, and HANI.  Hydraulic conductivity is now specified as ``K''.  If hydraulic conductivity is isotropic, then this is all that needs to be specified.  For anisotropic cases, the user can specify an optional ``K22'' array and an optional ``K33'' array.  For an unrotated conductivity ellipsoid ``K22'' corresponds to hydraulic conductivity in the y direction and ``K33'' corresponds to hydraulic conductivity in the z direction, respectively.
	\end {itemize}
	
	\underline{ADVANCED STRESS PACKAGES}
	\begin{itemize}
		\item Modified the MAW Package to include the effects of aquifer anisotropy in the calculation of conductance.
		\item Simplified the SFR Package connectivity to reflect feedback from beta users. There is no longer a requirement to connect reaches that do not have flow between them.  Program will now terminate with an error if this condition is encountered.
		\item Added simple routing option to SFR package. This is the equivalent of the specified depth option (icalc=0) in previous versions of MODFLOW. If water is available in the reach, then there can be leakage from the SFR reach into the aquifer.  If no water is available, then no leakage is applied.  STAGE keyword also added and only applies to reaches that use the simple routing option. If the STAGE keyword is not specified for reaches that use the simple routing option the specified stage is set to the top of the reach (depth = 0).
		\item Added functionality to pass SFR leakage to the aquifer to the highest active layer.
		\item Converted SFR Manning's to a time-varying, time series aware variable.  
		\item Updated LAK package so that conductance calculations correctly account for THICKSTRT in the NPF package for layers that use THICKSTRT (and are confined). Also updated EMBEDDEDH and EMBEDDEDV so that the conductance for these connection types are constant for confined layers.
		\item Converted UZF stress period data to time series aware data.
		\item Added time-series aware AUXILIARY variables to UZF package.
		\item Implemented AUXMULTNAME in options block for UZF package (AUXILIARY variables have to be specified). AUXMULTNAME is applied to the GWF cell area and is used to simulated more than one UZF cell per GWF cell. This could be used to simulate different land use classifications (i.e., agricultural and natural land use types) in the same GWF cell.
	\end{itemize}
	
	\underline{SOLUTION}
	\begin{itemize}
		\item Reworked IMS convergence information so that model specific convergence information is also printed to each model listing file when PRINT\_OPTION ALL is specified in the IMS OPTIONS block.
		\item Added csv output option for IMS convergence information. Solution convergence information and model specific convergence information (if the solution includes more than one model) is written to a comma separated value file. If PRINT\_OPTION is NONE or SUMMARY, csv output includes maximum head change convergence information at the end of each outer iteration for each time step. If PRINT\_OPTION is ALL, csv output includes maximum head change and maximum residual convergence information for the solution and each model (if the solution includes more than one model) and linear acceleration information for each inner iteration. 
	\end{itemize}
	
	\item
	Version mf6beta0.9.02---May 19, 2017
	\begin{itemize}
		\item Renamed gwf3.f90 to be lower case.
		\item Added the missing ``divrate'' variable to the ``sfrsetting'' description in mf6io.pdf.
		\item Added additional error trapping to the array reading utilities.
		\item There was a problem with the binary budget file when a GWF Exchange was used to connect a GWF Model with itself.  This has been fixed.
		\item Standardized `\texttt{to-mvr}' cell-by-cell item in standard stress packages and UZF package.
		\item Fixed incorrect `\texttt{UZF-EVT}' budget accumulator used in GWF listing budget. 
		\item Standardized justification of cell-by-cell `\texttt{text}' strings.
		\item Standardized use of AUXILIARY keyword.
	\end{itemize}
	
	\item
	Version mf6beta0.9.01---May 11, 2017
	\begin{itemize}
		\item Added a copy of the third MODFLOW 6 report. 
		\item Made several minor corrections to doc/mf6io.pdf.  
		\item If vertices were specified for DISU, then the last header line was not written to the binary grid file.  This has been corrected.
	\end{itemize}
	
	\item
	Version mf6beta0.9.00---May 10, 2017
	\begin{itemize}
		\item First public release of MODFLOW 6 in beta form. 
	\end{itemize}

\end{itemize}


\end{document}