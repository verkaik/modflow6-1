The GWT Model simulates three-dimensional transport of a single solute species in flowing groundwater.  The GWT Model solves the solute transport equation using numerical methods and a generalized control-volume finite-difference approach, which can be used with regular MODFLOW grids (DIS Package) or with unstructured grids (DISV and DISU Packages).  The GWT Model is designed to work with most of the new capabilities released with the GWF Model, including the Newton flow formulation, unstructured grids, advanced packages, and the movement of water between packages.  The GWF and GWT Models operate simultaneously during a \mf simulation to represent coupled groundwater flow and solute transport.  The GWT Model can also run separately from a GWF Model by reading the heads and flows saved by a previously run GWF Model.  The GWT model is also capable of working with the flows from another groundwater flow model, as long as the flows from that model can be written in the correct form to flow and head files.  

The purpose of the GWT Model is to calculate changes in solute concentration in both space and time.  Solute concentrations within an aquifer can change in response to multiple solute transport processes.  These processes include (1) advective transport of solute with flowing groundwater, (2) the combined hydrodynamic dispersion processes of velocity-dependent mechanical dispersion and chemical diffusion, (3) sorbtion of solutes by the aquifer matrix either by adsorption to individual solid grains or by absorbtion into solid grains, (4) transfer of solute into very low permeability aquifer material (called an immobile domain) where it can be stored and later released, (5) first- or zero-order solute decay or production in response to chemical or biological reactions, (6) mixing with fluids from groundwater sources and sinks, and (7) direct addition of solute mass.

With the present implementation, there can be multiple domains and multiple phases.  There is a single mobile domain, which normally consists of flowing groundwater, and there can be one or more immobile domains.  The GWT Model simulates the dissolved phase of chemical constituents in both the mobile and immobile domains.  The dissolved phase is also referred to in this report as the aqueous phase.  If sorbtion is represented, then the GWT Model also simulates the solid phase of the chemical constituent in both the mobile and immobile domains.  The dissolved and solid phases of the chemical constituent are tracked in the different domains by the GWT Model and can be reported as output as requested by the user.

This section describes the data files for a \mf Groundwater Transport (GWT) Model.  A GWT Model is added to the simulation by including a GWT entry in the MODELS block of the simulation name file.  There are three types of spatial discretization approaches that can be used with the GWT Model: DIS, DISV, and DISU.  The input instructions for these three packages are not described here in this section on GWT Model input; input instructions for these three packages are described in the section on GWF Model input.

The GWT Model is designed to permit input to be gathered, as it is needed, from many different files.  Likewise, results from the model calculations can be written to a number of output files. The GWT Model Listing File is a key file to which the GWT model output is written.  As \mf runs, information about the GWT Model is written to the GWT Model Listing File, including much of the input data (as a record of the simulation) and calculated results.  Details about the files used by each package are provided in this section on the GWT Model Instructions.

The GWT Model reads a file called the Name File, which specifies most of the files that will be used in a simulation. Several files are always required whereas other files are optional depending on the simulation. The Output Control Package receives instructions from the user to control the amount and frequency of output.  Details about the Name File and the Output Control Package are described in this section.

For the GWT Model, ``flows'' (unless stated otherwise) represent solute mass ``flow'' in mass per time, rather than groundwater flow.  

\subsection{Information for Existing Solute Transport Modelers}
The \mf GWT Model contains most of the functionality of MODFLOW-GWT, MT3DMS, MT3D-USGS and MODFLOW-USG.  The following list summarizes major differences between the GWT Model in \mf and previous MODFLOW-based solute transport programs.

\begin{enumerate}

\item The GWT Model simulates transport of a single chemical species; however, because \mf allows for multiple models of the same type to be included in a single simulation, multiple species can be represented by using multiple GWT Models.

\item There is no specialized flow and transport link file \citep{zheng2001modflow} used to pass the simulated groundwater flows to the transport model.  Instead, simulated flows from the GWF Model are passed in memory to the GWT Model while the program is running.  Alternatively, the GWT Model can read binary flow and head files saved by the GWF Model while it is running.  If the user intends to simulate transport through the advanced stress packages and Water Mover Package, then flows from these advanced packages must also be saved to binary files.  Names for these binary files are provided as input to the FMI Package.

\item The GWT Model is based on a generalized control-volume finite-difference method, which means that solute transport can be simulated using regular MODFLOW grids consisting of layers, rows, and columns, or solute transport can be simulated using unstructured grids.

\item Advection can be simulated using central-in-space weighting, upstream weighting, or an implicit second-order TVD scheme.  The GWT model does not have the Method of Characteristics (particle-based approaches) or an explicit TVD scheme.  Consequently, the GWT Model may require a higher level of spatial discretization than other transport models that use higher order terms for advection dominated systems.  This can be an important limitation for some problems, which require the preservation of sharp solute fronts. 

\item Variable-density flow and transport can be simulated by including a GWF Model and a GWT Model in the same \mf simulation.  The Buoyancy Package should be activated for the GWF Model so that fluid density is calculated as a function of simulated concentration.  If more than one chemical species is represented then the Buoyancy Package allows the simulated concentration for each of them to be used in the density equation of state.   \cite{langevin2020hydraulic} describe the hydraulic-head formation that is implemented in the Buoyancy Package for variable-density groundwater flow and present the results from \mf variable-density simulations.  The variable-density capabilities available in \mf replicate and extend the capabilities available in SEAWAT to include the Newton flow formulation and unstructured grids, for example.  

\item The GWT model includes the MST and IST Packages.  These two package collectively comprise the capabilities of the MT3DMS Reactions Package.

\item The MST Package contains the linear, Freundlich, and Langmuir isotherms for representing sorption.  The IST Packages contains only the linear isotherm for representation of sorption. 

\item The GWT model was designed so that the user can specify as many immobile domains and necessary to represent observed contaminant transport patterns and solute breakthrough curves.  The effects of an immobile domain are represented using the Immobile Storage and Transfer (IST) Package, and the user can specify as many IST Packages as necessary.  

\item Although there is GWF-GWF Exchange, a GWT-GWT Exchange has not yet been developed to connect multiple transport models, as might be done in a nested grid configuration.  

\item There is no option to automatically run the GWT Model to steady state using a single time step.  This is an option available in MT3DMS \citep{zheng2010supplemental}.  Steady state conditions must be determined by running the transport model under transient conditions until concentrations stabilize.

\item The GWT Model described in this report is capable of simulating solute transport in the advanced stress packages of \mfcomma including the Lake, Streamflow Routing, Multi-Aquifer Well and Unsaturated Zone Transport Packages.  The present implementation simulates solute advection between package features, such as between two stream reaches, but dispersive transport is not represented.  Likewise, solute transport between the advanced packages and the aquifer occurs only through advection.

\item The GWT Model has not yet been programmed to work with the Skeletal Storage, Compaction, and Subsidence (CSUB) Package for the GWF Model.  

\item There are many other differences between the \mf GWT Model and other solute transport models that work with MODFLOW, especially with regards to program design and input and output.  Descriptions for the GWT input and output are described here.

\end{enumerate}

\subsection{Units of Length and Time}
The GWF Model formulates the groundwater flow equation without using prescribed length and time units. Any consistent units of length and time can be used when specifying the input data for a simulation. This capability gives a certain amount of freedom to the user, but care must be exercised to avoid mixing units.  The program cannot detect the use of inconsistent units.

\subsection{Solute Mass Budget}
A summary of all inflow (sources) and outflow (sinks) of solute mass is called a mass budget.  \mf calculates a mass budget for the overall model as a check on the acceptability of the solution, and to provide a summary of the sources and sinks of mass to the flow system.  The solute mass budget is printed to the GWT Model Listing File for selected time steps.

\subsection{Time Stepping}

For the present implementation of the GWT Model, all terms in the solute transport equation are solved implicitly.  With the implicit approach applied to the transport equation, it is possible to take relatively large time steps and efficiently obtain a stable solution.  If the time steps are too large, however, accuracy of the model results will suffer, so there is usually some compromise required between the desired level of accuracy and length of the time step.  An assessment of accuracy can be performed by simply running simulations with shorter time steps and comparing results.

In \mf time step lengths are controlled by the user and specified in the Temporal Discretization (TDIS) input file.  When the flow model and transport model are included in the same simulation, then the length of the time step specified in TDIS is used for both models.  If the GWT Model runs in a separate simulation from the GWT Model, then the time steps used for the transport model can be different, and likely shorter, than the time steps used for the flow solution.  Instructions for specifying time steps are described in the TDIS section of this user guide; additional information on GWF and GWT configurations are in the Flow Model Interface section.  



\newpage
\subsection{GWT Model Name File}
The GWF Model Name File specifies the options and packages that are active for a GWF model.  The Name File contains two blocks: OPTIONS  and PACKAGES. The length of each line must be 299 characters or less. The lines in each block can be in any order.  Files listed in the PACKAGES block must exist when the program starts. 

Comment lines are indicated when the first character in a line is one of the valid comment characters.  Commented lines can be located anywhere in the file. Any text characters can follow the comment character. Comment lines have no effect on the simulation; their purpose is to allow users to provide documentation about a particular simulation. 

\vspace{5mm}
\subsubsection{Structure of Blocks}
\lstinputlisting[style=blockdefinition]{./mf6ivar/tex/gwf-nam-options.dat}
\lstinputlisting[style=blockdefinition]{./mf6ivar/tex/gwf-nam-packages.dat}

\vspace{5mm}
\subsubsection{Explanation of Variables}
\begin{description}
\input{./mf6ivar/tex/gwf-nam-desc.tex}
\end{description}

\begin{table}[H]
\caption{Ftype values described in this report.  The \texttt{Pname} column indicates whether or not a package name can be provided in the name file}
\small
\begin{center}
\begin{tabular*}{\columnwidth}{l l l}
\hline
\hline
Ftype & Input File Description & \texttt{Pname}\\
\hline
DIS6 & Rectilinear Discretization Input File \\
DISV6 & Discretization by Vertices Input File \\
DISU6 & Unstructured Discretization Input File \\
IC6 & Initial Conditions Package \\
OC6 & Output Control Option \\
NPF6 & Node Property Flow Package \\ 
STO6 & Storage Package \\
CSUB6 & Compaction and Subsidence Package \\
HFB6 & Horizontal Flow Barrier Package\\
CHD6 & Time-Variant Specified Head Option & * \\
WEL6 & Well Package & * \\
DRN6 & Drain Package & * \\
RIV6 & River Package & * \\
GHB6 & General-Head Boundary Package & * \\
RCH6 & Recharge Package & * \\
EVT6 & Evapotranspiration Package & * \\
MAW6 & Multi-Aquifer Well Package & * \\
SFR6 & Streamflow Routing Package & * \\
LAK6 & Lake Package & * \\
UZF6 & Unsaturated Zone Flow Package & * \\
MVR6 & Water Mover Package \\
GNC6 & Ghost-Node Correction Package \\
OBS6 & Observations Option \\
\hline 
\end{tabular*}
\label{table:ftype}
\end{center}
\normalsize
\end{table}

\vspace{5mm}
\subsubsection{Example Input File}
\lstinputlisting[style=inputfile]{./mf6ivar/examples/gwf-nam-example.dat}



%\newpage
%\subsection{Structured Discretization (DIS) Input File}
%\input{gwf/dis}

%\newpage
%\subsection{Discretization with Vertices (DISV) Input File}
%\input{gwf/disv}

%\newpage
%\subsection{Unstructured Discretization (DISU) Input File}
%\input{gwf/disu}

\newpage
\subsection{Initial Conditions (IC) Package}
Initial Conditions (IC) Package information is read from the file that is specified by ``IC6'' as the file type.  Only one IC Package can be specified for a GWT model. 

\vspace{5mm}
\subsubsection{Structure of Blocks}
%\lstinputlisting[style=blockdefinition]{./mf6ivar/tex/gwf-ic-options.dat}
\lstinputlisting[style=blockdefinition]{./mf6ivar/tex/gwt-ic-griddata.dat}

\vspace{5mm}
\subsubsection{Explanation of Variables}
\begin{description}
% DO NOT MODIFY THIS FILE DIRECTLY.  IT IS CREATED BY mf6ivar.py 

\item \textbf{Block: GRIDDATA}

\begin{description}
\item \texttt{strt}---is the initial (starting) concentration---that is, concentration at the beginning of the GWT Model simulation.  STRT must be specified for all GWT Model simulations. One value is read for every model cell.

\end{description}


\end{description}

\vspace{5mm}
\subsubsection{Example Input File}
\lstinputlisting[style=inputfile]{./mf6ivar/examples/gwt-ic-example.dat}



\newpage
\subsection{Output Control (OC) Option}
Input to the Output Control Option of the Groundwater Transport Model is read from the file that is specified as type ``OC6'' in the Name File. If no ``OC6'' file is specified, default output control is used. The Output Control Option determines how and when concentrations are printed to the listing file and/or written to a separate binary output file.  Under the default, concentration and overall transport budget are written to the Listing File at the end of every stress period. The default printout format for concentrations is 10G11.4.  The concentrations and overall transport budget are also written to the list file if the simulation terminates prematurely due to failed convergence.

Output Control data must be specified using words.  The numeric codes supported in earlier MODFLOW versions can no longer be used.

For the PRINT and SAVE options of concentration, there is no option to specify individual layers.  Whenever the concentration array is printed or saved, all layers are printed or saved.

\vspace{5mm}
\subsubsection{Structure of Blocks}
\vspace{5mm}

\noindent \textit{FOR EACH SIMULATION}
\lstinputlisting[style=blockdefinition]{./mf6ivar/tex/gwt-oc-options.dat}
\vspace{5mm}
\noindent \textit{FOR ANY STRESS PERIOD}
\lstinputlisting[style=blockdefinition]{./mf6ivar/tex/gwt-oc-period.dat}

\vspace{5mm}
\subsubsection{Explanation of Variables}
\begin{description}
% DO NOT MODIFY THIS FILE DIRECTLY.  IT IS CREATED BY mf6ivar.py 

\item \textbf{Block: OPTIONS}

\begin{description}
\item \texttt{BUDGET}---keyword to specify that record corresponds to the budget.

\item \texttt{FILEOUT}---keyword to specify that an output filename is expected next.

\item \texttt{budgetfile}---name of the output file to write budget information.

\item \texttt{CONCENTRATION}---keyword to specify that record corresponds to concentration.

\item \texttt{concentrationfile}---name of the output file to write conc information.

\item \texttt{PRINT\_FORMAT}---keyword to specify format for printing to the listing file.

\item \texttt{columns}---number of columns for writing data.

\item \texttt{width}---width for writing each number.

\item \texttt{digits}---number of digits to use for writing a number.

\item \texttt{format}---write format can be EXPONENTIAL, FIXED, GENERAL, or SCIENTIFIC.

\end{description}
\item \textbf{Block: PERIOD}

\begin{description}
\item \texttt{iper}---integer value specifying the starting stress period number for which the data specified in the PERIOD block apply.  IPER must be less than or equal to NPER in the TDIS Package and greater than zero.  The IPER value assigned to a stress period block must be greater than the IPER value assigned for the previous PERIOD block.  The information specified in the PERIOD block will continue to apply for all subsequent stress periods, unless the program encounters another PERIOD block.

\item \texttt{SAVE}---keyword to indicate that information will be saved this stress period.

\item \texttt{PRINT}---keyword to indicate that information will be printed this stress period.

\item \texttt{rtype}---type of information to save or print.  Can be BUDGET or CONCENTRATION.

\item \texttt{ocsetting}---specifies the steps for which the data will be saved.

\begin{lstlisting}[style=blockdefinition]
ALL
FIRST
LAST
FREQUENCY <frequency>
STEPS <steps(<nstp)>
\end{lstlisting}

\item \texttt{ALL}---keyword to indicate save for all time steps in period.

\item \texttt{FIRST}---keyword to indicate save for first step in period. This keyword may be used in conjunction with other keywords to print or save results for multiple time steps.

\item \texttt{LAST}---keyword to indicate save for last step in period. This keyword may be used in conjunction with other keywords to print or save results for multiple time steps.

\item \texttt{frequency}---save at the specified time step frequency. This keyword may be used in conjunction with other keywords to print or save results for multiple time steps.

\item \texttt{steps}---save for each step specified in STEPS. This keyword may be used in conjunction with other keywords to print or save results for multiple time steps.

\end{description}


\end{description}

\vspace{5mm}
\subsubsection{Example Input File}
\lstinputlisting[style=inputfile]{./mf6ivar/examples/gwt-oc-example.dat}


\newpage
\subsection{Observation (OBS) Utility for a GWT Model}

GWT Model observations include the simulated groundwater concentration (\texttt{concentration}), and the mass flow, with units of mass per time, between two connected cells (\texttt{flow-ja-face}). The data required for each GWT Model observation type is defined in table~\ref{table:gwtobstype}. For \texttt{flow-ja-face} observation types, negative and positive values represent a loss from and gain to the \texttt{cellid} specified for ID, respectively.

\subsubsection{Structure of Blocks}
\vspace{5mm}

\noindent \textit{FOR EACH SIMULATION}
\lstinputlisting[style=blockdefinition]{./mf6ivar/tex/utl-obs-options.dat}
\lstinputlisting[style=blockdefinition]{./mf6ivar/tex/utl-obs-continuous.dat}

\subsubsection{Explanation of Variables}
\begin{description}
% DO NOT MODIFY THIS FILE DIRECTLY.  IT IS CREATED BY mf6ivar.py 

\item \textbf{Block: OPTIONS}

\begin{description}
\item \texttt{digits}---Keyword and an integer digits specifier used for conversion of simulated values to text on output. The default is 5 digits. When simulated values are written to a file specified as file type DATA in the Name File, the digits specifier controls the number of significant digits with which simulated values are written to the output file. The digits specifier has no effect on the number of significant digits with which the simulation time is written for continuous observations.

\item \texttt{PRINT\_INPUT}---keyword to indicate that the list of observation information will be written to the listing file immediately after it is read.

\end{description}
\item \textbf{Block: CONTINUOUS}

\begin{description}
\item \texttt{FILEOUT}---keyword to specify that an output filename is expected next.

\item \texttt{obs\_output\_file\_name}---Name of a file to which simulated values corresponding to observations in the block are to be written. The file name can be an absolute or relative path name. A unique output file must be specified for each CONTINUOUS block. If the ``BINARY'' option is used, output is written in binary form. By convention, text output files have the extension ``csv'' (for ``Comma-Separated Values'') and binary output files have the extension ``bsv'' (for ``Binary Simulated Values'').

\item \texttt{BINARY}---an optional keyword used to indicate that the output file should be written in binary (unformatted) form.

\item \texttt{obsname}---string of 1 to 40 nonblank characters used to identify the observation. The identifier need not be unique; however, identification and post-processing of observations in the output files are facilitated if each observation is given a unique name.

\item \texttt{obstype}---a string of characters used to identify the observation type.

\item \texttt{id}---Text identifying cell where observation is located. For packages other than NPF, if boundary names are defined in the corresponding package input file, ID can be a boundary name. Otherwise ID is a cellid. If the model discretization is type DIS, cellid is three integers (layer, row, column). If the discretization is DISV, cellid is two integers (layer, cell number). If the discretization is DISU, cellid is one integer (node number).

\item \texttt{id2}---Text identifying cell adjacent to cell identified by ID. The form of ID2 is as described for ID. ID2 is used for intercell-flow observations of a GWF model, for three observation types of the LAK Package, for two observation types of the MAW Package, and one observation type of the UZF Package.

\end{description}


\end{description}


\begin{longtable}{p{2cm} p{2.75cm} p{2cm} p{1.25cm} p{7cm}}
\caption{Available GWT model observation types} \tabularnewline

\hline
\hline
\textbf{Model} & \textbf{Observation type} & \textbf{ID} & \textbf{ID2} & \textbf{Description} \\
\hline
\endhead

\hline
\endfoot


GWT Model observations include the simulated groundwater concentration (\texttt{concentration}), and the mass flow, with units of mass per time, between two connected cells (\texttt{flow-ja-face}). The data required for each GWT Model observation type is defined in table~\ref{table:gwtobstype}. For \texttt{flow-ja-face} observation types, negative and positive values represent a loss from and gain to the \texttt{cellid} specified for ID, respectively.

\subsubsection{Structure of Blocks}
\vspace{5mm}

\noindent \textit{FOR EACH SIMULATION}
\lstinputlisting[style=blockdefinition]{./mf6ivar/tex/utl-obs-options.dat}
\lstinputlisting[style=blockdefinition]{./mf6ivar/tex/utl-obs-continuous.dat}

\subsubsection{Explanation of Variables}
\begin{description}
% DO NOT MODIFY THIS FILE DIRECTLY.  IT IS CREATED BY mf6ivar.py 

\item \textbf{Block: OPTIONS}

\begin{description}
\item \texttt{digits}---Keyword and an integer digits specifier used for conversion of simulated values to text on output. The default is 5 digits. When simulated values are written to a file specified as file type DATA in the Name File, the digits specifier controls the number of significant digits with which simulated values are written to the output file. The digits specifier has no effect on the number of significant digits with which the simulation time is written for continuous observations.

\item \texttt{PRINT\_INPUT}---keyword to indicate that the list of observation information will be written to the listing file immediately after it is read.

\end{description}
\item \textbf{Block: CONTINUOUS}

\begin{description}
\item \texttt{FILEOUT}---keyword to specify that an output filename is expected next.

\item \texttt{obs\_output\_file\_name}---Name of a file to which simulated values corresponding to observations in the block are to be written. The file name can be an absolute or relative path name. A unique output file must be specified for each CONTINUOUS block. If the ``BINARY'' option is used, output is written in binary form. By convention, text output files have the extension ``csv'' (for ``Comma-Separated Values'') and binary output files have the extension ``bsv'' (for ``Binary Simulated Values'').

\item \texttt{BINARY}---an optional keyword used to indicate that the output file should be written in binary (unformatted) form.

\item \texttt{obsname}---string of 1 to 40 nonblank characters used to identify the observation. The identifier need not be unique; however, identification and post-processing of observations in the output files are facilitated if each observation is given a unique name.

\item \texttt{obstype}---a string of characters used to identify the observation type.

\item \texttt{id}---Text identifying cell where observation is located. For packages other than NPF, if boundary names are defined in the corresponding package input file, ID can be a boundary name. Otherwise ID is a cellid. If the model discretization is type DIS, cellid is three integers (layer, row, column). If the discretization is DISV, cellid is two integers (layer, cell number). If the discretization is DISU, cellid is one integer (node number).

\item \texttt{id2}---Text identifying cell adjacent to cell identified by ID. The form of ID2 is as described for ID. ID2 is used for intercell-flow observations of a GWF model, for three observation types of the LAK Package, for two observation types of the MAW Package, and one observation type of the UZF Package.

\end{description}


\end{description}


\begin{longtable}{p{2cm} p{2.75cm} p{2cm} p{1.25cm} p{7cm}}
\caption{Available GWT model observation types} \tabularnewline

\hline
\hline
\textbf{Model} & \textbf{Observation type} & \textbf{ID} & \textbf{ID2} & \textbf{Description} \\
\hline
\endhead

\hline
\endfoot


GWT Model observations include the simulated groundwater concentration (\texttt{concentration}), and the mass flow, with units of mass per time, between two connected cells (\texttt{flow-ja-face}). The data required for each GWT Model observation type is defined in table~\ref{table:gwtobstype}. For \texttt{flow-ja-face} observation types, negative and positive values represent a loss from and gain to the \texttt{cellid} specified for ID, respectively.

\subsubsection{Structure of Blocks}
\vspace{5mm}

\noindent \textit{FOR EACH SIMULATION}
\lstinputlisting[style=blockdefinition]{./mf6ivar/tex/utl-obs-options.dat}
\lstinputlisting[style=blockdefinition]{./mf6ivar/tex/utl-obs-continuous.dat}

\subsubsection{Explanation of Variables}
\begin{description}
\input{./mf6ivar/tex/utl-obs-desc.tex}
\end{description}


\begin{longtable}{p{2cm} p{2.75cm} p{2cm} p{1.25cm} p{7cm}}
\caption{Available GWT model observation types} \tabularnewline

\hline
\hline
\textbf{Model} & \textbf{Observation type} & \textbf{ID} & \textbf{ID2} & \textbf{Description} \\
\hline
\endhead

\hline
\endfoot

\input{../Common/gwt-obs.tex}
\label{table:gwtobstype}
\end{longtable}

\vspace{5mm}
\subsubsection{Example Observation Input File}

An example GWT Model observation file is shown below.

\lstinputlisting[style=inputfile]{./mf6ivar/examples/utl-obs-gwt-example.dat}


\label{table:gwtobstype}
\end{longtable}

\vspace{5mm}
\subsubsection{Example Observation Input File}

An example GWT Model observation file is shown below.

\lstinputlisting[style=inputfile]{./mf6ivar/examples/utl-obs-gwt-example.dat}


\label{table:gwtobstype}
\end{longtable}

\vspace{5mm}
\subsubsection{Example Observation Input File}

An example GWT Model observation file is shown below.

\lstinputlisting[style=inputfile]{./mf6ivar/examples/utl-obs-gwt-example.dat}



\newpage
\subsection{Advection (ADV) Package}
Advection (ADV) Package information is read from the file that is specified by ``ADV6'' as the file type.  Only one ADV Package can be specified for a GWT model. 

\vspace{5mm}
\subsubsection{Structure of Blocks}
\lstinputlisting[style=blockdefinition]{./mf6ivar/tex/gwt-adv-options.dat}

\vspace{5mm}
\subsubsection{Explanation of Variables}
\begin{description}
% DO NOT MODIFY THIS FILE DIRECTLY.  IT IS CREATED BY mf6ivar.py 

\item \textbf{Block: OPTIONS}

\begin{description}
\item \texttt{scheme}---scheme used to solve the advection term.  Can be upstream, central, or TVD.  If not specified, upstream weighting is the default weighting scheme.

\end{description}


\end{description}

\vspace{5mm}
\subsubsection{Example Input File}
\lstinputlisting[style=inputfile]{./mf6ivar/examples/gwt-adv-example.dat}



\newpage
\subsection{Dispersion (DSP) Package}
Dispersion (DSP) Package information is read from the file that is specified by ``DSP6'' as the file type.  Only one DSP Package can be specified for a GWT model.  The DSP Package is based on the mathematical formulation presented for the XT3D option of the NPF Package available to represent full three-dimensional anisotropy in groundwater flow.  XT3D can be computationally expensive and can be turned off to use a simplified and approximate form of the dispersion equations.  For most problems, however, XT3D will be required to accurately represent dispersion.

\vspace{5mm}
\subsubsection{Structure of Blocks}
\lstinputlisting[style=blockdefinition]{./mf6ivar/tex/gwt-dsp-options.dat}
\lstinputlisting[style=blockdefinition]{./mf6ivar/tex/gwt-dsp-griddata.dat}

\vspace{5mm}
\subsubsection{Explanation of Variables}
\begin{description}
% DO NOT MODIFY THIS FILE DIRECTLY.  IT IS CREATED BY mf6ivar.py 

\item \textbf{Block: OPTIONS}

\begin{description}
\item \texttt{XT3D\_OFF}---deactivate the xt3d method and use the faster and less accurate approximation.  This option may provide a fast and accurate solution under some circumstances, such as when flow aligns with the model grid, there is no mechanical dispersion, or when the longitudinal and transverse dispersivities are equal.  This option may also be used to assess the computational demand of the XT3D approach by noting the run time differences with and without this option on.

\item \texttt{XT3D\_RHS}---add xt3d terms to right-hand side, when possible.  This option uses less memory, but may require more iterations.

\end{description}
\item \textbf{Block: GRIDDATA}

\begin{description}
\item \texttt{diffc}---effective molecular diffusion coefficient.

\item \texttt{alh}---longitudinal dispersivity in horizontal direction.  If flow is strictly horizontal, then this is the longitudinal dispersivity that will be used.  If flow is not strictly horizontal or strictly vertical, then the longitudinal dispersivity is a function of both ALH and ALV.  If mechanical dispersion is represented (by specifying any dispersivity values) then this array is required.

\item \texttt{alv}---longitudinal dispersivity in vertical direction.  If flow is strictly vertical, then this is the longitudinal dispsersivity value that will be used.  If flow is not strictly horizontal or strictly vertical, then the longitudinal dispersivity is a function of both ALH and ALV.  If this value is not specified and mechanical dispersion is represented, then this array is set equal to ALH.

\item \texttt{ath1}---transverse dispersivity in horizontal direction.  This is the transverse dispersivity value for the second ellipsoid axis.  If flow is strictly horizontal and directed in the x direction (along a row for a regular grid), then this value controls spreading in the y direction.  If mechanical dispersion is represented (by specifying any dispersivity values) then this array is required.

\item \texttt{ath2}---transverse dispersivity in horizontal direction.  This is the transverse dispersivity value for the third ellipsoid axis.  If flow is strictly horizontal and directed in the x direction (along a row for a regular grid), then this value controls spreading in the z direction.  If this value is not specified and mechanical dispersion is represented, then this array is set equal to ATH1.

\item \texttt{atv}---transverse dispersivity when flow is in vertical direction.  If flow is strictly vertical and directed in the z direction, then this value controls spreading in the x and y directions.  If this value is not specified and mechanical dispersion is represented, then this array is set equal to ATH2.

\end{description}


\end{description}

\vspace{5mm}
\subsubsection{Example Input File}
\lstinputlisting[style=inputfile]{./mf6ivar/examples/gwt-dsp-example.dat}



\newpage
\subsection{Source and Sink Mixing (SSM) Package}
Source and Sink Mixing (SSM) Package information is read from the file that is specified by ``SSM6'' as the file type.  Only one SSM Package can be specified for a GWT model.  

The SSM Package is used to add or remove solute mass from GWT model cells based on inflows and outflows from GWF stress packages.  If a GWF stress package provides flow into a model cell, that flow can be assigned a user-specified concentration.  This user-specified concentration must be entered as an auxiliary variable in the flow package.  In the SOURCES block below, the user provides the name of the package and the name of the auxiliary variable containing concentration values for each boundary.  As described below for srctype, there are multiple options for defining this behavior.

If the user does not enter a record for a GWF stress package in the SOURCES block, then inflow to the GWT model is assigned a concentration value of zero.  For negative flow rates in GWF stress packages (sinks), the sink concentration is set to the calculated cell concentration.

\vspace{5mm}
\subsubsection{Structure of Blocks}
\lstinputlisting[style=blockdefinition]{./mf6ivar/tex/gwt-ssm-options.dat}
\lstinputlisting[style=blockdefinition]{./mf6ivar/tex/gwt-ssm-sources.dat}

\vspace{5mm}
\subsubsection{Explanation of Variables}
\begin{description}
% DO NOT MODIFY THIS FILE DIRECTLY.  IT IS CREATED BY mf6ivar.py 

\item \textbf{Block: OPTIONS}

\begin{description}
\item \texttt{PRINT\_FLOWS}---keyword to indicate that the list of SSM flow rates will be printed to the listing file for every stress period time step in which ``BUDGET PRINT'' is specified in Output Control.  If there is no Output Control option and ``PRINT\_FLOWS'' is specified, then flow rates are printed for the last time step of each stress period.

\item \texttt{SAVE\_FLOWS}---keyword to indicate that SSM flow terms will be written to the file specified with ``BUDGET FILEOUT'' in Output Control.

\end{description}
\item \textbf{Block: SOURCES}

\begin{description}
\item \texttt{pname}---name of the package for which an auxiliary variable contains a source concentration.

\item \texttt{srctype}---keyword indicating how concentration will be assigned for sources and sinks.  Keyword must be specified as either AUX or AUXMIXED.  For both options the user must provide an auxiliary variable in the corresponding flow package.  The auxiliary variable must have the same name as the AUXNAME value that follows.  If the AUX keyword is specified, then the auxiliary variable specified by the user will be assigned as the concenration value for groundwater sources (flows with a positive sign).  For negative flow rates (sinks), groundwater will be withdrawn from the cell at the simulated concentration of the cell.  The AUXMIXED option provides an alternative method for how to determine the concentration of sinks.  If the cell concentration is larger than the user-specified auxiliary concentration, then the concentration of groundwater withdrawn from the cell will be assigned as the user-specified concentration.  Alternatively, if the user-specified auxiliary concentration is larger than the cell concentration, then groundwater will be withdrawn at the cell concentration.  Thus, the AUXMIXED option is designed to work with the Evapotranspiration (EVT) and Recharge (RCH) Packages where water may be withdrawn at a concentration that is less than the cell concentration.

\item \texttt{auxname}---name of the auxiliary variable in the package PNAME.  This auxiliary variable must exist and be specified by the user in that package.  The values in this auxiliary variable will be used to set the concentration associated with the flows for that boundary package.

\end{description}


\end{description}

\vspace{5mm}
\subsubsection{Example Input File}
\lstinputlisting[style=inputfile]{./mf6ivar/examples/gwt-ssm-example.dat}



\newpage
\subsection{Mobile Storage and Transfer (MST) Package}
Mobile Storage and Transfer (MST) Package information is read from the file that is specified by ``MST6'' as the file type.  Only one MST Package can be specified for a GWT model. 

\vspace{5mm}
\subsubsection{Structure of Blocks}
\lstinputlisting[style=blockdefinition]{./mf6ivar/tex/gwt-mst-options.dat}
\lstinputlisting[style=blockdefinition]{./mf6ivar/tex/gwt-mst-griddata.dat}

\vspace{5mm}
\subsubsection{Explanation of Variables}
\begin{description}
% DO NOT MODIFY THIS FILE DIRECTLY.  IT IS CREATED BY mf6ivar.py 

\item \textbf{Block: OPTIONS}

\begin{description}
\item \texttt{SAVE\_FLOWS}---keyword to indicate that MST flow terms will be written to the file specified with ``BUDGET FILEOUT'' in Output Control.

\item \texttt{FIRST\_ORDER\_DECAY}---is a text keyword to indicate that first-order decay will occur.  Use of this keyword requires that DECAY and DECAY\_SORBED (if sorbtion is active) are specified in the GRIDDATA block.

\item \texttt{ZERO\_ORDER\_DECAY}---is a text keyword to indicate that zero-order decay will occur.  Use of this keyword requires that DECAY and DECAY\_SORBED (if sorbtion is active) are specified in the GRIDDATA block.

\item \texttt{sorption}---is a text keyword to indicate that sorption will be activated.  Valid sorption options include LINEAR, FREUNDLICH, and LANGMUIR.  Use of this keyword requires that BULK\_DENSITY and DISTCOEF are specified in the GRIDDATA block.  If sorption is specified as FREUNDLICH or LANGMUIR then SP2 is also required in the GRIDDATA block.

\end{description}
\item \textbf{Block: GRIDDATA}

\begin{description}
\item \texttt{porosity}---is the aquifer porosity.

\item \texttt{decay}---is the rate coefficient for first or zero-order decay for the aqueous phase of the mobile domain.  A negative value indicates solute production.  The dimensions of decay for first-order decay is one over time.  The dimensions of decay for zero-order decay is mass per length cubed per time.  decay will have no affect on simulation results unless either first- or zero-order decay is specified in the options block.

\item \texttt{decay\_sorbed}---is the rate coefficient for first or zero-order decay for the sorbed phase of the mobile domain.  A negative value indicates solute production.  The dimensions of decay\_sorbed for first-order decay is one over time.  The dimensions of decay\_sorbed for zero-order decay is mass of solute per mass of aquifer per time.  If decay\_sorbed is not specified and both decay and sorbtion are active, then the program will terminate with an error.  decay\_sorbed will have no affect on simulation results unless the SORPTION keyword and either first- or zero-order decay are specified in the options block.

\item \texttt{bulk\_density}---is the bulk density of the aquifer in mass per length cubed.  bulk\_density is not required unless the SORBTION keyword is specified.

\item \texttt{distcoef}---is the distribution coefficient for the equilibrium-controlled linear sorption isotherm in dimensions of length cubed per mass.  distcoef is not required unless the SORPTION keyword is specified.

\item \texttt{sp2}---is the exponent for the Freundlich isotherm and the sorption capacity for the Langmuir isotherm.

\end{description}


\end{description}

\vspace{5mm}
\subsubsection{Example Input File}
\lstinputlisting[style=inputfile]{./mf6ivar/examples/gwt-mst-example.dat}



\newpage
\subsection{Immobile Storage and Transfer (IST) Package}
Immobile Storage and Transfer (IMD) Package information is read from the file that is specified by ``IST6'' as the file type.  Any number of IST Packages can be specified for a single GWT model.  This allows the user to specify triple porosity systems, or systems with as many immobile domains as necessary. 

\vspace{5mm}
\subsubsection{Structure of Blocks}
\lstinputlisting[style=blockdefinition]{./mf6ivar/tex/gwt-ist-options.dat}
\lstinputlisting[style=blockdefinition]{./mf6ivar/tex/gwt-ist-griddata.dat}

\vspace{5mm}
\subsubsection{Explanation of Variables}
\begin{description}
% DO NOT MODIFY THIS FILE DIRECTLY.  IT IS CREATED BY mf6ivar.py 

\item \textbf{Block: OPTIONS}

\begin{description}
\item \texttt{SAVE\_FLOWS}---keyword to indicate that IST flow terms will be written to the file specified with ``BUDGET FILEOUT'' in Output Control.

\item \texttt{SORPTION}---is a text keyword to indicate that sorption will be activated.  Use of this keyword requires that BULK\_DENSITY and DISTCOEF are specified in the GRIDDATA block.  The linear sorption isotherm is the only isotherm presently supported in the IST Package.

\item \texttt{FIRST\_ORDER\_DECAY}---is a text keyword to indicate that first-order decay will occur.  Use of this keyword requires that DECAY and DECAY\_SORBED (if sorbtion is active) are specified in the GRIDDATA block.

\item \texttt{ZERO\_ORDER\_DECAY}---is a text keyword to indicate that zero-order decay will occur.  Use of this keyword requires that DECAY and DECAY\_SORBED (if sorbtion is active) are specified in the GRIDDATA block.

\item \texttt{CIM}---keyword to specify that record corresponds to immobile concentration.

\item \texttt{FILEOUT}---keyword to specify that an output filename is expected next.

\item \texttt{cimfile}---name of the output file to write immobile concentrations.  This file is a binary file that has the same format and structure as a binary head and concentration file.  The value for the text variable written to the file is CIM.  Immobile domain concentrations will be written to this file at the same interval as mobile domain concentrations are saved, as specified in the GWT Model Output Control file.

\item \texttt{PRINT\_FORMAT}---keyword to specify format for printing to the listing file.

\item \texttt{columns}---number of columns for writing data.

\item \texttt{width}---width for writing each number.

\item \texttt{digits}---number of digits to use for writing a number.

\item \texttt{format}---write format can be EXPONENTIAL, FIXED, GENERAL, or SCIENTIFIC.

\end{description}
\item \textbf{Block: GRIDDATA}

\begin{description}
\item \texttt{cim}---initial concentration of the immobile domain in mass per length cubed.  If CIM is not specified, then it is assumed to be zero.

\item \texttt{thetaim}---porosity of the immobile domain specified as the volume of immobile pore space per total volume (dimensionless).

\item \texttt{zetaim}---mass transfer rate coefficient between the mobile and immobile domains, in dimensions of per time.

\item \texttt{decay}---is the rate coefficient for first or zero-order decay for the aqueous phase of the immobile domain.  A negative value indicates solute production.  The dimensions of decay for first-order decay is one over time.  The dimensions of decay for zero-order decay is mass per length cubed per time.  decay will have no affect on simulation results unless either first- or zero-order decay is specified in the options block.

\item \texttt{decay\_sorbed}---is the rate coefficient for first or zero-order decay for the sorbed phase of the immobile domain.  A negative value indicates solute production.  The dimensions of decay\_sorbed for first-order decay is one over time.  The dimensions of decay\_sorbed for zero-order decay is mass of solute per mass of aquifer per time.  If decay\_sorbed is not specified and both decay and sorbtion are active, then the program will terminate with an error.  decay\_sorbed will have no affect on simulation results unless the SORPTION keyword and either first- or zero-order decay are specified in the options block.

\item \texttt{bulk\_density}---is the bulk density of the aquifer in mass per length cubed.  bulk\_density will have no affect on simulation results unless the SORBTION keyword is specified in the options block.

\item \texttt{distcoef}---is the distribution coefficient for the equilibrium-controlled linear sorption isotherm in dimensions of length cubed per mass.  distcoef will have no affect on simulation results unless the SORBTION keyword is specified in the options block.

\end{description}


\end{description}

\vspace{5mm}
\subsubsection{Example Input File}
\lstinputlisting[style=inputfile]{./mf6ivar/examples/gwt-ist-example.dat}



\newpage
\subsection{Constant Concentration (CNC) Package}
Constant Concentration (CNC) Package information is read from the file that is specified by ``CNC6'' as the file type.  Any number of CNC Packages can be specified for a single GWT model, but the same cell cannot be designated as a constant concentration by more than one CNC entry. 

\vspace{5mm}
\subsubsection{Structure of Blocks}
\vspace{5mm}

\noindent \textit{FOR EACH SIMULATION}
\lstinputlisting[style=blockdefinition]{./mf6ivar/tex/gwt-cnc-options.dat}
\lstinputlisting[style=blockdefinition]{./mf6ivar/tex/gwt-cnc-dimensions.dat}
\vspace{5mm}
\noindent \textit{FOR ANY STRESS PERIOD}
\lstinputlisting[style=blockdefinition]{./mf6ivar/tex/gwt-cnc-period.dat}
\packageperioddescription

\vspace{5mm}
\subsubsection{Explanation of Variables}
\begin{description}
% DO NOT MODIFY THIS FILE DIRECTLY.  IT IS CREATED BY mf6ivar.py 

\item \textbf{Block: OPTIONS}

\begin{description}
\item \texttt{auxiliary}---defines an array of one or more auxiliary variable names.  There is no limit on the number of auxiliary variables that can be provided on this line; however, lists of information provided in subsequent blocks must have a column of data for each auxiliary variable name defined here.   The number of auxiliary variables detected on this line determines the value for naux.  Comments cannot be provided anywhere on this line as they will be interpreted as auxiliary variable names.  Auxiliary variables may not be used by the package, but they will be available for use by other parts of the program.  The program will terminate with an error if auxiliary variables are specified on more than one line in the options block.

\item \texttt{auxmultname}---name of auxiliary variable to be used as multiplier of concentration value.

\item \texttt{BOUNDNAMES}---keyword to indicate that boundary names may be provided with the list of constant concentration cells.

\item \texttt{PRINT\_INPUT}---keyword to indicate that the list of constant concentration information will be written to the listing file immediately after it is read.

\item \texttt{PRINT\_FLOWS}---keyword to indicate that the list of constant concentration flow rates will be printed to the listing file for every stress period time step in which ``BUDGET PRINT'' is specified in Output Control.  If there is no Output Control option and ``PRINT\_FLOWS'' is specified, then flow rates are printed for the last time step of each stress period.

\item \texttt{SAVE\_FLOWS}---keyword to indicate that constant concentration flow terms will be written to the file specified with ``BUDGET FILEOUT'' in Output Control.

\item \texttt{TS6}---keyword to specify that record corresponds to a time-series file.

\item \texttt{FILEIN}---keyword to specify that an input filename is expected next.

\item \texttt{ts6\_filename}---defines a time-series file defining time series that can be used to assign time-varying values. See the ``Time-Variable Input'' section for instructions on using the time-series capability.

\item \texttt{OBS6}---keyword to specify that record corresponds to an observations file.

\item \texttt{obs6\_filename}---name of input file to define observations for the Constant Concentration package. See the ``Observation utility'' section for instructions for preparing observation input files. Tables \ref{table:gwf-obstypetable} and \ref{table:gwt-obstypetable} lists observation type(s) supported by the Constant Concentration package.

\end{description}
\item \textbf{Block: DIMENSIONS}

\begin{description}
\item \texttt{maxbound}---integer value specifying the maximum number of constant concentrations cells that will be specified for use during any stress period.

\end{description}
\item \textbf{Block: PERIOD}

\begin{description}
\item \texttt{iper}---integer value specifying the starting stress period number for which the data specified in the PERIOD block apply.  IPER must be less than or equal to NPER in the TDIS Package and greater than zero.  The IPER value assigned to a stress period block must be greater than the IPER value assigned for the previous PERIOD block.  The information specified in the PERIOD block will continue to apply for all subsequent stress periods, unless the program encounters another PERIOD block.

\item \texttt{cellid}---is the cell identifier, and depends on the type of grid that is used for the simulation.  For a structured grid that uses the DIS input file, CELLID is the layer, row, and column.   For a grid that uses the DISV input file, CELLID is the layer and CELL2D number.  If the model uses the unstructured discretization (DISU) input file, CELLID is the node number for the cell.

\item \textcolor{blue}{\texttt{conc}---is the constant concentration value. If the Options block includes a TIMESERIESFILE entry (see the ``Time-Variable Input'' section), values can be obtained from a time series by entering the time-series name in place of a numeric value.}

\item \textcolor{blue}{\texttt{aux}---represents the values of the auxiliary variables for each constant concentration. The values of auxiliary variables must be present for each constant concentration. The values must be specified in the order of the auxiliary variables specified in the OPTIONS block.  If the package supports time series and the Options block includes a TIMESERIESFILE entry (see the ``Time-Variable Input'' section), values can be obtained from a time series by entering the time-series name in place of a numeric value.}

\item \texttt{boundname}---name of the constant concentration cell.  BOUNDNAME is an ASCII character variable that can contain as many as 40 characters.  If BOUNDNAME contains spaces in it, then the entire name must be enclosed within single quotes.

\end{description}


\end{description}

\vspace{5mm}
\subsubsection{Example Input File}
\lstinputlisting[style=inputfile]{./mf6ivar/examples/gwt-cnc-example.dat}

\vspace{5mm}
\subsubsection{Available observation types}
CNC Package observations are limited to the simulated constant concentration mass flow rate (\texttt{cnc}). The data required for the CNC Package observation type is defined in table~\ref{table:gwt-cncobstype}. Negative and positive values for an observation represent a loss from and gain to the GWT model, respectively.

\begin{longtable}{p{2cm} p{2.75cm} p{2cm} p{1.25cm} p{7cm}}
\caption{Available CNC Package observation types} \tabularnewline

\hline
\hline
\textbf{Model} & \textbf{Observation type} & \textbf{ID} & \textbf{ID2} & \textbf{Description} \\
\hline
\endhead

\hline
\endfoot

CNC & cnc & cellid or boundname & -- & Mass flow between the groundwater system and a constant-concentration boundary or a group of cells with constant-concentration boundaries.

\label{table:gwt-cncobstype}
\end{longtable}

\vspace{5mm}
\subsubsection{Example Observation Input File}
\lstinputlisting[style=inputfile]{./mf6ivar/examples/gwt-cnc-example-obs.dat}


\newpage
\subsection{Mass Source Loading (SRC) Package}
Input to the Mass Source Loading (SRC) Package is read from the file that has type ``SRC6'' in the Name File.  Any number of SRC Packages can be specified for a single groundwater transport model.

\vspace{5mm}
\subsubsection{Structure of Blocks}
\vspace{5mm}

\noindent \textit{FOR EACH SIMULATION}
\lstinputlisting[style=blockdefinition]{./mf6ivar/tex/gwt-src-options.dat}
\lstinputlisting[style=blockdefinition]{./mf6ivar/tex/gwt-src-dimensions.dat}
\vspace{5mm}
\noindent \textit{FOR ANY STRESS PERIOD}
\lstinputlisting[style=blockdefinition]{./mf6ivar/tex/gwt-src-period.dat}
\packageperioddescription

\vspace{5mm}
\subsubsection{Explanation of Variables}
\begin{description}
% DO NOT MODIFY THIS FILE DIRECTLY.  IT IS CREATED BY mf6ivar.py 

\item \textbf{Block: OPTIONS}

\begin{description}
\item \texttt{auxiliary}---defines an array of one or more auxiliary variable names.  There is no limit on the number of auxiliary variables that can be provided on this line; however, lists of information provided in subsequent blocks must have a column of data for each auxiliary variable name defined here.   The number of auxiliary variables detected on this line determines the value for naux.  Comments cannot be provided anywhere on this line as they will be interpreted as auxiliary variable names.  Auxiliary variables may not be used by the package, but they will be available for use by other parts of the program.  The program will terminate with an error if auxiliary variables are specified on more than one line in the options block.

\item \texttt{auxmultname}---name of auxiliary variable to be used as multiplier of mass loading rate.

\item \texttt{BOUNDNAMES}---keyword to indicate that boundary names may be provided with the list of mass source cells.

\item \texttt{PRINT\_INPUT}---keyword to indicate that the list of mass source information will be written to the listing file immediately after it is read.

\item \texttt{PRINT\_FLOWS}---keyword to indicate that the list of mass source flow rates will be printed to the listing file for every stress period time step in which ``BUDGET PRINT'' is specified in Output Control.  If there is no Output Control option and ``PRINT\_FLOWS'' is specified, then flow rates are printed for the last time step of each stress period.

\item \texttt{SAVE\_FLOWS}---keyword to indicate that mass source flow terms will be written to the file specified with ``BUDGET FILEOUT'' in Output Control.

\item \texttt{TS6}---keyword to specify that record corresponds to a time-series file.

\item \texttt{FILEIN}---keyword to specify that an input filename is expected next.

\item \texttt{ts6\_filename}---defines a time-series file defining time series that can be used to assign time-varying values. See the ``Time-Variable Input'' section for instructions on using the time-series capability.

\item \texttt{OBS6}---keyword to specify that record corresponds to an observations file.

\item \texttt{obs6\_filename}---name of input file to define observations for the Mass Source package. See the ``Observation utility'' section for instructions for preparing observation input files. Tables \ref{table:gwf-obstypetable} and \ref{table:gwt-obstypetable} lists observation type(s) supported by the Mass Source package.

\end{description}
\item \textbf{Block: DIMENSIONS}

\begin{description}
\item \texttt{maxbound}---integer value specifying the maximum number of sources cells that will be specified for use during any stress period.

\end{description}
\item \textbf{Block: PERIOD}

\begin{description}
\item \texttt{iper}---integer value specifying the starting stress period number for which the data specified in the PERIOD block apply.  IPER must be less than or equal to NPER in the TDIS Package and greater than zero.  The IPER value assigned to a stress period block must be greater than the IPER value assigned for the previous PERIOD block.  The information specified in the PERIOD block will continue to apply for all subsequent stress periods, unless the program encounters another PERIOD block.

\item \texttt{cellid}---is the cell identifier, and depends on the type of grid that is used for the simulation.  For a structured grid that uses the DIS input file, CELLID is the layer, row, and column.   For a grid that uses the DISV input file, CELLID is the layer and CELL2D number.  If the model uses the unstructured discretization (DISU) input file, CELLID is the node number for the cell.

\item \textcolor{blue}{\texttt{smassrate}---is the mass source loading rate. A positive value indicates addition of solute mass and a negative value indicates removal of solute mass. If the Options block includes a TIMESERIESFILE entry (see the ``Time-Variable Input'' section), values can be obtained from a time series by entering the time-series name in place of a numeric value.}

\item \textcolor{blue}{\texttt{aux}---represents the values of the auxiliary variables for each mass source. The values of auxiliary variables must be present for each mass source. The values must be specified in the order of the auxiliary variables specified in the OPTIONS block.  If the package supports time series and the Options block includes a TIMESERIESFILE entry (see the ``Time-Variable Input'' section), values can be obtained from a time series by entering the time-series name in place of a numeric value.}

\item \texttt{boundname}---name of the mass source cell.  BOUNDNAME is an ASCII character variable that can contain as many as 40 characters.  If BOUNDNAME contains spaces in it, then the entire name must be enclosed within single quotes.

\end{description}


\end{description}

\vspace{5mm}
\subsubsection{Example Input File}
\lstinputlisting[style=inputfile]{./mf6ivar/examples/gwt-src-example.dat}

\vspace{5mm}
\subsubsection{Available observation types}
Mass Source Loading Package observations include the simulated source loading rates (\texttt{src}). The data required for each SRC Package observation type is defined in table~\ref{table:gwt-srcobstype}. The \texttt{src} observation is equal to the simulated mass source loading rate. Negative and positive values for an observation represent a loss from and gain to the GWT model, respectively.

\begin{longtable}{p{2cm} p{2.75cm} p{2cm} p{1.25cm} p{7cm}}
\caption{Available SRC Package observation types} \tabularnewline

\hline
\hline
\textbf{Stress Package} & \textbf{Observation type} & \textbf{ID} & \textbf{ID2} & \textbf{Description} \\
\hline
\endhead

\hline
\endfoot

SRC & src & cellid or boundname & -- & Mass source loading rate between the groundwater system and a mass source loading boundary or a group of  boundaries.
\label{table:gwt-srcobstype}
\end{longtable}

\vspace{5mm}
\subsubsection{Example Observation Input File}
\lstinputlisting[style=inputfile]{./mf6ivar/examples/gwt-src-example-obs.dat}


\newpage
\subsection{Streamflow Transport (SFT) Package}
Streamflow Transport (SFT) Package information is read from the file that is specified by ``SFT6'' as the file type.  There can be as many SFT Packages as necessary for a GWT model. Each SFT Package is designed to work with flows from a corresponding GWF SFR Package. By default \mf uses the SFT package name to determine which SFR Package corresponds to the SFT Package.  Therefore, the package name of the SFT Package (as specified in the GWT name file) must match with the name of the corresponding SFR Package (as specified in the GWF name file).  Alternatively, the name of the flow package can be specified using the FLOW\_PACKAGE\_NAME keyword in the options block.  The GWT SFT Package cannot be used without a corresponding GWF SFR Package.

The SFT Package does not have a dimensions block; instead, dimensions for the SFT Package are set using the dimensions from the corresponding SFR Package.  For example, the SFR Package requires specification of the number of reaches (NREACHES).  SFT sets the number of reaches equal to NREACHES.  Therefore, the PACKAGEDATA block below must have NREACHES entries in it.

\vspace{5mm}
\subsubsection{Structure of Blocks}
\lstinputlisting[style=blockdefinition]{./mf6ivar/tex/gwt-sft-options.dat}
\lstinputlisting[style=blockdefinition]{./mf6ivar/tex/gwt-sft-packagedata.dat}
\lstinputlisting[style=blockdefinition]{./mf6ivar/tex/gwt-sft-period.dat}

\vspace{5mm}
\subsubsection{Explanation of Variables}
\begin{description}
% DO NOT MODIFY THIS FILE DIRECTLY.  IT IS CREATED BY mf6ivar.py 

\item \textbf{Block: OPTIONS}

\begin{description}
\item \texttt{flow\_package\_name}---keyword to specify the name of the corresponding flow package.  If not specified, then the corresponding flow package must have the same name as this advanced transport package (the name associated with this package in the GWT name file).

\item \texttt{auxiliary}---defines an array of one or more auxiliary variable names.  There is no limit on the number of auxiliary variables that can be provided on this line; however, lists of information provided in subsequent blocks must have a column of data for each auxiliary variable name defined here.   The number of auxiliary variables detected on this line determines the value for naux.  Comments cannot be provided anywhere on this line as they will be interpreted as auxiliary variable names.  Auxiliary variables may not be used by the package, but they will be available for use by other parts of the program.  The program will terminate with an error if auxiliary variables are specified on more than one line in the options block.

\item \texttt{flow\_package\_auxiliary\_name}---keyword to specify the name of an auxiliary variable in the corresponding flow package.  If specified, then the simulated concentrations from this advanced transport package will be copied into the auxiliary variable specified with this name.  Note that the flow package must have an auxiliary variable with this name or the program will terminate with an error.  If the flows for this advanced transport package are read from a file, then this option will have no affect.

\item \texttt{BOUNDNAMES}---keyword to indicate that boundary names may be provided with the list of reach cells.

\item \texttt{PRINT\_INPUT}---keyword to indicate that the list of reach information will be written to the listing file immediately after it is read.

\item \texttt{PRINT\_CONCENTRATION}---keyword to indicate that the list of reach stages will be printed to the listing file for every stress period in which ``HEAD PRINT'' is specified in Output Control.  If there is no Output Control option and PRINT\_STAGE is specified, then stages are printed for the last time step of each stress period.

\item \texttt{PRINT\_FLOWS}---keyword to indicate that the list of reach flow rates will be printed to the listing file for every stress period time step in which ``BUDGET PRINT'' is specified in Output Control.  If there is no Output Control option and ``PRINT\_FLOWS'' is specified, then flow rates are printed for the last time step of each stress period.

\item \texttt{SAVE\_FLOWS}---keyword to indicate that reach flow terms will be written to the file specified with ``BUDGET FILEOUT'' in Output Control.

\item \texttt{CONCENTRATION}---keyword to specify that record corresponds to concentration.

\item \texttt{concfile}---name of the binary output file to write concentration information.

\item \texttt{BUDGET}---keyword to specify that record corresponds to the budget.

\item \texttt{FILEOUT}---keyword to specify that an output filename is expected next.

\item \texttt{budgetfile}---name of the binary output file to write budget information.

\item \texttt{TS6}---keyword to specify that record corresponds to a time-series file.

\item \texttt{FILEIN}---keyword to specify that an input filename is expected next.

\item \texttt{ts6\_filename}---defines a time-series file defining time series that can be used to assign time-varying values. See the ``Time-Variable Input'' section for instructions on using the time-series capability.

\item \texttt{OBS6}---keyword to specify that record corresponds to an observations file.

\item \texttt{obs6\_filename}---name of input file to define observations for the SFT package. See the ``Observation utility'' section for instructions for preparing observation input files. Tables \ref{table:gwf-obstypetable} and \ref{table:gwt-obstypetable} lists observation type(s) supported by the SFT package.

\end{description}
\item \textbf{Block: PACKAGEDATA}

\begin{description}
\item \texttt{rno}---integer value that defines the reach number associated with the specified PACKAGEDATA data on the line. RNO must be greater than zero and less than or equal to NREACHES. Reach information must be specified for every reach or the program will terminate with an error.  The program will also terminate with an error if information for a reach is specified more than once.

\item \texttt{strt}---real value that defines the starting concentration for the reach.

\item \textcolor{blue}{\texttt{aux}---represents the values of the auxiliary variables for each reach. The values of auxiliary variables must be present for each reach. The values must be specified in the order of the auxiliary variables specified in the OPTIONS block.  If the package supports time series and the Options block includes a TIMESERIESFILE entry (see the ``Time-Variable Input'' section), values can be obtained from a time series by entering the time-series name in place of a numeric value.}

\item \texttt{boundname}---name of the reach cell.  BOUNDNAME is an ASCII character variable that can contain as many as 40 characters.  If BOUNDNAME contains spaces in it, then the entire name must be enclosed within single quotes.

\end{description}
\item \textbf{Block: PERIOD}

\begin{description}
\item \texttt{iper}---integer value specifying the starting stress period number for which the data specified in the PERIOD block apply.  IPER must be less than or equal to NPER in the TDIS Package and greater than zero.  The IPER value assigned to a stress period block must be greater than the IPER value assigned for the previous PERIOD block.  The information specified in the PERIOD block will continue to apply for all subsequent stress periods, unless the program encounters another PERIOD block.

\item \texttt{rno}---integer value that defines the reach number associated with the specified PERIOD data on the line. RNO must be greater than zero and less than or equal to NREACHES.

\item \texttt{reachsetting}---line of information that is parsed into a keyword and values.  Keyword values that can be used to start the REACHSETTING string include: STATUS, CONCENTRATION, RAINFALL, EVAPORATION, RUNOFF, and AUXILIARY.  These settings are used to assign the concentration of associated with the corresponding flow terms.  Concentrations cannot be specified for all flow terms.  For example, the Streamflow Package supports a ``DIVERSION'' flow term.  Diversion water will be routed using the calculated concentration of the reach.

\begin{lstlisting}[style=blockdefinition]
STATUS <status>
CONCENTRATION <@concentration@>
RAINFALL <@rainfall@>
EVAPORATION <@evaporation@>
RUNOFF <@runoff@>
INFLOW <@inflow@>
AUXILIARY <auxname> <@auxval@> 
\end{lstlisting}

\item \texttt{status}---keyword option to define reach status.  STATUS can be ACTIVE, INACTIVE, or CONSTANT. By default, STATUS is ACTIVE, which means that concentration will be calculated for the reach.  If a reach is inactive, then there will be no solute mass fluxes into or out of the reach and the inactive value will be written for the reach concentration.  If a reach is constant, then the concentration for the reach will be fixed at the user specified value.

\item \textcolor{blue}{\texttt{concentration}---real or character value that defines the concentration for the reach. The specified CONCENTRATION is only applied if the reach is a constant concentration reach. If the Options block includes a TIMESERIESFILE entry (see the ``Time-Variable Input'' section), values can be obtained from a time series by entering the time-series name in place of a numeric value.}

\item \textcolor{blue}{\texttt{rainfall}---real or character value that defines the rainfall solute concentration $(ML^{-3})$ for the reach. If the Options block includes a TIMESERIESFILE entry (see the ``Time-Variable Input'' section), values can be obtained from a time series by entering the time-series name in place of a numeric value.}

\item \textcolor{blue}{\texttt{evaporation}---real or character value that defines the concentration of evaporated water $(ML^{-3})$ for the reach. If this concentration value is larger than the simulated concentration in the reach, then the evaporated water will be removed at the same concentration as the reach.  If the Options block includes a TIMESERIESFILE entry (see the ``Time-Variable Input'' section), values can be obtained from a time series by entering the time-series name in place of a numeric value.}

\item \textcolor{blue}{\texttt{runoff}---real or character value that defines the concentration of runoff $(ML^{-3})$ for the reach. Value must be greater than or equal to zero. If the Options block includes a TIMESERIESFILE entry (see the ``Time-Variable Input'' section), values can be obtained from a time series by entering the time-series name in place of a numeric value.}

\item \textcolor{blue}{\texttt{inflow}---real or character value that defines the concentration of inflow $(ML^{-3})$ for the reach. Value must be greater than or equal to zero. If the Options block includes a TIMESERIESFILE entry (see the ``Time-Variable Input'' section), values can be obtained from a time series by entering the time-series name in place of a numeric value.}

\item \texttt{AUXILIARY}---keyword for specifying auxiliary variable.

\item \texttt{auxname}---name for the auxiliary variable to be assigned AUXVAL.  AUXNAME must match one of the auxiliary variable names defined in the OPTIONS block. If AUXNAME does not match one of the auxiliary variable names defined in the OPTIONS block the data are ignored.

\item \textcolor{blue}{\texttt{auxval}---value for the auxiliary variable. If the Options block includes a TIMESERIESFILE entry (see the ``Time-Variable Input'' section), values can be obtained from a time series by entering the time-series name in place of a numeric value.}

\end{description}


\end{description}

\vspace{5mm}
\subsubsection{Example Input File}
\lstinputlisting[style=inputfile]{./mf6ivar/examples/gwt-sft-example.dat}

\vspace{5mm}
\subsubsection{Available observation types}
Streamflow Transport Package observations include reach concentration and all of the terms that contribute to the continuity equation for each reach. Additional SFT Package observations include mass flow rates for individual reaches, or groups of reaches. The data required for each SFT Package observation type is defined in table~\ref{table:gwt-sftobstype}. Negative and positive values for \texttt{sft} observations represent a loss from and gain to the GWT model, respectively. For all other flow terms, negative and positive values represent a loss from and gain from the SFT package, respectively.

\begin{longtable}{p{2cm} p{2.75cm} p{2cm} p{1.25cm} p{7cm}}
\caption{Available SFT Package observation types} \tabularnewline

\hline
\hline
\textbf{Stress Package} & \textbf{Observation type} & \textbf{ID} & \textbf{ID2} & \textbf{Description} \\
\hline
\endfirsthead

\captionsetup{textformat=simple}
\caption*{\textbf{Table \arabic{table}.}{\quad}Available SFT Package observation types.---Continued} \tabularnewline

\hline
\hline
\textbf{Stress Package} & \textbf{Observation type} & \textbf{ID} & \textbf{ID2} & \textbf{Description} \\
\hline
\endhead


\hline
\endfoot

% general APT observations
SFT & concentration & rno or boundname & -- & Reach concentration. If boundname is specified, boundname must be unique for each reach. \\
SFT & flow-ja-face & rno or boundname & rno or -- & Mass flow between two reaches.  If a boundname is specified for ID1, then the result is the total mass flow for all reaches. If a boundname is specified for ID1 then ID2 is not used.\\
SFT & storage & rno or boundname & -- & Simulated mass storage flow rate for a reach or group of reaches. \\
SFT & constant & rno or boundname & -- & Simulated mass constant-flow rate for a reach or group of reaches. \\
SFT & from-mvr & rno or boundname & -- & Simulated mass inflow into a reach or group of reaches from the MVT package. Mass inflow is calculated as the product of provider concentration and the mover flow rate. \\
SFT & to-mvr & rno or boundname & -- & Mass outflow from a reach, or a group of reaches that is available for the MVR package. If boundname is not specified for ID, then the outflow available for the MVR package from a specific reach is observed. \\
SFT & sft & rno or boundname & \texttt{iconn} or -- & Mass flow rate for a reach or group of reaches and its aquifer connection(s). If boundname is not specified for ID, then the simulated reach-aquifer flow rate at a specific reach connection is observed. In this case, ID2 must be specified and is the connection number \texttt{iconn} for reach \texttt{rno}. \\

%observations specific to the stream transport package
% rainfall evaporation runoff ext-inflow withdrawal outflow
SFT & rainfall & rno or boundname & -- & Rainfall rate applied to a reach or group of reaches multiplied by the rainfall concentration. \\
SFT & evaporation & rno or boundname & -- & Simulated evaporation rate from a reach or group of reaches multiplied by the evaporation concentration. \\
SFT & runoff & rno or boundname & -- & Runoff rate applied to a reach or group of reaches multiplied by the runoff concentration. \\
SFT & ext-inflow & rno or boundname & -- & Mass inflow into a reach or group of reaches calculated as the external inflow rate multiplied by the inflow concentration. \\
SFT & ext-outflow & rno or boundname & -- & External outflow from a reach or group of reaches to an external boundary. If boundname is not specified for ID, then the external outflow from a specific reach is observed. In this case, ID is the reach rno.

\label{table:gwt-sftobstype}
\end{longtable}

\vspace{5mm}
\subsubsection{Example Observation Input File}
\lstinputlisting[style=inputfile]{./mf6ivar/examples/gwt-sft-example-obs.dat}




\newpage
\subsection{Lake Transport (LKT) Package}
Lake Transport (LKT) Package information is read from the file that is specified by ``LKT6'' as the file type.  There can be as many LKT Packages as necessary for a GWT model. Each LKT Package is designed to work with flows from a single corresponding GWF LAK Package. By default \mf uses the LKT package name to determine which LAK Package corresponds to the LKT Package.  Therefore, the package name of the LKT Package (as specified in the GWT name file) must match with the name of the corresponding LAK Package (as specified in the GWF name file).  Alternatively, the name of the flow package can be specified using the FLOW\_PACKAGE\_NAME keyword in the options block.  The GWT LKT Package cannot be used without a corresponding GWF LAK Package.

The LKT Package does not have a dimensions block; instead, dimensions for the LKT Package are set using the dimensions from the corresponding LAK Package.  For example, the LAK Package requires specification of the number of lakes (NLAKES).  LKT sets the number of lakes equal to NLAKES.  Therefore, the PACKAGEDATA block below must have NLAKES entries in it.

\vspace{5mm}
\subsubsection{Structure of Blocks}
\lstinputlisting[style=blockdefinition]{./mf6ivar/tex/gwt-lkt-options.dat}
\lstinputlisting[style=blockdefinition]{./mf6ivar/tex/gwt-lkt-packagedata.dat}
\lstinputlisting[style=blockdefinition]{./mf6ivar/tex/gwt-lkt-period.dat}

\vspace{5mm}
\subsubsection{Explanation of Variables}
\begin{description}
% DO NOT MODIFY THIS FILE DIRECTLY.  IT IS CREATED BY mf6ivar.py 

\item \textbf{Block: OPTIONS}

\begin{description}
\item \texttt{flow\_package\_name}---keyword to specify the name of the corresponding flow package.  If not specified, then the corresponding flow package must have the same name as this advanced transport package (the name associated with this package in the GWT name file).

\item \texttt{auxiliary}---defines an array of one or more auxiliary variable names.  There is no limit on the number of auxiliary variables that can be provided on this line; however, lists of information provided in subsequent blocks must have a column of data for each auxiliary variable name defined here.   The number of auxiliary variables detected on this line determines the value for naux.  Comments cannot be provided anywhere on this line as they will be interpreted as auxiliary variable names.  Auxiliary variables may not be used by the package, but they will be available for use by other parts of the program.  The program will terminate with an error if auxiliary variables are specified on more than one line in the options block.

\item \texttt{flow\_package\_auxiliary\_name}---keyword to specify the name of an auxiliary variable in the corresponding flow package.  If specified, then the simulated concentrations from this advanced transport package will be copied into the auxiliary variable specified with this name.  Note that the flow package must have an auxiliary variable with this name or the program will terminate with an error.  If the flows for this advanced transport package are read from a file, then this option will have no affect.

\item \texttt{BOUNDNAMES}---keyword to indicate that boundary names may be provided with the list of lake cells.

\item \texttt{PRINT\_INPUT}---keyword to indicate that the list of lake information will be written to the listing file immediately after it is read.

\item \texttt{PRINT\_CONCENTRATION}---keyword to indicate that the list of lake concentration will be printed to the listing file for every stress period in which ``HEAD PRINT'' is specified in Output Control.  If there is no Output Control option and PRINT\_CONCENTRATION is specified, then concentration are printed for the last time step of each stress period.

\item \texttt{PRINT\_FLOWS}---keyword to indicate that the list of lake flow rates will be printed to the listing file for every stress period time step in which ``BUDGET PRINT'' is specified in Output Control.  If there is no Output Control option and ``PRINT\_FLOWS'' is specified, then flow rates are printed for the last time step of each stress period.

\item \texttt{SAVE\_FLOWS}---keyword to indicate that lake flow terms will be written to the file specified with ``BUDGET FILEOUT'' in Output Control.

\item \texttt{CONCENTRATION}---keyword to specify that record corresponds to concentration.

\item \texttt{concfile}---name of the binary output file to write concentration information.

\item \texttt{BUDGET}---keyword to specify that record corresponds to the budget.

\item \texttt{FILEOUT}---keyword to specify that an output filename is expected next.

\item \texttt{budgetfile}---name of the binary output file to write budget information.

\item \texttt{TS6}---keyword to specify that record corresponds to a time-series file.

\item \texttt{FILEIN}---keyword to specify that an input filename is expected next.

\item \texttt{ts6\_filename}---defines a time-series file defining time series that can be used to assign time-varying values. See the ``Time-Variable Input'' section for instructions on using the time-series capability.

\item \texttt{OBS6}---keyword to specify that record corresponds to an observations file.

\item \texttt{obs6\_filename}---name of input file to define observations for the LKT package. See the ``Observation utility'' section for instructions for preparing observation input files. Tables \ref{table:gwf-obstypetable} and \ref{table:gwt-obstypetable} lists observation type(s) supported by the LKT package.

\end{description}
\item \textbf{Block: PACKAGEDATA}

\begin{description}
\item \texttt{lakeno}---integer value that defines the lake number associated with the specified PACKAGEDATA data on the line. LAKENO must be greater than zero and less than or equal to NLAKES. Lake information must be specified for every lake or the program will terminate with an error.  The program will also terminate with an error if information for a lake is specified more than once.

\item \texttt{strt}---real value that defines the starting concentration for the lake.

\item \textcolor{blue}{\texttt{aux}---represents the values of the auxiliary variables for each lake. The values of auxiliary variables must be present for each lake. The values must be specified in the order of the auxiliary variables specified in the OPTIONS block.  If the package supports time series and the Options block includes a TIMESERIESFILE entry (see the ``Time-Variable Input'' section), values can be obtained from a time series by entering the time-series name in place of a numeric value.}

\item \texttt{boundname}---name of the lake cell.  BOUNDNAME is an ASCII character variable that can contain as many as 40 characters.  If BOUNDNAME contains spaces in it, then the entire name must be enclosed within single quotes.

\end{description}
\item \textbf{Block: PERIOD}

\begin{description}
\item \texttt{iper}---integer value specifying the starting stress period number for which the data specified in the PERIOD block apply.  IPER must be less than or equal to NPER in the TDIS Package and greater than zero.  The IPER value assigned to a stress period block must be greater than the IPER value assigned for the previous PERIOD block.  The information specified in the PERIOD block will continue to apply for all subsequent stress periods, unless the program encounters another PERIOD block.

\item \texttt{lakeno}---integer value that defines the lake number associated with the specified PERIOD data on the line. LAKENO must be greater than zero and less than or equal to NLAKES.

\item \texttt{laksetting}---line of information that is parsed into a keyword and values.  Keyword values that can be used to start the LAKSETTING string include: STATUS, CONCENTRATION, RAINFALL, EVAPORATION, RUNOFF, and AUXILIARY.  These settings are used to assign the concentration of associated with the corresponding flow terms.  Concentrations cannot be specified for all flow terms.  For example, the Lake Package supports a ``WITHDRAWAL'' flow term.  If this withdrawal term is active, then water will be withdrawn from the lake at the calculated concentration of the lake.

\begin{lstlisting}[style=blockdefinition]
STATUS <status>
CONCENTRATION <@concentration@>
RAINFALL <@rainfall@>
EVAPORATION <@evaporation@>
RUNOFF <@runoff@>
EXT-INFLOW <@ext-inflow@>
AUXILIARY <auxname> <@auxval@> 
\end{lstlisting}

\item \texttt{status}---keyword option to define lake status.  STATUS can be ACTIVE, INACTIVE, or CONSTANT. By default, STATUS is ACTIVE, which means that concentration will be calculated for the lake.  If a lake is inactive, then there will be no solute mass fluxes into or out of the lake and the inactive value will be written for the lake concentration.  If a lake is constant, then the concentration for the lake will be fixed at the user specified value.

\item \textcolor{blue}{\texttt{concentration}---real or character value that defines the concentration for the lake. The specified CONCENTRATION is only applied if the lake is a constant concentration lake. If the Options block includes a TIMESERIESFILE entry (see the ``Time-Variable Input'' section), values can be obtained from a time series by entering the time-series name in place of a numeric value.}

\item \textcolor{blue}{\texttt{rainfall}---real or character value that defines the rainfall solute concentration $(ML^{-3})$ for the lake. If the Options block includes a TIMESERIESFILE entry (see the ``Time-Variable Input'' section), values can be obtained from a time series by entering the time-series name in place of a numeric value.}

\item \textcolor{blue}{\texttt{evaporation}---real or character value that defines the concentration of evaporated water $(ML^{-3})$ for the lake. If this concentration value is larger than the simulated concentration in the lake, then the evaporated water will be removed at the same concentration as the lake.  If the Options block includes a TIMESERIESFILE entry (see the ``Time-Variable Input'' section), values can be obtained from a time series by entering the time-series name in place of a numeric value.}

\item \textcolor{blue}{\texttt{runoff}---real or character value that defines the concentration of runoff $(ML^{-3})$ for the lake. Value must be greater than or equal to zero. If the Options block includes a TIMESERIESFILE entry (see the ``Time-Variable Input'' section), values can be obtained from a time series by entering the time-series name in place of a numeric value.}

\item \textcolor{blue}{\texttt{ext-inflow}---real or character value that defines the concentration of external inflow $(ML^{-3})$ for the lake. Value must be greater than or equal to zero. If the Options block includes a TIMESERIESFILE entry (see the ``Time-Variable Input'' section), values can be obtained from a time series by entering the time-series name in place of a numeric value.}

\item \texttt{AUXILIARY}---keyword for specifying auxiliary variable.

\item \texttt{auxname}---name for the auxiliary variable to be assigned AUXVAL.  AUXNAME must match one of the auxiliary variable names defined in the OPTIONS block. If AUXNAME does not match one of the auxiliary variable names defined in the OPTIONS block the data are ignored.

\item \textcolor{blue}{\texttt{auxval}---value for the auxiliary variable. If the Options block includes a TIMESERIESFILE entry (see the ``Time-Variable Input'' section), values can be obtained from a time series by entering the time-series name in place of a numeric value.}

\end{description}


\end{description}

\vspace{5mm}
\subsubsection{Example Input File}
\lstinputlisting[style=inputfile]{./mf6ivar/examples/gwt-lkt-example.dat}

\vspace{5mm}
\subsubsection{Available observation types}
Lake Transport Package observations include lake concentration and all of the terms that contribute to the continuity equation for each lake. Additional LKT Package observations include mass flow rates for individual outlets, lakes, or groups of lakes (\texttt{outlet}). The data required for each LKT Package observation type is defined in table~\ref{table:gwt-lktobstype}. Negative and positive values for \texttt{lkt} observations represent a loss from and gain to the GWT model, respectively. For all other flow terms, negative and positive values represent a loss from and gain from the LKT package, respectively.

\begin{longtable}{p{2cm} p{2.75cm} p{2cm} p{1.25cm} p{7cm}}
\caption{Available LKT Package observation types} \tabularnewline

\hline
\hline
\textbf{Stress Package} & \textbf{Observation type} & \textbf{ID} & \textbf{ID2} & \textbf{Description} \\
\hline
\endfirsthead

\captionsetup{textformat=simple}
\caption*{\textbf{Table \arabic{table}.}{\quad}Available LKT Package observation types.---Continued} \tabularnewline

\hline
\hline
\textbf{Stress Package} & \textbf{Observation type} & \textbf{ID} & \textbf{ID2} & \textbf{Description} \\
\hline
\endhead


\hline
\endfoot

% general APT observations
LKT & concentration & lakeno or boundname & -- & Lake concentration. If boundname is specified, boundname must be unique for each lake. \\
LKT & flow-ja-face & lakeno or boundname & lakeno or -- & Mass flow between two lakes connected by an outlet.  If more than one outlet is used to connect the same two lakes, then the mass flow for only the first outlet can be observed.  If a boundname is specified for ID1, then the result is the total mass flow for all outlets for a lake. If a boundname is specified for ID1 then ID2 is not used.\\
LKT & storage & lakeno or boundname & -- & Simulated mass storage flow rate for a lake or group of lakes. \\
LKT & constant & lakeno or boundname & -- & Simulated mass constant-flow rate for a lake or group of lakes. \\
LKT & from-mvr & lakeno or boundname & -- & Simulated mass inflow into a lake or group of lakes from the MVT package. Mass inflow is calculated as the product of provider concentration and the mover flow rate. \\
LKT & to-mvr & outletno or boundname & -- & Mass outflow from a lake outlet, a lake, or a group of lakes that is available for the MVR package. If boundname is not specified for ID, then the outflow available for the MVR package from a specific lake outlet is observed. In this case, ID is the outlet number, which must be between 1 and NOUTLETS. \\
LKT & lkt & lakeno or boundname & \texttt{iconn} or -- & Mass flow rate for a lake or group of lakes and its aquifer connection(s). If boundname is not specified for ID, then the simulated lake-aquifer flow rate at a specific lake connection is observed. In this case, ID2 must be specified and is the connection number \texttt{iconn} for lake \texttt{lakeno}. \\

%observations specific to the lake package
% rainfall evaporation runoff ext-inflow withdrawal outflow
LKT & rainfall & lakeno or boundname & -- & Rainfall rate applied to a lake or group of lakes multiplied by the rainfall concentration. \\
LKT & evaporation & lakeno or boundname & -- & Simulated evaporation rate from a lake or group of lakes multiplied by the evaporation concentration. \\
LKT & runoff & lakeno or boundname & -- & Runoff rate applied to a lake or group of lakes multiplied by the runoff concentration. \\
LKT & ext-inflow & lakeno or boundname & -- & Mass inflow into a lake or group of lakes calculated as the external inflow rate multiplied by the inflow concentration. \\
LKT & withdrawal & lakeno or boundname & -- & Specified withdrawal rate from a lake or group of lakes multiplied by the simulated lake concentration. \\
LKT & ext-outflow & outletno or boundname & -- & External outflow from a lake outlet, a lake, or a group of lakes to an external boundary. If boundname is not specified for ID, then the external outflow from a specific lake outlet is observed. In this case, ID is the outlet number outletno.

%LKT & outlet-inflow & lakeno or boundname & -- & Simulated inflow from upstream lake outlets into a lake or group of lakes. \\
%LKT & inflow & lakeno or boundname & -- & Sum of specified inflow and simulated inflow from upstream lake outlets into a lake or group of lakes. \\
%LKT & outlet & outletno or boundname & -- & Simulate outlet flow rate from a lake outlet, a lake, or a group of lakes. If boundname is not specified for ID, then the flow from a specific lake outlet is observed. In this case, ID is the outlet number outletno. \\
%LKT & volume & lakeno or boundname & -- & Simulated lake volume or group of lakes. \\
%LKT & surface-area & lakeno or boundname & -- & Simulated surface area for a lake or group of lakes. \\
%LKT & wetted-area & lakeno or boundname & \texttt{iconn} or -- & Simulated wetted-area for a lake or group of lakes and its aquifer connection(s). If boundname is not specified for ID, then the wetted area of a specific lake connection is observed. In this case, ID2 must be specified and is the connection number \texttt{iconn}. \\
%LKT & conductance & lakeno or boundname & \texttt{iconn} or -- & Calculated conductance for a lake or group of lakes and its aquifer connection(s). If boundname is not specified for ID, then the calculated conductance of a specific lake connection is observed. In this case, ID2 must be specified and is the connection number \texttt{iconn}.

\label{table:gwt-lktobstype}
\end{longtable}

\vspace{5mm}
\subsubsection{Example Observation Input File}
\lstinputlisting[style=inputfile]{./mf6ivar/examples/gwt-lkt-example-obs.dat}




\newpage
\subsection{Multi-Aquifer Well Transport (MWT) Package}
Multi-Aquifer Well Transport (MWT) Package information is read from the file that is specified by ``MWT6'' as the file type.  There can be as many MWT Packages as necessary for a GWT model. Each MWT Package is designed to work with flows from a corresponding GWF MAW Package. By default \mf uses the MWT package name to determine which MAW Package corresponds to the MWT Package.  Therefore, the package name of the MWT Package (as specified in the GWT name file) must match with the name of the corresponding MAW Package (as specified in the GWF name file).  Alternatively, the name of the flow package can be specified using the FLOW\_PACKAGE\_NAME keyword in the options block.  The GWT MWT Package cannot be used without a corresponding GWF MAW Package.

The MWT Package does not have a dimensions block; instead, dimensions for the MWT Package are set using the dimensions from the corresponding MAW Package.  For example, the MAW Package requires specification of the number of wells (NMAWWELLS).  MWT sets the number of wells equal to NMAWWELLS.  Therefore, the PACKAGEDATA block below must have NMAWWELLS entries in it.

\vspace{5mm}
\subsubsection{Structure of Blocks}
\lstinputlisting[style=blockdefinition]{./mf6ivar/tex/gwt-mwt-options.dat}
\lstinputlisting[style=blockdefinition]{./mf6ivar/tex/gwt-mwt-packagedata.dat}
\lstinputlisting[style=blockdefinition]{./mf6ivar/tex/gwt-mwt-period.dat}

\vspace{5mm}
\subsubsection{Explanation of Variables}
\begin{description}
% DO NOT MODIFY THIS FILE DIRECTLY.  IT IS CREATED BY mf6ivar.py 

\item \textbf{Block: OPTIONS}

\begin{description}
\item \texttt{flow\_package\_name}---keyword to specify the name of the corresponding flow package.  If not specified, then the corresponding flow package must have the same name as this advanced transport package (the name associated with this package in the GWT name file).

\item \texttt{auxiliary}---defines an array of one or more auxiliary variable names.  There is no limit on the number of auxiliary variables that can be provided on this line; however, lists of information provided in subsequent blocks must have a column of data for each auxiliary variable name defined here.   The number of auxiliary variables detected on this line determines the value for naux.  Comments cannot be provided anywhere on this line as they will be interpreted as auxiliary variable names.  Auxiliary variables may not be used by the package, but they will be available for use by other parts of the program.  The program will terminate with an error if auxiliary variables are specified on more than one line in the options block.

\item \texttt{flow\_package\_auxiliary\_name}---keyword to specify the name of an auxiliary variable in the corresponding flow package.  If specified, then the simulated concentrations from this advanced transport package will be copied into the auxiliary variable specified with this name.  Note that the flow package must have an auxiliary variable with this name or the program will terminate with an error.  If the flows for this advanced transport package are read from a file, then this option will have no affect.

\item \texttt{BOUNDNAMES}---keyword to indicate that boundary names may be provided with the list of well cells.

\item \texttt{PRINT\_INPUT}---keyword to indicate that the list of well information will be written to the listing file immediately after it is read.

\item \texttt{PRINT\_CONCENTRATION}---keyword to indicate that the list of well concentration will be printed to the listing file for every stress period in which ``HEAD PRINT'' is specified in Output Control.  If there is no Output Control option and PRINT\_CONCENTRATION is specified, then concentration are printed for the last time step of each stress period.

\item \texttt{PRINT\_FLOWS}---keyword to indicate that the list of well flow rates will be printed to the listing file for every stress period time step in which ``BUDGET PRINT'' is specified in Output Control.  If there is no Output Control option and ``PRINT\_FLOWS'' is specified, then flow rates are printed for the last time step of each stress period.

\item \texttt{SAVE\_FLOWS}---keyword to indicate that well flow terms will be written to the file specified with ``BUDGET FILEOUT'' in Output Control.

\item \texttt{CONCENTRATION}---keyword to specify that record corresponds to concentration.

\item \texttt{concfile}---name of the binary output file to write concentration information.

\item \texttt{BUDGET}---keyword to specify that record corresponds to the budget.

\item \texttt{FILEOUT}---keyword to specify that an output filename is expected next.

\item \texttt{budgetfile}---name of the binary output file to write budget information.

\item \texttt{TS6}---keyword to specify that record corresponds to a time-series file.

\item \texttt{FILEIN}---keyword to specify that an input filename is expected next.

\item \texttt{ts6\_filename}---defines a time-series file defining time series that can be used to assign time-varying values. See the ``Time-Variable Input'' section for instructions on using the time-series capability.

\item \texttt{OBS6}---keyword to specify that record corresponds to an observations file.

\item \texttt{obs6\_filename}---name of input file to define observations for the MWT package. See the ``Observation utility'' section for instructions for preparing observation input files. Tables \ref{table:gwf-obstypetable} and \ref{table:gwt-obstypetable} lists observation type(s) supported by the MWT package.

\end{description}
\item \textbf{Block: PACKAGEDATA}

\begin{description}
\item \texttt{mawno}---integer value that defines the well number associated with the specified PACKAGEDATA data on the line. MAWNO must be greater than zero and less than or equal to NMAWWELLS. Well information must be specified for every well or the program will terminate with an error.  The program will also terminate with an error if information for a well is specified more than once.

\item \texttt{strt}---real value that defines the starting concentration for the well.

\item \textcolor{blue}{\texttt{aux}---represents the values of the auxiliary variables for each well. The values of auxiliary variables must be present for each well. The values must be specified in the order of the auxiliary variables specified in the OPTIONS block.  If the package supports time series and the Options block includes a TIMESERIESFILE entry (see the ``Time-Variable Input'' section), values can be obtained from a time series by entering the time-series name in place of a numeric value.}

\item \texttt{boundname}---name of the well cell.  BOUNDNAME is an ASCII character variable that can contain as many as 40 characters.  If BOUNDNAME contains spaces in it, then the entire name must be enclosed within single quotes.

\end{description}
\item \textbf{Block: PERIOD}

\begin{description}
\item \texttt{iper}---integer value specifying the starting stress period number for which the data specified in the PERIOD block apply.  IPER must be less than or equal to NPER in the TDIS Package and greater than zero.  The IPER value assigned to a stress period block must be greater than the IPER value assigned for the previous PERIOD block.  The information specified in the PERIOD block will continue to apply for all subsequent stress periods, unless the program encounters another PERIOD block.

\item \texttt{mawno}---integer value that defines the well number associated with the specified PERIOD data on the line. MAWNO must be greater than zero and less than or equal to NMAWWELLS.

\item \texttt{mwtsetting}---line of information that is parsed into a keyword and values.  Keyword values that can be used to start the MWTSETTING string include: STATUS, CONCENTRATION, RAINFALL, EVAPORATION, RUNOFF, and AUXILIARY.  These settings are used to assign the concentration of associated with the corresponding flow terms.  Concentrations cannot be specified for all flow terms.  For example, the Multi-Aquifer Well Package supports a ``WITHDRAWAL'' flow term.  If this withdrawal term is active, then water will be withdrawn from the well at the calculated concentration of the well.

\begin{lstlisting}[style=blockdefinition]
STATUS <status>
CONCENTRATION <@concentration@>
RATE <@rate@>
AUXILIARY <auxname> <@auxval@> 
\end{lstlisting}

\item \texttt{status}---keyword option to define well status.  STATUS can be ACTIVE, INACTIVE, or CONSTANT. By default, STATUS is ACTIVE, which means that concentration will be calculated for the well.  If a well is inactive, then there will be no solute mass fluxes into or out of the well and the inactive value will be written for the well concentration.  If a well is constant, then the concentration for the well will be fixed at the user specified value.

\item \textcolor{blue}{\texttt{concentration}---real or character value that defines the concentration for the well. The specified CONCENTRATION is only applied if the well is a constant concentration well. If the Options block includes a TIMESERIESFILE entry (see the ``Time-Variable Input'' section), values can be obtained from a time series by entering the time-series name in place of a numeric value.}

\item \textcolor{blue}{\texttt{rate}---real or character value that defines the injection solute concentration $(ML^{-3})$ for the well. If the Options block includes a TIMESERIESFILE entry (see the ``Time-Variable Input'' section), values can be obtained from a time series by entering the time-series name in place of a numeric value.}

\item \texttt{AUXILIARY}---keyword for specifying auxiliary variable.

\item \texttt{auxname}---name for the auxiliary variable to be assigned AUXVAL.  AUXNAME must match one of the auxiliary variable names defined in the OPTIONS block. If AUXNAME does not match one of the auxiliary variable names defined in the OPTIONS block the data are ignored.

\item \textcolor{blue}{\texttt{auxval}---value for the auxiliary variable. If the Options block includes a TIMESERIESFILE entry (see the ``Time-Variable Input'' section), values can be obtained from a time series by entering the time-series name in place of a numeric value.}

\end{description}


\end{description}

\vspace{5mm}
\subsubsection{Example Input File}
\lstinputlisting[style=inputfile]{./mf6ivar/examples/gwt-mwt-example.dat}

\vspace{5mm}
\subsubsection{Available observation types}
Multi-Aquifer Well Transport Package observations include well concentration and all of the terms that contribute to the continuity equation for each well. Additional MWT Package observations include mass flow rates for individual wells, or groups of wells; the well volume (\texttt{volume}); and the conductance for a well-aquifer connection conductance (\texttt{conductance}). The data required for each MWT Package observation type is defined in table~\ref{table:gwt-mwtobstype}. Negative and positive values for \texttt{mwt} observations represent a loss from and gain to the GWT model, respectively. For all other flow terms, negative and positive values represent a loss from and gain from the MWT package, respectively.

\begin{longtable}{p{2cm} p{2.75cm} p{2cm} p{1.25cm} p{7cm}}
\caption{Available MWT Package observation types} \tabularnewline

\hline
\hline
\textbf{Stress Package} & \textbf{Observation type} & \textbf{ID} & \textbf{ID2} & \textbf{Description} \\
\hline
\endfirsthead

\captionsetup{textformat=simple}
\caption*{\textbf{Table \arabic{table}.}{\quad}Available MWT Package observation types.---Continued} \tabularnewline

\hline
\hline
\textbf{Stress Package} & \textbf{Observation type} & \textbf{ID} & \textbf{ID2} & \textbf{Description} \\
\hline
\endhead


\hline
\endfoot

% general APT observations
MWT & concentration & mawno or boundname & -- & Well concentration. If boundname is specified, boundname must be unique for each well. \\
%flowjaface not included
MWT & storage & mawno or boundname & -- & Simulated mass storage flow rate for a well or group of wells. \\
MWT & constant & mawno or boundname & -- & Simulated mass constant-flow rate for a well or group of wells. \\
MWT & from-mvr & mawno or boundname & -- & Simulated mass inflow into a well or group of wells from the MVT package. Mass inflow is calculated as the product of provider concentration and the mover flow rate. \\
MWT & mwt & mawno or boundname & \texttt{iconn} or -- & Mass flow rate for a well or group of wells and its aquifer connection(s). If boundname is not specified for ID, then the simulated well-aquifer flow rate at a specific well connection is observed. In this case, ID2 must be specified and is the connection number \texttt{iconn} for well \texttt{mawno}. \\

% observations specific to the mwt package
MWT & rate & mawno or boundname & -- & Simulated mass flow rate for a well or group of wells. \\
MWT & fw-rate & mawno or boundname & -- & Simulated mass flow rate for a flowing well or group of flowing wells. \\
MWT & rate-to-mvr & well or boundname & -- & Simulated mass flow rate that is sent to the MVT Package for a well or group of wells.\\
MWT & fw-rate-to-mvr & well or boundname & -- & Simulated mass flow rate that is sent to the MVT Package from a flowing well or group of flowing wells. \\

\label{table:gwt-mwtobstype}
\end{longtable}

\vspace{5mm}
\subsubsection{Example Observation Input File}
\lstinputlisting[style=inputfile]{./mf6ivar/examples/gwt-mwt-example-obs.dat}




\newpage
\subsection{Unsaturated Zone Transport (UZT) Package}
Unsaturated Zone Transport (UZT) Package information is read from the file that is specified by ``UZT6'' as the file type.  There can be as many UZT Packages as necessary for a GWT model. Each UZT Package is designed to work with flows from a corresponding GWF UZF Package. By default \mf uses the UZT package name to determine which UZF Package corresponds to the UZT Package.  Therefore, the package name of the UZT Package (as specified in the GWT name file) must match with the name of the corresponding UZF Package (as specified in the GWF name file).  Alternatively, the name of the flow package can be specified using the FLOW\_PACKAGE\_NAME keyword in the options block.  The GWT UZT Package cannot be used without a corresponding GWF UZF Package.

The UZT Package does not have a dimensions block; instead, dimensions for the UZT Package are set using the dimensions from the corresponding UZF Package.  For example, the UZF Package requires specification of the number of cells (NUZFCELLS).  UZT sets the number of UZT cells equal to NUZFCELLS.  Therefore, the PACKAGEDATA block below must have NUZFCELLS entries in it.

\vspace{5mm}
\subsubsection{Structure of Blocks}
\lstinputlisting[style=blockdefinition]{./mf6ivar/tex/gwt-uzt-options.dat}
\lstinputlisting[style=blockdefinition]{./mf6ivar/tex/gwt-uzt-packagedata.dat}
\lstinputlisting[style=blockdefinition]{./mf6ivar/tex/gwt-uzt-period.dat}

\vspace{5mm}
\subsubsection{Explanation of Variables}
\begin{description}
% DO NOT MODIFY THIS FILE DIRECTLY.  IT IS CREATED BY mf6ivar.py 

\item \textbf{Block: OPTIONS}

\begin{description}
\item \texttt{flow\_package\_name}---keyword to specify the name of the corresponding flow package.  If not specified, then the corresponding flow package must have the same name as this advanced transport package (the name associated with this package in the GWT name file).

\item \texttt{auxiliary}---defines an array of one or more auxiliary variable names.  There is no limit on the number of auxiliary variables that can be provided on this line; however, lists of information provided in subsequent blocks must have a column of data for each auxiliary variable name defined here.   The number of auxiliary variables detected on this line determines the value for naux.  Comments cannot be provided anywhere on this line as they will be interpreted as auxiliary variable names.  Auxiliary variables may not be used by the package, but they will be available for use by other parts of the program.  The program will terminate with an error if auxiliary variables are specified on more than one line in the options block.

\item \texttt{flow\_package\_auxiliary\_name}---keyword to specify the name of an auxiliary variable in the corresponding flow package.  If specified, then the simulated concentrations from this advanced transport package will be copied into the auxiliary variable specified with this name.  Note that the flow package must have an auxiliary variable with this name or the program will terminate with an error.  If the flows for this advanced transport package are read from a file, then this option will have no affect.

\item \texttt{BOUNDNAMES}---keyword to indicate that boundary names may be provided with the list of unsaturated zone flow cells.

\item \texttt{PRINT\_INPUT}---keyword to indicate that the list of unsaturated zone flow information will be written to the listing file immediately after it is read.

\item \texttt{PRINT\_CONCENTRATION}---keyword to indicate that the list of UZF cell concentration will be printed to the listing file for every stress period in which ``HEAD PRINT'' is specified in Output Control.  If there is no Output Control option and PRINT\_CONCENTRATION is specified, then concentration are printed for the last time step of each stress period.

\item \texttt{PRINT\_FLOWS}---keyword to indicate that the list of unsaturated zone flow rates will be printed to the listing file for every stress period time step in which ``BUDGET PRINT'' is specified in Output Control.  If there is no Output Control option and ``PRINT\_FLOWS'' is specified, then flow rates are printed for the last time step of each stress period.

\item \texttt{SAVE\_FLOWS}---keyword to indicate that unsaturated zone flow terms will be written to the file specified with ``BUDGET FILEOUT'' in Output Control.

\item \texttt{CONCENTRATION}---keyword to specify that record corresponds to concentration.

\item \texttt{concfile}---name of the binary output file to write concentration information.

\item \texttt{BUDGET}---keyword to specify that record corresponds to the budget.

\item \texttt{FILEOUT}---keyword to specify that an output filename is expected next.

\item \texttt{budgetfile}---name of the binary output file to write budget information.

\item \texttt{TS6}---keyword to specify that record corresponds to a time-series file.

\item \texttt{FILEIN}---keyword to specify that an input filename is expected next.

\item \texttt{ts6\_filename}---defines a time-series file defining time series that can be used to assign time-varying values. See the ``Time-Variable Input'' section for instructions on using the time-series capability.

\item \texttt{OBS6}---keyword to specify that record corresponds to an observations file.

\item \texttt{obs6\_filename}---name of input file to define observations for the UZT package. See the ``Observation utility'' section for instructions for preparing observation input files. Tables \ref{table:gwf-obstypetable} and \ref{table:gwt-obstypetable} lists observation type(s) supported by the UZT package.

\end{description}
\item \textbf{Block: PACKAGEDATA}

\begin{description}
\item \texttt{uzfno}---integer value that defines the UZF cell number associated with the specified PACKAGEDATA data on the line. UZFNO must be greater than zero and less than or equal to NUZFCELLS. Unsaturated zone flow information must be specified for every UZF cell or the program will terminate with an error.  The program will also terminate with an error if information for a UZF cell is specified more than once.

\item \texttt{strt}---real value that defines the starting concentration for the unsaturated zone flow cell.

\item \textcolor{blue}{\texttt{aux}---represents the values of the auxiliary variables for each unsaturated zone flow. The values of auxiliary variables must be present for each unsaturated zone flow. The values must be specified in the order of the auxiliary variables specified in the OPTIONS block.  If the package supports time series and the Options block includes a TIMESERIESFILE entry (see the ``Time-Variable Input'' section), values can be obtained from a time series by entering the time-series name in place of a numeric value.}

\item \texttt{boundname}---name of the unsaturated zone flow cell.  BOUNDNAME is an ASCII character variable that can contain as many as 40 characters.  If BOUNDNAME contains spaces in it, then the entire name must be enclosed within single quotes.

\end{description}
\item \textbf{Block: PERIOD}

\begin{description}
\item \texttt{iper}---integer value specifying the starting stress period number for which the data specified in the PERIOD block apply.  IPER must be less than or equal to NPER in the TDIS Package and greater than zero.  The IPER value assigned to a stress period block must be greater than the IPER value assigned for the previous PERIOD block.  The information specified in the PERIOD block will continue to apply for all subsequent stress periods, unless the program encounters another PERIOD block.

\item \texttt{uzfno}---integer value that defines the UZF cell number associated with the specified PERIOD data on the line. UZFNO must be greater than zero and less than or equal to NUZFCELLS.

\item \texttt{uztsetting}---line of information that is parsed into a keyword and values.  Keyword values that can be used to start the UZTSETTING string include: STATUS, CONCENTRATION, INFILTRATION, UZET, and AUXILIARY.  These settings are used to assign the concentration of associated with the corresponding flow terms.  Concentrations cannot be specified for all flow terms.

\begin{lstlisting}[style=blockdefinition]
STATUS <status>
CONCENTRATION <@concentration@>
INFILTRATION <@infiltration@>
UZET <@uzet@>
AUXILIARY <auxname> <@auxval@> 
\end{lstlisting}

\item \texttt{status}---keyword option to define UZF cell status.  STATUS can be ACTIVE, INACTIVE, or CONSTANT. By default, STATUS is ACTIVE, which means that concentration will be calculated for the UZF cell.  If a UZF cell is inactive, then there will be no solute mass fluxes into or out of the UZF cell and the inactive value will be written for the UZF cell concentration.  If a UZF cell is constant, then the concentration for the UZF cell will be fixed at the user specified value.

\item \textcolor{blue}{\texttt{concentration}---real or character value that defines the concentration for the unsaturated zone flow cell. The specified CONCENTRATION is only applied if the unsaturated zone flow cell is a constant concentration cell. If the Options block includes a TIMESERIESFILE entry (see the ``Time-Variable Input'' section), values can be obtained from a time series by entering the time-series name in place of a numeric value.}

\item \textcolor{blue}{\texttt{infiltration}---real or character value that defines the infiltration solute concentration $(ML^{-3})$ for the UZF cell. If the Options block includes a TIMESERIESFILE entry (see the ``Time-Variable Input'' section), values can be obtained from a time series by entering the time-series name in place of a numeric value.}

\item \textcolor{blue}{\texttt{uzet}---real or character value that defines the concentration of unsaturated zone evapotranspiration water $(ML^{-3})$ for the UZF cell. If this concentration value is larger than the simulated concentration in the UZF cell, then the unsaturated zone ET water will be removed at the same concentration as the UZF cell.  If the Options block includes a TIMESERIESFILE entry (see the ``Time-Variable Input'' section), values can be obtained from a time series by entering the time-series name in place of a numeric value.}

\item \texttt{AUXILIARY}---keyword for specifying auxiliary variable.

\item \texttt{auxname}---name for the auxiliary variable to be assigned AUXVAL.  AUXNAME must match one of the auxiliary variable names defined in the OPTIONS block. If AUXNAME does not match one of the auxiliary variable names defined in the OPTIONS block the data are ignored.

\item \textcolor{blue}{\texttt{auxval}---value for the auxiliary variable. If the Options block includes a TIMESERIESFILE entry (see the ``Time-Variable Input'' section), values can be obtained from a time series by entering the time-series name in place of a numeric value.}

\end{description}


\end{description}

\vspace{5mm}
\subsubsection{Example Input File}
\lstinputlisting[style=inputfile]{./mf6ivar/examples/gwt-uzt-example.dat}

\vspace{5mm}
\subsubsection{Available observation types}
Unsaturated Zone Transport Package observations include UZF cell concentration and all of the terms that contribute to the continuity equation for each UZF cell. Additional UZT Package observations include mass flow rates for individual UZF cells, or groups of UZF cells. The data required for each UZT Package observation type is defined in table~\ref{table:gwt-uztobstype}. Negative and positive values for \texttt{uzt} observations represent a loss from and gain to the GWT model, respectively. For all other flow terms, negative and positive values represent a loss from and gain from the UZT package, respectively.

\begin{longtable}{p{2cm} p{2.75cm} p{2cm} p{1.25cm} p{7cm}}
\caption{Available UZT Package observation types} \tabularnewline

\hline
\hline
\textbf{Stress Package} & \textbf{Observation type} & \textbf{ID} & \textbf{ID2} & \textbf{Description} \\
\hline
\endfirsthead

\captionsetup{textformat=simple}
\caption*{\textbf{Table \arabic{table}.}{\quad}Available UZT Package observation types.---Continued} \tabularnewline

\hline
\hline
\textbf{Stress Package} & \textbf{Observation type} & \textbf{ID} & \textbf{ID2} & \textbf{Description} \\
\hline
\endhead


\hline
\endfoot

% general APT observations
UZT & concentration & uztno or boundname & -- & uzt cell concentration. If boundname is specified, boundname must be unique for each uzt cell. \\
UZT & flow-ja-face & uztno or boundname & uztno or -- & Mass flow between two uzt cells.  If a boundname is specified for ID1, then the result is the total mass flow for all uzt cells. If a boundname is specified for ID1 then ID2 is not used.\\
UZT & storage & uztno or boundname & -- & Simulated mass storage flow rate for a uzt cell or group of uzt cells. \\
UZT & from-mvr & uztno or boundname & -- & Simulated mass inflow into a uzt cell or group of uzt cells from the MVT package. Mass inflow is calculated as the product of provider concentration and the mover flow rate. \\
UZT & uzt & uztno or boundname & \texttt{iconn} or -- & Mass flow rate for a uzt cell or group of uzt cells and its aquifer connection(s). If boundname is not specified for ID, then the simulated uzt-aquifer flow rate at a specific uzt cell connection is observed. In this case, ID2 must be specified and is the connection number for the uzt cell. \\

%observations specific to the uzt package
% infiltration rej-inf uzet rej-inf-to-mvr
UZT & infiltration & uztno or boundname & -- & Infiltration rate applied to a uzt cell or group of uzt cells multiplied by the infiltration concentration. \\
UZT & rej-inf & uztno or boundname & -- & Rejected infiltration rate applied to a uzt cell or group of uzt cells multiplied by the infiltration concentration. \\
UZT & uzet & uztno or boundname & -- & Unsaturated zone evapotranspiration rate applied to a uzt cell or group of uzt cells multiplied by the uzt cell concentration. \\
UZT & infiltration & uztno or boundname & -- & Rejected infiltration rate applied to a uzt cell or group of uzt cells multiplied by the infiltration concentration that is sent to the mover package. \\

\label{table:gwt-uztobstype}
\end{longtable}

\vspace{5mm}
\subsubsection{Example Observation Input File}
\lstinputlisting[style=inputfile]{./mf6ivar/examples/gwt-uzt-example-obs.dat}




\newpage
\subsection{Flow Model Interface (FMI) Package}
Flow Model Interface (FMI) Package information is read from the file that is specified by ``FMI6'' as the file type.  The FMI Package is optional, but if provided, only one FMI Package can be specified for a GWT model.

For most simulations, the GWT Model needs groundwater flows for every cell in the model grid, for all boundary conditions, and for other terms, such as the flow of water in or out of storage.  The FMI Package is the interface between the GWT Model and simulated groundwater flows provided by a corresponding GWF Model that is running concurrently within the simulation or from binary budget files that were created from a previous GWF model run.  The following are several different FMI simulation cases:

\begin{itemize}

\item Flows are provided by a corresponding GWF Model running in the same simulation---in this case, all groundwater flows are calculated by the corresponding GWF Model and provided through FMI to the transport model.  This is a common use case in which the user wants to run the flow and transport models as part of a single simulation.  The GWF and GWT models must be part of a GWF-GWT Exchange that is listed in mfsim.nam.  If a GWF-GWT Exchange is specified by the user, then the user does not need to specify an FMI Package input file for the simulation, unless an FMI option is needed.  If a GWF-GWT Exchange is specified and the FMI Package is specified, then the PACKAGEDATA block below is not read or used.

\item There is no groundwater flow and the user is interested only in the effects of diffusion, sorption, and decay or production---in this case, FMI should not be provided in the GWT name file and the GWT model should not be listed in any GWF-GWT Exchanges in mfsim.nam.  In this case, all groundwater flows are assumed to be zero and cells are assumed to be fully saturated.  The SSM Package should not be activated in this case, because there can be no sources or sinks of water.  Likewise, none of the advanced transport packages (LKT, SFT, MWT, and UZT) should be specified in the GWT name file.  This type of model simulation without an FMI Package is included as an option to represent diffusion, sorption, and decay or growth in the absence of any groundwater flow.

\item Flows are provided from a previous GWF model simulation---in this case FMI should be provided in the GWT name file and the head and budget files should be listed in the FMI options block.  In this case, FMI reads the simulated head and flows from these files and makes them available to the transport model.  There are some additional considerations when the heads and flows are provided from binary files.

\begin{itemize}
\item The binary budget file must contain the simulated flows for all of the packages that were included in the GWF model run.  Saving of flows can be activated for all packages by specifying ``SAVE\_FLOWS'' as an option in the GWF name file.  The GWF Output Control Package must also have ``SAVE BUGET ALL'' specified.  The easiest way to ensure that all flows and heads are saved is to use the following simple form of a GWF Output Control file:

\begin{verbatim}
BEGIN OPTIONS
  HEAD FILEOUT mymodel.hds
  BUDGET FILEOUT mymodel.bud
END OPTIONS

BEGIN PERIOD 1
  SAVE HEAD ALL
  SAVE BUDGET ALL
END PERIOD
\end{verbatim}

\item The binary budget file must have the same number of budget terms listed for each time step.  This will always be the case when the binary budget file is created by \mfdot
\item The advanced flow packages (LAK, SFR, MAW, and UZF) all have options for saving a detailed budget file the describes all of the flows for each lake, reach, well, or UZF cell.  These budget files can also be used as input to FMI if a corresponding advanced transport package is needed, such as LKT, SFT, MWT, and UZT.  If the Water Mover Package is also specified for the GWF Model, then the the budget file for the Water Mover Package will also need to be specified as input to this FMI Package.
\item The binary heads file must have heads saved for all layers in the model.  This will always be the case when the binary head file is created by \mfdot  This was not always the case as previous MODFLOW versions allowed different save options for each layer.
\item If the binary budget and head files have more than one time step for a single stress period, then the budget and head information must be contained within the binary file for every time step in the simulation stress period.
\item The binary budget and head files must correspond in terms of information stored for each time step and stress period.
\item If the binary budget and head files have information provided for only the first time step of each stress period, then this information will be used for all time steps in the GWT model run for that stress period.  This makes it possible to provide flows, for example, from a steady state GWF stress period and have those flows used for all steps in the GWT simulation.  With this option, it is possible to have smaller time steps in the GWT model than the time steps used in the GWF model.  Note that this cannot be done when the GWF and GWT models are run in the same simulation, because in that case, both models are solved for each time step in the stress period, as listed in the TDIS Package.  This option for reading flows from a previous GWF simulation may offer an efficient alternative to running both models in the same simulation, but it comes at the cost of having potentially very large budget files.
\end{itemize}

\end{itemize}

\noindent Determination of which FMI use case to invoke requires careful consideration of the different advantages and disadvantages of each case.  For example, running GWT and GWF in the same simulation can often be faster because GWF flows are passed through memory to the GWT model instead of being written to files.  The disadvantage of this approach is that the same time step lengths must be used for both GWF and GWT.  Ultimately, it should be relatively straightforward to test different ways in which GWF and GWT interact and select the use case most appropriate for the particular problem. 

\vspace{5mm}
\subsubsection{Structure of Blocks}
\lstinputlisting[style=blockdefinition]{./mf6ivar/tex/gwt-fmi-options.dat}
\lstinputlisting[style=blockdefinition]{./mf6ivar/tex/gwt-fmi-packagedata.dat}

\vspace{5mm}
\subsubsection{Explanation of Variables}
\begin{description}
% DO NOT MODIFY THIS FILE DIRECTLY.  IT IS CREATED BY mf6ivar.py 

\item \textbf{Block: OPTIONS}

\begin{description}
\item \texttt{FLOW\_IMBALANCE\_CORRECTION}---correct for an imbalance in flows by assuming that any residual flow error comes in or leaves at the concentration of the cell.  When this option is activated, the GWT Model budget written to the listing file will contain two additional entries: FLOW-ERROR and FLOW-CORRECTION.  These two entries will be equal but opposite in sign.  The FLOW-CORRECTION term is a mass flow that is added to offset the error caused by an imprecise flow balance.  If these terms are not relatively small, the flow model should be rerun with stricter convergence tolerances.

\end{description}
\item \textbf{Block: PACKAGEDATA}

\begin{description}
\item \texttt{flowtype}---is the word GWFBUDGET, GWFHEAD, GWFMOVER or the name of an advanced GWF stress package.  If GWFBUDGET is specified, then the corresponding file must be a budget file from a previous GWF Model run.  If an advanced GWF stress package name appears then the corresponding file must be the budget file saved by a LAK, SFR, MAW or UZF Package.

\item \texttt{FILEIN}---keyword to specify that an input filename is expected next.

\item \texttt{fname}---is the name of the file containing flows.  The path to the file should be included if the file is not located in the folder where the program was run.

\end{description}


\end{description}

\vspace{5mm}
\subsubsection{Example Input File}
\lstinputlisting[style=inputfile]{./mf6ivar/examples/gwt-fmi-example.dat}



\newpage
\subsection{Mover Transport (MVT) Package}
Mover Transport (MVT) Package information is read from the file that is specified by ``MVT6'' as the file type.  Only one MVT Package can be specified for a GWT model.  

The MVT Package is used to route solute mass according to flows from the GWF Water Mover (MVR) Package.  This MVT Package must be activated by the user if the MVR Package was active for the GWF Model.  Flows from the GWF MVR Package must be available to the GWT model either through activation of a GWF-GWT Exchange or through specification of ``GWFMOVER'' in the PACKAGEDATA block of the GWT FMI Package.  

\vspace{5mm}
\subsubsection{Structure of Blocks}
\lstinputlisting[style=blockdefinition]{./mf6ivar/tex/gwt-mvt-options.dat}

\vspace{5mm}
\subsubsection{Explanation of Variables}
\begin{description}
% DO NOT MODIFY THIS FILE DIRECTLY.  IT IS CREATED BY mf6ivar.py 

\item \textbf{Block: OPTIONS}

\begin{description}
\item \texttt{PRINT\_INPUT}---keyword to indicate that the list of mover information will be written to the listing file immediately after it is read.

\item \texttt{PRINT\_FLOWS}---keyword to indicate that the list of lake flow rates will be printed to the listing file for every stress period time step in which ``BUDGET PRINT'' is specified in Output Control.  If there is no Output Control option and ``PRINT\_FLOWS'' is specified, then flow rates are printed for the last time step of each stress period.

\item \texttt{SAVE\_FLOWS}---keyword to indicate that lake flow terms will be written to the file specified with ``BUDGET FILEOUT'' in Output Control.

\item \texttt{BUDGET}---keyword to specify that record corresponds to the budget.

\item \texttt{FILEOUT}---keyword to specify that an output filename is expected next.

\item \texttt{budgetfile}---name of the binary output file to write budget information.

\end{description}


\end{description}

\vspace{5mm}
\subsubsection{Example Input File}
\lstinputlisting[style=inputfile]{./mf6ivar/examples/gwt-mvt-example.dat}



