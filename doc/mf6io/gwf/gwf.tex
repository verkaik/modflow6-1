This section describes the data files for a \mf Groundwater Flow (GWF) Model.  A GWF Model is added to the simulation by including a GWF entry in the MODELS block of the simulation name file.

There are three types of spatial discretization approaches that can be used with the GWF Model.  Input for a GWF Model may be entered in a structured form, like for previous MODFLOW versions, in that users specify cells using their layer, row, and column indices.  Users may also work with a layered grid in which cells are defined using vertices.  In this case, users specify cells using the layer number and the cell number.  Lastly, GWF Models may be entered as fully unstructured models, in which cells are specified using only their cell number.  Once a spatial discretization approach has been selected, then all input with cell indices must be entered accordingly.

The GWF Model is designed to permit input to be gathered, as it is needed, from many different files.  Likewise, results from the model calculations can be written to a number of output files. The GWF Model Listing File is a key file to which the GWF model output is written.  As \mf runs, information about the GWF Model is written to the GWF Model Listing File, including much of the input data (as a record of the simulation) and calculated results.  Details about the files used by each package are provided in this section on the GWF Model Instructions.

\mf is further designed to allow the user to control the amount, type, and frequency of information to be output. Much of the output will be written to the Simulation and GWF Model Listing Files, but some model output can be written to other files.  The Listing Files can become very large for common models.  Text editors are useful for examining the Listing File. The GWF Model Listing File includes a summary of the input data read for all packages.  In addition, the GWF Model Listing File optionally contains calculated head controlled by time step, and the overall volumetric budget controlled by time step. The Listing Files also contain information about solver convergence and error messages.  Output to other files can include head and cell-by-cell flow terms for use in calculations external to the model or in user-supplied applications such as plotting programs.

The GWF Model reads a file called the Name File, which specifies most of the files that will be used in a simulation. Several files are always required whereas other files are optional depending on the simulation. The Output Control Package receives instructions from the user to control the amount and frequency of output.  Details about the Name File and the Output Control Package are described in this section.

\subsection{Information for Existing MODFLOW Users}
\mf contains most of the functionality of MODFLOW-2005, MODFLOW-NWT, MODFLOW-USG, and MODFLOW-LGR.  To the existing MODFLOW user, however, \mf will feel different from previous MODFLOW versions.  Some packages have been divided, renamed, or removed, and some capabilities, which previously caused confusion or were implemented due to computer memory limitations, are no longer supported (for example, ``quasi-3d confining units'' are not supported in the GWF Model).  The form of the input files for \mf is different from previous MODFLOW versions in that input files are now divided into blocks, and keywords are used to specify options and input variables.  Extensive testing was used as part of the development process to ensure that \mf simulation results are identical to the results from previous MODFLOW versions.  In some cases, it was not possible to exactly replicate the simulation results from previous MODFLOW versions.  In those cases, the differences could be explained by an option that is no longer supported, or because of slight differences in the underlying formulation.  

The following list has been updated from \cite{modflow6gwf}, and summarizes the major differences between the GWF Model in \mf and previous versions of MODFLOW.  This list is intended for those with a general understanding of the capabilities in previous versions of MODFLOW.

\begin{enumerate}

\item The GWF Model in \mf supports three alternative input packages for specifying the grid used to discretize the groundwater system.  
\begin{itemize}
\item The Discretization (DIS) Package defines a grid based on layers, rows, and columns.  In this report, this type of grid is referred to as a ``regular MODFLOW grid'' because it corresponds to traditional MODFLOW grids.  An interior cell in a regular MODFLOW grid is connected to four adjacent cells in the same layer, to one overlying cell, and to one underlying cell.
\item The Discretization by Vertices (DISV) Package defines a grid using a list of ($x$, $y$) vertex pairs and the number of layers.  A list of vertices is provided by the user to define a two-dimensional horizontal grid in plan view.  This list of vertices may define a regular MODFLOW grid, or they may define more complex grids, such as grids consisting of triangles, hexagons, or Voronoi polygons, for example.  This same two-dimensional horizontal grid applies to each layer in the model.  Cells defined using the DISV Package are referenced by layer number and by the cell number within the horizontal grid.  Within a layer, a cell may be horizontally connected to any number of surrounding cells in that layer.  In the vertical direction a cell can be connected to only one overlying cell and only one underlying cell.  Grids defined with the DISV Package are considered to be unstructured.
\item The unstructured Discretization (DISU) Package is the most flexible of the three packages and is patterned after the unstructured grid implemented in MODFLOW-USG.  For each cell, the user specifies a list of connected cells and the connection properties.  When the DISU Package is used, cells are referenced only by their cell number; unlike the MODFLOW-USG approach, there is no concept of a layer in the DISU Package in \mfcomma but cells may still overlie or underlie one another.  
\end{itemize}

\item For the three grid types supported in the GWF Model (DIS, DISV, and DISU), cells can be permanently excluded from the grid for the simulation.  Input values (such as hydraulic conductivity) are still required for these excluded cells, and the program will write special codes or zero values for output, but the program does not allocate memory or store values for excluded cells during run time.  In this case, the matrix equations are formulated for a reduced system in which only the included cells are numbered.  Users can also mark excluded cells as ``vertical pass-through cells,'' but this option is only available for DIS and DISV grids.  When these vertical pass-through cells are encountered, the program connects the cells overlying and underlying the pass-through cell.  This capability allows ``pinched'' cells to be removed from the solution.  These options to exclude cells or exclude them as pass-through cells are available through specification of the IDOMAIN array.

\item There is no longer a Basic Package input file.  Initial head values are specified using an Initial Conditions (IC) Package, and constant heads are specified using the Time Varying Specified Head (CHD) Package.  Cells that are permanently excluded from the simulation can be eliminated using the IDOMAIN capability entered through the DIS or DISV Packages.  For a cell that may transition from inactive (``dry'') to active (``wet'') during a simulation, the user can start the cell as inactive by assigning an initial head below the cell bottom.

\item The Newton-Raphson formulations and accompanying upstream weighting schemes implemented in MODFLOW-NWT and MODFLOW-USG for handling dry or nearly dry cells have been synthesized into a single formulation.  The Newton-Raphson formulation in the GWF Model for \mf remains an optional alternative to the standard formulation used in most previous MODFLOW versions. Much of the \cite{modflow6gwf} report is focused on systematically explaining standard and Newton-Raphson formulations for the GWF Model and its packages.

\item Information on temporal discretization, such as number of stress periods, period lengths, number of time steps, and time step multipliers, is specified at the simulation level, rather than for an individual model.  This information is provided in the Timing Module, which controls the temporal discretization and applies to all models within a simulation.  The Timing Module is part of the \mf framework and is described separately in \cite{modflow6framework}.

\item Aquifer properties used to calculate hydraulic conductance are specified in the Node Property Flow (NPF) Package.  In \mfcomma the NPF Package calculates intercell conductance values, manages cell wetting and drying, and adds Newton-Raphson terms for intercell flow expressions.  The NPF Package allows individual cells to be designated as confined or convertible; this was not an option in previous MODFLOW versions as the designation was by layer.  The NPF Package also has several options for simulating drainage problems and problems involving perched aquifers where an active cell overlies a partially saturated cell.  The default NPF Package behavior (in which none of these options are set) is the most stable for typical groundwater problems.  The default NPF Package behavior does not correspond to the default behavior for other MODFLOW internal flow packages.  The NPF Package does not support quasi-3D confining units.  The NPF Package replaces the Layer Property Flow (LPF), Block-Centered Flow (BCF), and Upstream Weighting (UPW) Packages from previous MODFLOW versions.  Capabilities of the Hydrogeologic Unit Flow (HUF) Package \citep{anderman2000modflow, anderman2003modflow} are not supported in the GWF Model of \mf.

\item Aquifer storage properties are specified in the Storage (STO) Package.  If the STO Package is excluded for a model, then the model represents steady-state conditions.  If the STO Package is included, users can specify steady-state or transient conditions by stress period as needed.  Compressible storage contributions are no longer approximated as zero for unconfined layers; contributions from pore drainage and compressible storage are separated in the model output.

\item The Horizontal Flow Barrier (HFB) Package \citep{hsieh1993hfb, modflow2005} in \mf allows barrier properties and locations to change by stress period.  The capability to change barriers by stress period was not supported in previous MODFLOW versions.

\item The GWF Model in \mf allows multiple stress packages of the same type to be specified for a single GWF Model.  This capability is also available in MODFLOW-CDSS \citep{banta2011modflow}.  Package entries written to the budget file and budget terms in the listing file are written separately for each package.

\item Input of boundary conditions for simulation in multiple stress periods is entered differently than for previous MODFLOW versions. Boundary conditions are specified for a stress period in a ``PERIOD'' block. These boundary conditions remain active at their specified values until a subsequent ``PERIOD'' block is encountered or the end of the simulation is reached.  Individual entries within the ``PERIOD'' block can be specified as a time-series entry.  Values for these variables, which may correspond to a well pumping rate or a drain conductance, for example, are interpolated from a time-series dataset, for each time step, using several different interpolation options.

\item The Flow and Head Boundary (FHB) Package \citep{leake1997documentation, modflow2005} is not supported in \mf; however, its capabilities can be replicated using the WEL Package, the CHD Package, and the new time-series capability.

\item There is one Evapotranspiration (EVT) Package for \mf. The \mf EVT Package contains the functionality of the MODFLOW-2005 EVT Package, the Segmented Evapotranspiration (ETS) Package \citep{modflowdrtpack}, and the Riparian Evapotranspiration (RIP-ET) Package \citep{modflowripetpack}.

\item A new Multi-Aquifer Well (MAW) Package replaces the Multi-Node Well (MNW1 and MNW2) Packages \citep{halford2002, konikow2009}. The new package does not contain all of the options available in MNW1 and MNW2, but it does contain the most commonly used ones.  It also has new capabilities for simulating flowing wells. The MAW Package is solved as part of the matrix solution and is tightly coupled with the GWF Model. This tight coupling with the GWF Model may substantially improve convergence for simulations of groundwater flow to multi-aquifer wells.

\item Most capabilities of the Stream (STR) and Streamflow Routing (SFR) Packages \citep{prudic1989str, modflowsfr1pack, modflowsfr2pack} are included in \mf as a new SFR Package.  The new SFR Package contains all of the functionality of the SFR Package in MODFLOW-2005 with the following exceptions: (a) the concept of a ``segment'' has been eliminated, (b) only rectangular cross sections are supported for stream reaches, and (c) unsaturated zone flow beneath stream reaches cannot be simulated.

\item A new Lake (LAK) Package replaces the existing MODFLOW Lake Packages \citep{modflowlak3pack}. In addition to being able to represent lakes that are incised into the model grid, the new LAK Package can also represent sub-grid scale lakes that are conceptualized as being on top of the model.  The status of a lake can change during the simulation between \texttt{ACTIVE}, \texttt{INACTIVE}, and \texttt{CONSTANT}.  The new package contains most of the capabilities available in previous LAK Packages, including the ability to apply recharge and evapotranspiration to underlying cells if the lake is dry.  The LAK Package documented here does not represent unsaturated zone flow beneath a lake or support for the coalescing lake option described in \cite{modflowlak3pack}. 

\item A new Unsaturated Zone Flow (UZF) Package, based on the one described by \cite{UZF}, is included in the GWF Model of \mfdot The new UZF Package allows the UZF capabilities to be applied to only selected cells of the GWF model. The new UZF Package also supports a multi-layer option, which allows for vertical heterogeneity in unsaturated zone properties.

\item A new Water Mover (MVR) Package is included in \mfdot  The MVR Package can be used to transfer water from individual ``provider'' features of selected packages (WEL, DRN, RIV, GHB, MAW, SFR, LAK, and UZF) to individual ''receiver'' features of the advanced packages (MAW, SFR, LAK, and UZF).  Simple rules are used to determine how much of the available water is moved from the provider to the receiver, which allows management controls to be represented. 

\item A new Skeletal Storage, Compaction, and Subsidence (CSUB) Package was added to \mf in version 6.1.0. The one-dimensional effective-stress based compaction theory implemented in the CSUB Package is documented in \cite{leake2007modflow}. The numerical approach used for delay interbeds in the CSUB package is documented in \cite{hoffmann2003modflow} and uses the same one-dimensional effective-stress based compaction theory as coarse-grained and fine-grained no-delay interbed sediments.

\item \mf contains a flexible new Observation (OBS) capability, which allows the user to define many different types of continuous-in-time observations.  The new OBS capability replaces the Observation Process \citep{hill2000modflow}, the Gage Package, and the HYDMOD capability \citep{hanson1999documentation} in previous MODFLOW versions.  Flow, head, and drawdown observations can be obtained for the GWF Model.  Flow and other package-specific observations, such as the head in a multi-aquifer well or lake stage, for example, can also be obtained.  These observed values can be used subsequently with a parameter estimation program or they can be used to make time-series plots of a wide range of simulated values.  The new OBS capability does not support specification of field-measured observations, calculation of residuals, or interpolation within a grid, as was supported in previous versions of the MODFLOW OBS Process.

\item The GWF Model described in this report does not support the following list of packages and capabilities.  Support for some of these capabilities may be added in future \mf versions.
  \begin{itemize}
    \item Drain with Return Flow Package \citep{modflowdrtpack},
    \item Reservoir Package \citep{fenske1996documentation},
    \item Seawater Intrusion Package \citep{bakker2013documentation},
    \item Surface-Water Routing Process \citep{hughes2012documentation},
    \item Connected Linear Network Process \citep{modflowusg},
    \item Parameter Value File \citep{modflow2005}, and
    \item Link to the MT3DMS Contaminant Transport Model \citep{zheng2001modflow}.  However, MT3D-USGS can read the head and budget files created by MODFLOW 6, but only if the GWF Model uses the DIS Package.  MT3D-USGS will not work with GWF output if the DISV or DISU Packages are used.
  \end{itemize}

\end{enumerate}

In addition to this list of major differences, there are other differences between \mf and previous MODFLOW versions in terms of the input and output files and the way users interact with the program.  These differences include:

\begin{enumerate}

\item The \mf program begins by reading a simulation name file.  The simulation name file must be named ``mfsim.nam.''

\item All real variables in \mf are declared as double precision floating point numbers.  Real variables written to binary output files are also written in double precision.

\item Unit numbers are no longer specified by the user.  Unit numbers are determined automatically by \mf based upon user-provided file names.

\item The GWF Model name file contains a list of packages that are active for the model.  Names for output files are not specified in the name file.  Names for output files, such as the head and budget files are specified in the OC Package.

\item The EXTERNAL option for reading arrays and lists is no longer supported; however, the OPEN/CLOSE option is still supported.  The SFAC option for lists is no longer supported; however, many packages allow for specification of an auxiliary variable which can serve as a multiplier on a column of values in the list.

\item The CHD Package contains new flexibility.  Cells can transition between constant-head cells and active cells during the simulation.  This was not allowed in previous MODFLOW versions.  Also, the CHD Packages no longer performs linear interpolation between a starting (shead) and ending head (ehead).  Only a single head value is provided for each constant-head cell.  The capability to linearly interpolate a head value for each time step within a stress period is available through the use of time series.

\item There are two different forms of input for the RCH and EVT Packages: array-based input and list-based input.  For models that use DIS Package, the RCH and EVT input can be provided as arrays, which is consistent with previous MODFLOW versions.  To use array input, the user must specify the READASARRAYS keyword in the options block.  The READASARRAYS option can also be used for models that use the DISV Package.  If the READASARRAYS option is not specified, then input to the RCH and EVT Packages is provided in list form.  List-based input is the only option supported when the DISU Package is used.

List-based input offers several advantages over the array-based input for specifying recharge and evapotranspiration.  First, multiple list entries can be specified for a single cell.  This makes it possible to divide a cell into multiple areas, and assign a different recharge or evapotranspiration rate for each area (perhaps based on land use or some other criteria).  In this case, the user would likely specify an auxiliary variable to serve as a multiplier.  This multiplier would be calculated by the user and provided in the input file as the fractional cell are for the individual recharge entries.  Another advantage to using list-based input for specifying recharge is that ``boundnames'' can be specified.  Boundnames work with the Observations capability and can be used to sum recharge or evapotranspiration rates for entries with the same boundname.  A disadvantage of the list-based input is that one cannot easily assign recharge or evapotranspiration rates to the entire model without specifying a list of model cells.  For this reason \mf also supports array-based input.

\item Calculation and reporting of drawdown for the model grid is no longer supported, as this calculation is easily performed as a postprocessing step.  Calculation of drawdown is supported as an observation type by the OBS Package; 
drawdown is calculated as the difference between the starting head specified in the IC Package and the calculated head.

\item There are differences in the output files created by \mfcomma such as:
\begin{itemize}

\item A separate listing file is written for the simulation.  This simulation listing file contains information about the simulation, including solver information.  Separate listing files are written for each GWF Model that is part of the simulation.

\item Unformatted head files written by \mf are consistent with those written by previous MODFLOW versions; however, all real values are written in double precision.

\item The budget file written by the GWF Model is always written in ``compact'' form (as opposed to full three-dimensional arrays) and uses new method codes, which allow model and package names to be written to the file.  Simulated intercell flows are always written in a compressed sparse row format, even for regular MODFLOW grids.

\item Information about the GWF Model grid is written to a separate file, called a ``binary grid file'' each time the model runs.  The binary grid file can be used by postprocessing programs for subsequent analysis.  The format of the binary grid file is described in a section on ``Binary Output Files.''

\end{itemize}


\end{enumerate}


\input{gwf/array_data.tex}

\subsection{Units of Length and Time}
The GWF Model formulates the groundwater flow equation without using prescribed length and time units. Any consistent units of length and time can be used when specifying the input data for a simulation. This capability gives a certain amount of freedom to the user, but care must be exercised to avoid mixing units.  The program cannot detect the use of inconsistent units.  For example, if hydraulic conductivity is entered in units of feet per day and pumpage as cubic meters per second, the program will run, but the results will be meaningless. Other processes generally are expected to work with consistent length and time units; however, other processes could conceivably place restrictions on which units are supported.

The user can set flags that specify the length and time units (see the input instructions for the Timing Module and Spatial Discretization Files), which may be useful in various parts of MODFLOW.  For example, the program will label the table of simulation time with time units if the time units are specified by the optional TIME\_UNITS label, which can be set in the TDIS Package.  If the time units are not specified, the program still runs, but the table of simulation time does not indicate the time units. An optional LENGTH\_UNITS label can be set in the Discretization Package. Situations in other processes may require that the length or time units be specified.  In such situations, the input instructions will state the requirements. Remember that specifying the unit flags does not enforce consistent use of units.  The user must insure that consistent units are used in all input data.

\subsection{Steady-State Simulations}
A steady-state simulation is represented by a single stress period having a single time step with the storage term set to zero. Setting the number and length of stress periods and time steps is the responsibility of the Timing Module of the \mf framework. The length of the stress period and time step will not affect the head solution because the time derivative is not calculated in a steady-state problem. Setting the storage term to zero is the responsibility of the Storage Package. Most other packages need not "know" that a simulation is steady state.

A GWF Model also can be mixed transient and steady state because each stress period can be designated transient or steady state.  Thus, a GWF Model can start with a steady-state stress period and continue with one or more transient stress periods.  The settings for controlling steady-state and transient options are in the Storage Package.  If the Storage Package is not specified for a GWF Model, then the storage terms are zero and the GWF Model will be steady state.

\subsection{Volumetric Budget}
A summary of all inflows (sources) and outflows (sinks) of water is called a water budget.  The water budget for the GWF Model is termed a volumetric budget because volumes of water and volumetric flow rates are involved; thus strictly speaking, a volumetric budget is not a mass balance, although this term has been used in other model reports.  \mf calculates a water budget for the overall model as a check on the acceptability of the solution, and to provide a summary of the sources and sinks of water to the flow system.  The water budget is printed to the GWF Model Listing File for selected time steps.

Numerical solution techniques for simultaneous equations do not always result in a correct answer; in particular, iterative solvers may stop iterating before a sufficiently close approximation to the solution is attained.  A water budget provides an indication of the overall acceptability of the solution.  The system of equations solved by the model actually consists of a flow continuity statement for each model cell.  Continuity should also exist for the total flows into and out of the model---that is, the difference between total inflow and total outflow should equal the total change in storage.  In the model program, the water budget is calculated independently of the equation solution process, and in this sense may provide independent evidence of a valid solution.

The total budget as printed in the output does not include internal flows between model cells---only flows into or out of the model as a whole. For example, flow to or from rivers, flow to or from constant-head cells, and flow to or from wells are all included in the overall budget terms.  Flow into and out of storage is also considered part of the overall budget inasmuch as accumulation in storage effectively removes water from the flow system and storage release effectively adds water to the flow---even though neither process, in itself, involves the transfer of water into or out of the ground-water regime.  Each hydrologic package calculates its own contribution to the budget.

For every time step, the budget subroutine of each hydrologic package calculates the rate of flow into and out of the system due to the process simulated by the package.  The inflows and outflows for each component of flow are stored separately.  Most packages deal with only one such component of flow.  In addition to flow, the volumes of water entering and leaving the model during the time step are calculated as the product of flow rate and time-step length.  Cumulative volumes, from the beginning of the simulation, are then calculated and stored.

The GWF Model uses the inflows, outflows, and cumulative volumes to write the budget to the Listing File at the times requested by the model user.  When a budget is written, the flow rates for the last time step and cumulative volumes from the beginning of simulation are written for each component of flow.  Inflows are written separately from outflows.  Following the convention indicated above, water entering storage is treated as an outflow (that is, as a loss of water from the flow system) while water released from storage is treated as an inflow (that is, a source of water to the flow system).  In addition, total inflow and total outflow are written, as well as the difference between total inflow and outflow.  The difference is then written as a percentage error, calculated using the formula:

\begin{equation}
D = \frac{100 (IN-OUT)}{(IN + OUT) / 2}
\end{equation}

\noindent where $D$ is the percentage error term, $IN$ is the total inflow to the system, and $OUT$ is the total outflow.

If the model equations are solved correctly, the percentage error should be small.  In general, flow rates may be taken as an indication of solution validity for the time step to which they apply, while cumulative volumes are an indication of validity for the entire simulation up to the time of the output.  The budget is written to the GWF Model Listing File at the end of each stress period whether requested or not.

\subsection{Cell-By-Cell Flows}
In some situations, calculating flow terms for various subregions of the model is useful.  To facilitate such calculations, provision has been made to save flow terms for individual cells in a separate binary file so they can be used in computations external to the model itself.  These individual cell flows are referred to here as ``cell-by-cell'' flow terms and are of four general types: (1) cell-by-cell stress flows, or flows into or from an individual cell caused by one of the external stresses represented in the model, such as evapotranspiration or recharge; (2) cell-by-cell storage terms, which give the rate of accumulation or depletion of storage in an individual cell; and (3) internal cell-by-cell flows, which are actually the flows across individual cell faces---that is, between adjacent model cells.  These four kinds of cell-by-cell flow terms are discussed further in subsequent paragraphs.  To save any of these cell-by-cell terms, two flags in the model input must be set.  The input to the Output Control file indicates the time steps for which cell-by-cell terms are to be saved. In addition, each hydrologic package includes an option called SAVE\_FLOWS that must be set if the cell-by-cell terms computed by that package are to be saved.  Thus, if the appropriate option in the Evapotranspiration Package input is set, cell-by-cell evapotranspiration terms will be saved for each time step for which the saving of cell-by-cell flow is requested through the Output Control Option.  Only flow values are saved in the cell-by-cell files; neither water volumes nor cumulative water volumes are included.  The flow dimensions are volume per unit time, where volume and time are in the same units used for all model input data.  The cell-by-cell flow values are stored in unformatted form to make the most efficient use of disk space; see the Budget File section toward the end of this user guide for information on how the data are written to a file.

The cell-by-cell storage term gives the net flow to or from storage in a variable-head cell.  The net storage for each cell in the grid is saved in transient simulations if the appropriate flags are set.  Withdrawal from storage in the cell is considered positive, whereas accumulation in storage is considered negative.

The cell-by-cell constant-head flow term gives the flow into or out of an individual constant-head cell (specified with the CHD Package).  This term is always associated with the constant-head cell itself, rather than with the surrounding cells that contribute or receive the flow.  A constant-head cell may be surrounded by as many as six adjacent variable-head cells for a regular grid or any number of cells for the other grid types.  The cell-by-cell calculation provides a single flow value for each constant-head cell, representing the algebraic sum of the flows between that cell and all of the adjacent variable-head cells.  A positive value indicates that the net flow is away from the constant-head cell (into the variable-head part of the grid); a negative value indicates that the net flow is into the constant-head cell.

The internal cell-by-cell flow values represent flows across the individual faces of a model cell.  Flows between cells are written in the compressed row storage format, whereby the flow between cell $n$ and each one of its connecting $m$ neighbor cells are contained in a single one-dimensional array.  Flows are positive for the cell in question.  Thus the flow reported for cell $n$ and its connection with cell $m$ is opposite in sign to the flow reported for cell $m$ and its connection with cell $n$.  These internal cell-by-cell flow values are useful in calculations of the groundwater flow into various subregions of the model, or in constructing flow vectors.

Cell-by-cell stress flows are flow rates into or out of the model, at a particular cell, owing to one particular external stress.  For example, the cell-by-cell evapotranspiration term for cell $n$ would give the flow out of the model by evapotranspiration from cell $n$.  Cell-by-cell stress flows are considered positive if flow is into the cell, and negative if out of the cell.

\newpage
\subsection{GWF Model Name File}
The GWF Model Name File specifies the options and packages that are active for a GWF model.  The Name File contains two blocks: OPTIONS  and PACKAGES. The length of each line must be 299 characters or less. The lines in each block can be in any order.  Files listed in the PACKAGES block must exist when the program starts. 

Comment lines are indicated when the first character in a line is one of the valid comment characters.  Commented lines can be located anywhere in the file. Any text characters can follow the comment character. Comment lines have no effect on the simulation; their purpose is to allow users to provide documentation about a particular simulation. 

\vspace{5mm}
\subsubsection{Structure of Blocks}
\lstinputlisting[style=blockdefinition]{./mf6ivar/tex/gwf-nam-options.dat}
\lstinputlisting[style=blockdefinition]{./mf6ivar/tex/gwf-nam-packages.dat}

\vspace{5mm}
\subsubsection{Explanation of Variables}
\begin{description}
\input{./mf6ivar/tex/gwf-nam-desc.tex}
\end{description}

\begin{table}[H]
\caption{Ftype values described in this report.  The \texttt{Pname} column indicates whether or not a package name can be provided in the name file}
\small
\begin{center}
\begin{tabular*}{\columnwidth}{l l l}
\hline
\hline
Ftype & Input File Description & \texttt{Pname}\\
\hline
DIS6 & Rectilinear Discretization Input File \\
DISV6 & Discretization by Vertices Input File \\
DISU6 & Unstructured Discretization Input File \\
IC6 & Initial Conditions Package \\
OC6 & Output Control Option \\
NPF6 & Node Property Flow Package \\ 
STO6 & Storage Package \\
CSUB6 & Compaction and Subsidence Package \\
HFB6 & Horizontal Flow Barrier Package\\
CHD6 & Time-Variant Specified Head Option & * \\
WEL6 & Well Package & * \\
DRN6 & Drain Package & * \\
RIV6 & River Package & * \\
GHB6 & General-Head Boundary Package & * \\
RCH6 & Recharge Package & * \\
EVT6 & Evapotranspiration Package & * \\
MAW6 & Multi-Aquifer Well Package & * \\
SFR6 & Streamflow Routing Package & * \\
LAK6 & Lake Package & * \\
UZF6 & Unsaturated Zone Flow Package & * \\
MVR6 & Water Mover Package \\
GNC6 & Ghost-Node Correction Package \\
OBS6 & Observations Option \\
\hline 
\end{tabular*}
\label{table:ftype}
\end{center}
\normalsize
\end{table}

\vspace{5mm}
\subsubsection{Example Input File}
\lstinputlisting[style=inputfile]{./mf6ivar/examples/gwf-nam-example.dat}



\newpage
\subsection{Structured Discretization (DIS) Input File}
\input{gwf/dis}

\newpage
\subsection{Discretization by Vertices (DISV) Input File}
\input{gwf/disv}

\newpage
\subsection{Unstructured Discretization (DISU) Input File}
\input{gwf/disu}

\newpage
\subsection{Initial Conditions (IC) Package}
Initial Conditions (IC) Package information is read from the file that is specified by ``IC6'' as the file type.  Only one IC Package can be specified for a GWT model. 

\vspace{5mm}
\subsubsection{Structure of Blocks}
%\lstinputlisting[style=blockdefinition]{./mf6ivar/tex/gwf-ic-options.dat}
\lstinputlisting[style=blockdefinition]{./mf6ivar/tex/gwt-ic-griddata.dat}

\vspace{5mm}
\subsubsection{Explanation of Variables}
\begin{description}
% DO NOT MODIFY THIS FILE DIRECTLY.  IT IS CREATED BY mf6ivar.py 

\item \textbf{Block: GRIDDATA}

\begin{description}
\item \texttt{strt}---is the initial (starting) concentration---that is, concentration at the beginning of the GWT Model simulation.  STRT must be specified for all GWT Model simulations. One value is read for every model cell.

\end{description}


\end{description}

\vspace{5mm}
\subsubsection{Example Input File}
\lstinputlisting[style=inputfile]{./mf6ivar/examples/gwt-ic-example.dat}



\newpage
\subsection{Output Control (OC) Option}
Input to the Output Control Option of the Groundwater Transport Model is read from the file that is specified as type ``OC6'' in the Name File. If no ``OC6'' file is specified, default output control is used. The Output Control Option determines how and when concentrations are printed to the listing file and/or written to a separate binary output file.  Under the default, concentration and overall transport budget are written to the Listing File at the end of every stress period. The default printout format for concentrations is 10G11.4.  The concentrations and overall transport budget are also written to the list file if the simulation terminates prematurely due to failed convergence.

Output Control data must be specified using words.  The numeric codes supported in earlier MODFLOW versions can no longer be used.

For the PRINT and SAVE options of concentration, there is no option to specify individual layers.  Whenever the concentration array is printed or saved, all layers are printed or saved.

\vspace{5mm}
\subsubsection{Structure of Blocks}
\vspace{5mm}

\noindent \textit{FOR EACH SIMULATION}
\lstinputlisting[style=blockdefinition]{./mf6ivar/tex/gwt-oc-options.dat}
\vspace{5mm}
\noindent \textit{FOR ANY STRESS PERIOD}
\lstinputlisting[style=blockdefinition]{./mf6ivar/tex/gwt-oc-period.dat}

\vspace{5mm}
\subsubsection{Explanation of Variables}
\begin{description}
% DO NOT MODIFY THIS FILE DIRECTLY.  IT IS CREATED BY mf6ivar.py 

\item \textbf{Block: OPTIONS}

\begin{description}
\item \texttt{BUDGET}---keyword to specify that record corresponds to the budget.

\item \texttt{FILEOUT}---keyword to specify that an output filename is expected next.

\item \texttt{budgetfile}---name of the output file to write budget information.

\item \texttt{CONCENTRATION}---keyword to specify that record corresponds to concentration.

\item \texttt{concentrationfile}---name of the output file to write conc information.

\item \texttt{PRINT\_FORMAT}---keyword to specify format for printing to the listing file.

\item \texttt{columns}---number of columns for writing data.

\item \texttt{width}---width for writing each number.

\item \texttt{digits}---number of digits to use for writing a number.

\item \texttt{format}---write format can be EXPONENTIAL, FIXED, GENERAL, or SCIENTIFIC.

\end{description}
\item \textbf{Block: PERIOD}

\begin{description}
\item \texttt{iper}---integer value specifying the starting stress period number for which the data specified in the PERIOD block apply.  IPER must be less than or equal to NPER in the TDIS Package and greater than zero.  The IPER value assigned to a stress period block must be greater than the IPER value assigned for the previous PERIOD block.  The information specified in the PERIOD block will continue to apply for all subsequent stress periods, unless the program encounters another PERIOD block.

\item \texttt{SAVE}---keyword to indicate that information will be saved this stress period.

\item \texttt{PRINT}---keyword to indicate that information will be printed this stress period.

\item \texttt{rtype}---type of information to save or print.  Can be BUDGET or CONCENTRATION.

\item \texttt{ocsetting}---specifies the steps for which the data will be saved.

\begin{lstlisting}[style=blockdefinition]
ALL
FIRST
LAST
FREQUENCY <frequency>
STEPS <steps(<nstp)>
\end{lstlisting}

\item \texttt{ALL}---keyword to indicate save for all time steps in period.

\item \texttt{FIRST}---keyword to indicate save for first step in period. This keyword may be used in conjunction with other keywords to print or save results for multiple time steps.

\item \texttt{LAST}---keyword to indicate save for last step in period. This keyword may be used in conjunction with other keywords to print or save results for multiple time steps.

\item \texttt{frequency}---save at the specified time step frequency. This keyword may be used in conjunction with other keywords to print or save results for multiple time steps.

\item \texttt{steps}---save for each step specified in STEPS. This keyword may be used in conjunction with other keywords to print or save results for multiple time steps.

\end{description}


\end{description}

\vspace{5mm}
\subsubsection{Example Input File}
\lstinputlisting[style=inputfile]{./mf6ivar/examples/gwt-oc-example.dat}


\newpage
\subsection{Observation (OBS) Utility for a GWF Model}

GWF Model observations include the simulated groundwater head (\texttt{head}), calculated drawdown (\texttt{drawdown}) at a node, and the flow between two connected nodes (\texttt{flow-ja-face}). The data required for each GWF Model observation type is defined in table~\ref{table:gwfobstype}. For \texttt{flow-ja-face} observation types, negative and positive values represent a loss from and gain to the \texttt{cellid} specified for ID, respectively.

\subsubsection{Structure of Blocks}
\vspace{5mm}

\noindent \textit{FOR EACH SIMULATION}
\lstinputlisting[style=blockdefinition]{./mf6ivar/tex/utl-obs-options.dat}
\lstinputlisting[style=blockdefinition]{./mf6ivar/tex/utl-obs-continuous.dat}

\subsubsection{Explanation of Variables}
\begin{description}
% DO NOT MODIFY THIS FILE DIRECTLY.  IT IS CREATED BY mf6ivar.py 

\item \textbf{Block: OPTIONS}

\begin{description}
\item \texttt{digits}---Keyword and an integer digits specifier used for conversion of simulated values to text on output. The default is 5 digits. When simulated values are written to a file specified as file type DATA in the Name File, the digits specifier controls the number of significant digits with which simulated values are written to the output file. The digits specifier has no effect on the number of significant digits with which the simulation time is written for continuous observations.

\item \texttt{PRINT\_INPUT}---keyword to indicate that the list of observation information will be written to the listing file immediately after it is read.

\end{description}
\item \textbf{Block: CONTINUOUS}

\begin{description}
\item \texttt{FILEOUT}---keyword to specify that an output filename is expected next.

\item \texttt{obs\_output\_file\_name}---Name of a file to which simulated values corresponding to observations in the block are to be written. The file name can be an absolute or relative path name. A unique output file must be specified for each CONTINUOUS block. If the ``BINARY'' option is used, output is written in binary form. By convention, text output files have the extension ``csv'' (for ``Comma-Separated Values'') and binary output files have the extension ``bsv'' (for ``Binary Simulated Values'').

\item \texttt{BINARY}---an optional keyword used to indicate that the output file should be written in binary (unformatted) form.

\item \texttt{obsname}---string of 1 to 40 nonblank characters used to identify the observation. The identifier need not be unique; however, identification and post-processing of observations in the output files are facilitated if each observation is given a unique name.

\item \texttt{obstype}---a string of characters used to identify the observation type.

\item \texttt{id}---Text identifying cell where observation is located. For packages other than NPF, if boundary names are defined in the corresponding package input file, ID can be a boundary name. Otherwise ID is a cellid. If the model discretization is type DIS, cellid is three integers (layer, row, column). If the discretization is DISV, cellid is two integers (layer, cell number). If the discretization is DISU, cellid is one integer (node number).

\item \texttt{id2}---Text identifying cell adjacent to cell identified by ID. The form of ID2 is as described for ID. ID2 is used for intercell-flow observations of a GWF model, for three observation types of the LAK Package, for two observation types of the MAW Package, and one observation type of the UZF Package.

\end{description}


\end{description}


\begin{longtable}{p{2cm} p{2.75cm} p{2cm} p{1.25cm} p{7cm}}
\caption{Available GWF model observation types} \tabularnewline

\hline
\hline
\textbf{Model} & \textbf{Observation type} & \textbf{ID} & \textbf{ID2} & \textbf{Description} \\
\hline
\endhead

\hline
\endfoot


GWF Model observations include the simulated groundwater head (\texttt{head}), calculated drawdown (\texttt{drawdown}) at a node, and the flow between two connected nodes (\texttt{flow-ja-face}). The data required for each GWF Model observation type is defined in table~\ref{table:gwfobstype}. For \texttt{flow-ja-face} observation types, negative and positive values represent a loss from and gain to the \texttt{cellid} specified for ID, respectively.

\subsubsection{Structure of Blocks}
\vspace{5mm}

\noindent \textit{FOR EACH SIMULATION}
\lstinputlisting[style=blockdefinition]{./mf6ivar/tex/utl-obs-options.dat}
\lstinputlisting[style=blockdefinition]{./mf6ivar/tex/utl-obs-continuous.dat}

\subsubsection{Explanation of Variables}
\begin{description}
% DO NOT MODIFY THIS FILE DIRECTLY.  IT IS CREATED BY mf6ivar.py 

\item \textbf{Block: OPTIONS}

\begin{description}
\item \texttt{digits}---Keyword and an integer digits specifier used for conversion of simulated values to text on output. The default is 5 digits. When simulated values are written to a file specified as file type DATA in the Name File, the digits specifier controls the number of significant digits with which simulated values are written to the output file. The digits specifier has no effect on the number of significant digits with which the simulation time is written for continuous observations.

\item \texttt{PRINT\_INPUT}---keyword to indicate that the list of observation information will be written to the listing file immediately after it is read.

\end{description}
\item \textbf{Block: CONTINUOUS}

\begin{description}
\item \texttt{FILEOUT}---keyword to specify that an output filename is expected next.

\item \texttt{obs\_output\_file\_name}---Name of a file to which simulated values corresponding to observations in the block are to be written. The file name can be an absolute or relative path name. A unique output file must be specified for each CONTINUOUS block. If the ``BINARY'' option is used, output is written in binary form. By convention, text output files have the extension ``csv'' (for ``Comma-Separated Values'') and binary output files have the extension ``bsv'' (for ``Binary Simulated Values'').

\item \texttt{BINARY}---an optional keyword used to indicate that the output file should be written in binary (unformatted) form.

\item \texttt{obsname}---string of 1 to 40 nonblank characters used to identify the observation. The identifier need not be unique; however, identification and post-processing of observations in the output files are facilitated if each observation is given a unique name.

\item \texttt{obstype}---a string of characters used to identify the observation type.

\item \texttt{id}---Text identifying cell where observation is located. For packages other than NPF, if boundary names are defined in the corresponding package input file, ID can be a boundary name. Otherwise ID is a cellid. If the model discretization is type DIS, cellid is three integers (layer, row, column). If the discretization is DISV, cellid is two integers (layer, cell number). If the discretization is DISU, cellid is one integer (node number).

\item \texttt{id2}---Text identifying cell adjacent to cell identified by ID. The form of ID2 is as described for ID. ID2 is used for intercell-flow observations of a GWF model, for three observation types of the LAK Package, for two observation types of the MAW Package, and one observation type of the UZF Package.

\end{description}


\end{description}


\begin{longtable}{p{2cm} p{2.75cm} p{2cm} p{1.25cm} p{7cm}}
\caption{Available GWF model observation types} \tabularnewline

\hline
\hline
\textbf{Model} & \textbf{Observation type} & \textbf{ID} & \textbf{ID2} & \textbf{Description} \\
\hline
\endhead

\hline
\endfoot


GWF Model observations include the simulated groundwater head (\texttt{head}), calculated drawdown (\texttt{drawdown}) at a node, and the flow between two connected nodes (\texttt{flow-ja-face}). The data required for each GWF Model observation type is defined in table~\ref{table:gwfobstype}. For \texttt{flow-ja-face} observation types, negative and positive values represent a loss from and gain to the \texttt{cellid} specified for ID, respectively.

\subsubsection{Structure of Blocks}
\vspace{5mm}

\noindent \textit{FOR EACH SIMULATION}
\lstinputlisting[style=blockdefinition]{./mf6ivar/tex/utl-obs-options.dat}
\lstinputlisting[style=blockdefinition]{./mf6ivar/tex/utl-obs-continuous.dat}

\subsubsection{Explanation of Variables}
\begin{description}
\input{./mf6ivar/tex/utl-obs-desc.tex}
\end{description}


\begin{longtable}{p{2cm} p{2.75cm} p{2cm} p{1.25cm} p{7cm}}
\caption{Available GWF model observation types} \tabularnewline

\hline
\hline
\textbf{Model} & \textbf{Observation type} & \textbf{ID} & \textbf{ID2} & \textbf{Description} \\
\hline
\endhead

\hline
\endfoot

\input{../Common/gwf-obs.tex}
\label{table:gwfobstype}
\end{longtable}

\vspace{5mm}
\subsubsection{Example Observation Input File}

An example GWF Model observation file is shown below.

\lstinputlisting[style=inputfile]{./mf6ivar/examples/utl-obs-example-obs.dat}


\label{table:gwfobstype}
\end{longtable}

\vspace{5mm}
\subsubsection{Example Observation Input File}

An example GWF Model observation file is shown below.

\lstinputlisting[style=inputfile]{./mf6ivar/examples/utl-obs-example-obs.dat}


\label{table:gwfobstype}
\end{longtable}

\vspace{5mm}
\subsubsection{Example Observation Input File}

An example GWF Model observation file is shown below.

\lstinputlisting[style=inputfile]{./mf6ivar/examples/utl-obs-example-obs.dat}



\newpage
\subsection{Node Property Flow (NPF) Package}
\input{gwf/npf}

\newpage
\subsection{Horizontal Flow Barrier (HFB) Package}
\input{gwf/hfb}

\newpage
\subsection{Storage (STO) Package}
\input{gwf/sto}

\newpage
\subsection{Skeletal Storage, Compaction, and Subsidence (CSUB) Package}
Input to the Skeletal Storage, Compaction, and Subsidence (CSUB) Package is read from the file that has type ``CSUB6'' in the Name File.  If the CSUB Package is not included for a model, then storage changes resulting from compaction will not be calculated.  Only one CSUB Package can be specified for a GWF model. Only the first and last stress period can be specified to be STEADY-STATE in the STO Package when the CSUB Package is being used in the GWF model. Also the specific storage (SS) must be specified to be zero in the STO Package for every cell.

\vspace{5mm}
\subsubsection{Structure of Blocks}

\vspace{5mm}
\noindent \textit{FOR EACH SIMULATION}
\lstinputlisting[style=blockdefinition]{./mf6ivar/tex/gwf-csub-options.dat}
\lstinputlisting[style=blockdefinition]{./mf6ivar/tex/gwf-csub-dimensions.dat}
\lstinputlisting[style=blockdefinition]{./mf6ivar/tex/gwf-csub-griddata.dat}
\lstinputlisting[style=blockdefinition]{./mf6ivar/tex/gwf-csub-packagedata.dat}
\vspace{5mm}
\noindent \textit{FOR ANY STRESS PERIOD}
\lstinputlisting[style=blockdefinition]{./mf6ivar/tex/gwf-csub-period.dat}
\packageperioddescription

\vspace{5mm}
\subsubsection{Explanation of Variables}
\begin{description}
% DO NOT MODIFY THIS FILE DIRECTLY.  IT IS CREATED BY mf6ivar.py 

\item \textbf{Block: OPTIONS}

\begin{description}
\item \texttt{BOUNDNAMES}---keyword to indicate that boundary names may be provided with the list of CSUB cells.

\item \texttt{PRINT\_INPUT}---keyword to indicate that the list of CSUB information will be written to the listing file immediately after it is read.

\item \texttt{SAVE\_FLOWS}---keyword to indicate that cell-by-cell flow terms will be written to the file specified with ``BUDGET SAVE FILE'' in Output Control.

\item \texttt{gammaw}---unit weight of water. For freshwater, GAMMAW is 9806.65 Newtons/cubic meters or 62.48 lb/cubic foot in SI and English units, respectively. By default, GAMMAW is 9806.65 Newtons/cubic meters.

\item \texttt{beta}---compressibility of water. Typical values of BETA are 4.6512e-10 1/Pa or 2.2270e-8 lb/square foot in SI and English units, respectively. By default, BETA is 4.6512e-10 1/Pa.

\item \texttt{HEAD\_BASED}---keyword to indicate the head-based formulation will be used to simulate coarse-grained aquifer materials and no-delay and delay interbeds. Specifying HEAD\_BASED also specifies the INITIAL\_PRECONSOLIDATION\_HEAD option.

\item \texttt{INITIAL\_PRECONSOLIDATION\_HEAD}---keyword to indicate that preconsolidation heads will be specified for no-delay and delay interbeds in the PACKAGEDATA block. If the SPECIFIED\_INITIAL\_INTERBED\_STATE option is specified in the OPTIONS block, user-specified preconsolidation heads in the PACKAGEDATA block are absolute values. Otherwise, user-specified preconsolidation heads in the PACKAGEDATA block are relative to steady-state or initial heads.

\item \texttt{ndelaycells}---number of nodes used to discretize delay interbeds. If not specified, then a default value of 19 is assigned.

\item \texttt{COMPRESSION\_INDICES}---keyword to indicate that the recompression (CR) and compression (CC) indices are specified instead of the elastic specific storage (SSE) and inelastic specific storage (SSV) coefficients. If not specified, then elastic specific storage (SSE) and inelastic specific storage (SSV) coefficients must be specified.

\item \texttt{UPDATE\_MATERIAL\_PROPERTIES}---keyword to indicate that the thickness and void ratio of coarse-grained and interbed sediments (delay and no-delay) will vary during the simulation. If not specified, the thickness and void ratio of coarse-grained and interbed sediments will not vary during the simulation.

\item \texttt{CELL\_FRACTION}---keyword to indicate that the thickness of interbeds will be specified in terms of the fraction of cell thickness. If not specified, interbed thicknness must be specified.

\item \texttt{SPECIFIED\_INITIAL\_INTERBED\_STATE}---keyword to indicate that absolute preconsolidation stresses (heads) and delay bed heads will be specified for interbeds defined in the PACKAGEDATA block. The SPECIFIED\_INITIAL\_INTERBED\_STATE option is equivalent to specifying the SPECIFIED\_INITIAL\_PRECONSOLITATION\_STRESS and SPECIFIED\_INITIAL\_DELAY\_HEAD. If SPECIFIED\_INITIAL\_INTERBED\_STATE is not specified then preconsolidation stress (head) and delay bed head values specified in the PACKAGEDATA block are relative to simulated values of the first stress period if steady-state or initial stresses and GWF heads if the first stress period is transient.

\item \texttt{SPECIFIED\_INITIAL\_PRECONSOLIDATION\_STRESS}---keyword to indicate that absolute preconsolidation stresses (heads) will be specified for interbeds defined in the PACKAGEDATA block. If SPECIFIED\_INITIAL\_PRECONSOLITATION\_STRESS and SPECIFIED\_INITIAL\_INTERBED\_STATE are not specified then preconsolidation stress (head) values specified in the PACKAGEDATA block are relative to simulated values if the first stress period is steady-state or initial stresses (heads) if the first stress period is transient.

\item \texttt{SPECIFIED\_INITIAL\_DELAY\_HEAD}---keyword to indicate that absolute initial delay bed head will be specified for interbeds defined in the PACKAGEDATA block. If SPECIFIED\_INITIAL\_DELAY\_HEAD and SPECIFIED\_INITIAL\_INTERBED\_STATE are not specified then delay bed head values specified in the PACKAGEDATA block are relative to simulated values if the first stress period is steady-state or initial GWF heads if the first stress period is transient.

\item \texttt{EFFECTIVE\_STRESS\_LAG}---keyword to indicate the effective stress from the previous time step will be used to calculate specific storage values. This option can 1) help with convergence in models with thin cells and water table elevations close to land surface; 2) is identical to the approach used in the SUBWT package for MODFLOW-2005; and 3) is only used if the effective-stress formulation is being used. By default, current effective stress values are used to calculate specific storage values.

\item \texttt{STRAIN\_CSV\_INTERBED}---keyword to specify the record that corresponds to final interbed strain output.

\item \texttt{FILEOUT}---keyword to specify that an output filename is expected next.

\item \texttt{interbedstrain\_filename}---name of the comma-separated-values output file to write final interbed strain information.

\item \texttt{STRAIN\_CSV\_COARSE}---keyword to specify the record that corresponds to final coarse-grained material strain output.

\item \texttt{coarsestrain\_filename}---name of the comma-separated-values output file to write final coarse-grained material strain information.

\item \texttt{COMPACTION}---keyword to specify that record corresponds to the compaction.

\item \texttt{compaction\_filename}---name of the binary output file to write compaction information.

\item \texttt{COMPACTION\_ELASTIC}---keyword to specify that record corresponds to the elastic interbed compaction binary file.

\item \texttt{elastic\_compaction\_filename}---name of the binary output file to write elastic interbed compaction information.

\item \texttt{COMPACTION\_INELASTIC}---keyword to specify that record corresponds to the inelastic interbed compaction binary file.

\item \texttt{inelastic\_compaction\_filename}---name of the binary output file to write inelastic interbed compaction information.

\item \texttt{COMPACTION\_INTERBED}---keyword to specify that record corresponds to the interbed compaction binary file.

\item \texttt{interbed\_compaction\_filename}---name of the binary output file to write interbed compaction information.

\item \texttt{COMPACTION\_COARSE}---keyword to specify that record corresponds to the elastic coarse-grained material compaction binary file.

\item \texttt{coarse\_compaction\_filename}---name of the binary output file to write elastic coarse-grained material compaction information.

\item \texttt{ZDISPLACEMENT}---keyword to specify that record corresponds to the z-displacement binary file.

\item \texttt{zdisplacement\_filename}---name of the binary output file to write z-displacement information.

\item \texttt{TS6}---keyword to specify that record corresponds to a time-series file.

\item \texttt{FILEIN}---keyword to specify that an input filename is expected next.

\item \texttt{ts6\_filename}---defines a time-series file defining time series that can be used to assign time-varying values. See the ``Time-Variable Input'' section for instructions on using the time-series capability.

\item \texttt{OBS6}---keyword to specify that record corresponds to an observations file.

\item \texttt{obs6\_filename}---name of input file to define observations for the CSUB package. See the ``Observation utility'' section for instructions for preparing observation input files. Table \ref{table:obstype} lists observation type(s) supported by the CSUB package.

\end{description}
\item \textbf{Block: DIMENSIONS}

\begin{description}
\item \texttt{ninterbeds}---is the number of CSUB interbed systems.  More than 1 CSUB interbed systems can be assigned to a GWF cell; however, only 1 GWF cell can be assigned to a single CSUB interbed system.

\item \texttt{maxsig0}---is the maximum number of cells that can have a specified stress offset.  More than 1 stress offset can be assigned to a GWF cell. By default, MAXSIG0 is 0.

\end{description}
\item \textbf{Block: GRIDDATA}

\begin{description}
\item \texttt{cg\_ske\_cr}---is the initial elastic coarse-grained material specific storage or recompression index. The recompression index is specified if COMPRESSION\_INDICES is specified in the OPTIONS block.  Specified or calculated elastic coarse-grained material specific storage values are not adjusted from initial values if HEAD\_BASED is specified in the OPTIONS block.

\item \texttt{cg\_theta}---is the initial porosity of coarse-grained materials.

\item \texttt{sgm}---is the specific gravity of moist or unsaturated sediments.  If not specified, then a default value of 1.7 is assigned.

\item \texttt{sgs}---is the specific gravity of saturated sediments. If not specified, then a default value of 2.0 is assigned.

\end{description}
\item \textbf{Block: PACKAGEDATA}

\begin{description}
\item \texttt{icsubno}---integer value that defines the CSUB interbed number associated with the specified PACKAGEDATA data on the line. CSUBNO must be greater than zero and less than or equal to NCSUBCELLS.  CSUB information must be specified for every CSUB cell or the program will terminate with an error.  The program will also terminate with an error if information for a CSUB interbed number is specified more than once.

\item \texttt{cellid}---is the cell identifier, and depends on the type of grid that is used for the simulation.  For a structured grid that uses the DIS input file, CELLID is the layer, row, and column.   For a grid that uses the DISV input file, CELLID is the layer and CELL2D number.  If the model uses the unstructured discretization (DISU) input file, CELLID is the node number for the cell.

\item \texttt{cdelay}---character string that defines the subsidence delay type for the interbed. Possible subsidence package CDELAY strings include: NODELAY--character keyword to indicate that delay will not be simulated in the interbed.  DELAY--character keyword to indicate that delay will be simulated in the interbed.

\item \texttt{pcs0}---is the initial offset from the calculated initial effective stress or initial preconsolidation stress in the interbed, in units of height of a column of water. PCS0 is the initial preconsolidation stress if SPECIFIED\_INITIAL\_INTERBED\_STATE or SPECIFIED\_INITIAL\_PRECONSOLIDATION\_STRESS are specified in the OPTIONS block. If HEAD\_BASED is specified in the OPTIONS block, PCS0 is the initial offset from the calculated initial head or initial preconsolidation head in the CSUB interbed and the initial preconsolidation stress is calculated from the calculated initial effective stress or calculated initial geostatic stress, respectively.

\item \texttt{thick\_frac}---is the interbed thickness or cell fraction of the interbed. Interbed thickness is specified as a fraction of the cell thickness if CELL\_FRACTION is specified in the OPTIONS block.

\item \texttt{rnb}---is the interbed material factor equivalent number of interbeds in the interbed system represented by the interbed. RNB must be greater than or equal to 1 if CDELAY is DELAY. Otherwise, RNB can be any value.

\item \texttt{ssv\_cc}---is the initial inelastic specific storage or compression index of the interbed. The compression index is specified if COMPRESSION\_INDICES is specified in the OPTIONS block. Specified or calculated interbed inelastic specific storage values are not adjusted from initial values if HEAD\_BASED is specified in the OPTIONS block.

\item \texttt{sse\_cr}---is the initial elastic coarse-grained material specific storage or recompression index of the interbed. The recompression index is specified if COMPRESSION\_INDICES is specified in the OPTIONS block. Specified or calculated interbed elastic specific storage values are not adjusted from initial values if HEAD\_BASED is specified in the OPTIONS block.

\item \texttt{theta}---is the initial porosity of the interbed.

\item \texttt{kv}---is the vertical hydraulic conductivity of the delay interbed. KV must be greater than 0 if CDELAY is DELAY. Otherwise, KV can be any value.

\item \texttt{h0}---is the initial offset from the head in cell cellid or the initial head in the delay interbed. H0 is the initial head in the delay bed if SPECIFIED\_INITIAL\_INTERBED\_STATE or SPECIFIED\_INITIAL\_DELAY\_HEAD are specified in the OPTIONS block. H0 can be any value if CDELAY is NODELAY.

\item \texttt{boundname}---name of the CSUB cell.  BOUNDNAME is an ASCII character variable that can contain as many as 40 characters.  If BOUNDNAME contains spaces in it, then the entire name must be enclosed within single quotes.

\end{description}
\item \textbf{Block: PERIOD}

\begin{description}
\item \texttt{iper}---integer value specifying the starting stress period number for which the data specified in the PERIOD block apply.  IPER must be less than or equal to NPER in the TDIS Package and greater than zero.  The IPER value assigned to a stress period block must be greater than the IPER value assigned for the previous PERIOD block.  The information specified in the PERIOD block will continue to apply for all subsequent stress periods, unless the program encounters another PERIOD block.

\item \texttt{cellid}---is the cell identifier, and depends on the type of grid that is used for the simulation.  For a structured grid that uses the DIS input file, CELLID is the layer, row, and column.   For a grid that uses the DISV input file, CELLID is the layer and CELL2D number.  If the model uses the unstructured discretization (DISU) input file, CELLID is the node number for the cell.

\item \textcolor{blue}{\texttt{sig0}---is the stress offset for the cell. SIG0 is added to the calculated geostatic stress for the cell. SIG0 is specified only if MAXSIG0 is specified to be greater than 0 in the DIMENSIONS block.}

\end{description}


\end{description}

\vspace{5mm}
\subsubsection{Example Input File}
\lstinputlisting[style=inputfile]{./mf6ivar/examples/gwf-csub-example.dat}


\vspace{5mm}
\subsubsection{Available observation types}
Subsidence Package observations include all of the terms that contribute to the continuity equation for each GWF cell. The data required for each CSUB Package observation type is defined in table~\ref{table:gwf-csubobstype}. Negative and positive values for \texttt{CSUB} observations represent a loss from and gain to the GWF model, respectively.


\begin{longtable}{p{2cm} p{2.75cm} p{2cm} p{1.25cm} p{7cm}}
\caption{Available CSUB Package observation types} \tabularnewline

\hline
\hline
\textbf{Stress Package} & \textbf{Observation type} & \textbf{ID} & \textbf{ID2} & \textbf{Description} \\
\hline
\endfirsthead

\captionsetup{textformat=simple}
\caption*{\textbf{Table \arabic{table}.}{\quad}Available CSUB Package observation types.---Continued} \\

\hline
\hline
\textbf{Stress Package} & \textbf{Observation type} & \textbf{ID} & \textbf{ID2} & \textbf{Description} \\
\hline
\endhead

\hline
\endfoot

CSUB & csub & icsubno or boundname & -- & Flow between the groundwater system and a interbed or group of interbeds. \\
CSUB & inelastic-csub & icsubno or boundname & -- & Flow between the groundwater system and a interbed or group of interbeds from inelastic compaction. \\
CSUB & elastic-csub & icsubno or boundname & -- & Flow between the groundwater system and a interbed or group of interbeds from elastic compaction. \\
CSUB & coarse-csub & cellid & -- & Flow between the groundwater system and coarse-grained materials in a GWF cell. \\
CSUB & csub-cell & cellid & -- & Flow between the groundwater system for all interbeds and coarse-grained materials in a GWF cell. \\
CSUB & wcomp-csub-cell & cellid & -- & Flow between the groundwater system for all interbeds and coarse-grained materials in a GWF cell from water compressibility. \\

CSUB & sk & icsubno or boundname & -- & Convertible interbed storativity in a interbed or group of interbeds. Convertible interbed storativity is inelastic interbed storativity if the current effective stress is greater than the preconsolidation stress. \\
CSUB & ske & icsubno or boundname & -- & Elastic interbed storativity in a interbed or group of interbeds. \\
CSUB & sk-cell & cellid & -- & Convertible interbed and coarse-grained material storativity in a GWF cell. Convertible interbed storativity is inelastic interbed storativity if the current effective stress is greater than the preconsolidation stress. \\
CSUB & ske-cell & cellid & -- & Elastic interbed and coarse-grained material storativity in a GWF cell. \\

CSUB & estress-cell & cellid & -- & effective stress in a GWF cell. \\
CSUB & gstress-cell & cellid & -- & geostatic stress in a GWF cell. \\

CSUB & interbed-compaction & icsubno or boundname  & -- & interbed compaction in a interbed or group of interbeds. \\
CSUB & inelastic-compaction &  icsubno or boundname & -- & inelastic interbed compaction in a interbed or group of interbeds. \\
CSUB & elastic-compaction &  icsubno or boundname & -- & elastic interbed compaction a interbed or group of interbeds. \\
CSUB & coarse-compaction & cellid  & -- & elastic compaction in coarse-grained materials in a GWF cell. \\
CSUB & compaction-cell & cellid  & -- & total compaction in coarse-grained materials and all interbeds in a GWF cell. \\

CSUB & thickness & icsubno or boundname & -- & thickness of a interbed or group of interbeds. \\
CSUB & coarse-thickness & cellid  & -- & thickness of coarse-grained materials in a GWF cell. \\
CSUB & thickness-cell & cellid  & -- & total thickness of coarse-grained materials and all interbeds in a GWF cell. \\

CSUB & theta & icsubno & -- & porosity of a interbed . \\
CSUB & coarse-theta & cellid  & -- & porosity of coarse-grained materials in a GWF cell. \\
CSUB & theta-cell & cellid  & -- & thickness-weighted porosity of coarse-grained materials and all interbeds in a GWF cell. \\

CSUB & delay-flowtop & icsubno  & -- & Flow between the groundwater system and a delay interbed across the top of the interbed. \\
CSUB & delay-flowbot & icsubno  & -- & Flow between the groundwater system and a delay interbed across the bottom of the interbed. \\

CSUB & delay-head & icsubno  & idcellno & head in interbed delay cell idcellno (1 $<=$ idcellno $<=$ NDELAYCELLS). \\
CSUB & delay-gstress & icsubno  & idcellno & geostatic stress in interbed delay cell idcellno (1 $<=$ idcellno $<=$ NDELAYCELLS). \\
CSUB & delay-estress & icsubno  & idcellno & effective stress in interbed delay cell idcellno (1 $<=$ idcellno $<=$ NDELAYCELLS). \\
CSUB & delay-preconstress & icsubno  & idcellno & preconsolidation stress in interbed delay cell idcellno (1 $<=$ idcellno $<=$ NDELAYCELLS). \\
CSUB & delay-compaction & icsubno  & idcellno & compaction in interbed delay cell idcellno (1 $<=$ idcellno $<=$ NDELAYCELLS). \\
CSUB & delay-thickness & icsubno  & idcellno & thickness of interbed delay cell idcellno (1 $<=$ idcellno $<=$ NDELAYCELLS). \\
CSUB & delay-theta & icsubno  & idcellno & porosity of interbed delay cell idcellno (1 $<=$ idcellno $<=$ NDELAYCELLS). \\

CSUB & preconstress-cell & cellid  & -- & preconsolidation stress in a GWF cell containing at least one interbed.


\label{table:gwf-csubobstype}
\end{longtable}

\vspace{5mm}
\subsubsection{Example Observation Input File}
\lstinputlisting[style=inputfile]{./mf6ivar/examples/gwf-csub-example-obs.dat}


\newpage
\subsection{Buoyancy (BUY) Package}
Input to the Buoyancy (BUY) Package is read from the file that has type ``BUY6'' in the Name File.  If the BUY Package is included for a model, then the model will use the variable-density form of Darcy's Law for all flow calculations using the approach described by \cite{langevin2020hydraulic}.  Only one BUY Package can be specified for a GWF model. The BUY Package can be coupled with one or more GWT Models so that fluid density is updated dynamically with one or more simulated concentration fields.

The BUY Package calculates fluid density using the following equation of state from \cite{langevin2008seawat}:

\begin{equation}
\label{eqn:volumeconservationdiscrete}
%\rho = \rho_0 + \sum_{i=1}^{NRHOSPECIES} \left ( C_i - C_{i,0} \right )
\rho = DENSEREF + \sum_{i=1}^{NRHOSPECIES} DRHODC_i \left ( CONCENTRATION_i - CRHOREF_i \right )
\end{equation}

\noindent where $\rho$ is the calculated density, $DENSEREF$ is the density of a reference fluid, typically taken to be freshwater at a temperature of 25 degrees Celsius; $NRHOSPECIES$ is the number of chemical species that contribute to the density calculation, $DRHODC_i$ is the parameter that describes how density changes as a function of concentration for chemical species $i$ (i.e. the slope of a line that relates density to concentration), $CONCENTRATION_i$ is the concentration of species $i$, and $CRHOREF_i$ is the concentration of species $i$ in the reference fluid, which is normally set to zero.

\subsubsection{Stress Packages}
For head-dependent stress packages, the BUY Package may require fluid density and elevation for each head-dependent boundary so that the model can use a variable-density form of Darcy's Law to calculate flow between the boundary and the aquifer.  By default, the boundary elevation is set equal to the cell elevation.  For water-table conditions, the cell elevation is calculated as bottom elevation plus half of saturation multiplied by the cell thickness.  If desired, the user can more precisely locate the boundary elevation by specifying an auxiliary variable with the name ``ELEVATION''.  The program will use the values in this column as the boundary elevation.  A situation where this may be required is for river or general-head boundaries that are conceptualized as being on top of a model cell.  In those cases, an ELEVATION column should be specified and the values set to the top of the cell or some other appropriate elevation that corresponds to where the boundary stage applies.

By default, the boundary density is set equal to DENSEREF, commonly specified as the density of freshwater; however, there are two other options for setting the density of a boundary package.  The first is to assign an auxiliary variable with the name ``DENSITY''.  If this auxiliary variable is detected, then the density value in this column will be assigned to the density for the boundary.  Alternatively, a density value can be calculated for each boundary using the density equation of state and one or more concentrations provided as auxiliary variables.  In this case, the user must assign one auxiliary variable for each AUXSPECIESNAME listed in the PACKAGEDATA block below.  Thus, there must be NRHOSPECIES auxiliary variables, each with the identical name as those specified in PACKAGEDATA.  The BUY Package will calculate the density for each boundary using these concentrations and the values specified for DENSEREF, DRHODC, and CRHOREF.  If the boundary package contains an auxiliary variable named DENSITY and also contains AUXSPECIESNAME auxiliary variables, then the boundary density value will be assigned to the one in the DENSITY auxiliary variable.

A GWT Model can be used to calculate concentrations for the advanced stress packages (LAK, SFR, MAW, and UZF) if corresponding advanced transport packages are specified (LKT, SFT, MWT, and UZT).  The advanced stress packages have an input option called FLOW\_PACKAGE\_AUXILIARY\_NAME.  When activated, this option will result in the simulated concentration for a lake or other feature being copied from the advanced transport package into the auxiliary variable for the corresponding GWF stress package.  This means that the density for a lake or stream, for example, can be dynamically updated during the simulation using concentrations from advanced transport packages that are fed into auxiliary variables in the advanced stress packages, and ultimately used by the BUY Package to calculate a fluid density using the equation of state.  This concept also applies when multiple GWT Models are used simultaneously to simulate multiple species.  In this case, multiple auxiliary variables are required for an advanced stress package, with each one representing a concentration from a different GWT Model.  

\begin{longtable}{p{3cm} p{12cm}}
\caption{Description of density terms for stress packages}
\tabularnewline
\hline
\hline
\textbf{Stress Package} & \textbf{Note} \\
\hline
\endhead
\hline
\endfoot
GHB & ELEVATION can be specified as an auxiliary variable.  A DENSITY auxiliary variable or one or more auxiliary variables for calculating density in the equation of state can be specified \\
RIV & ELEVATION can be specified as an auxiliary variable.  A DENSITY auxiliary variable or one or more auxiliary variables for calculating density in the equation of state can be specified \\
DRN & The drain formulation assumes that the drain boundary contains water of the same density as the discharging water; auxiliary variables have no affect on the drain calculation  \\
LAK & Elevation for each lake-aquifer connection is determined based on lake bottom and adjacent cell elevations. A DENSITY auxiliary variable or one or more auxiliary variables for calculating density in the equation of state can be specified \\
SFR & Elevation for each sfr-aquifer connection is determined based on stream bottom and adjacent cell elevations. A DENSITY auxiliary variable or one or more auxiliary variables for calculating density in the equation of state can be specified \\
MAW & Elevation for each maw-aquifer connection is determined based on cell elevation. A DENSITY auxiliary variable or one or more auxiliary variables for calculating density in the equation of state can be specified \\
UZF & No density terms implemented \\
\end{longtable}

\vspace{5mm}
\subsubsection{Structure of Blocks}

\vspace{5mm}
\noindent \textit{FOR EACH SIMULATION}
\lstinputlisting[style=blockdefinition]{./mf6ivar/tex/gwf-buy-options.dat}
\lstinputlisting[style=blockdefinition]{./mf6ivar/tex/gwf-buy-dimensions.dat}
\lstinputlisting[style=blockdefinition]{./mf6ivar/tex/gwf-buy-packagedata.dat}
%\vspace{5mm}
%\noindent \textit{FOR ANY STRESS PERIOD}
%\lstinputlisting[style=blockdefinition]{./mf6ivar/tex/gwf-buy-period.dat}

\vspace{5mm}
\subsubsection{Explanation of Variables}
\begin{description}
% DO NOT MODIFY THIS FILE DIRECTLY.  IT IS CREATED BY mf6ivar.py 

\item \textbf{Block: OPTIONS}

\begin{description}
\item \texttt{HHFORMULATION\_RHS}---use the variable-density hydraulic head formulation and add off-diagonal terms to the right-hand.  This option will prevent the BUY Package from adding asymmetric terms to the flow matrix.

\item \texttt{denseref}---fluid reference density used in the equation of state.  This value is set to 1000. if not specified as an option.

\item \texttt{DENSITY}---keyword to specify that record corresponds to density.

\item \texttt{FILEOUT}---keyword to specify that an output filename is expected next.

\item \texttt{densityfile}---name of the binary output file to write density information.  The density file has the same format as the head file.  Density values will be written to the density file whenever heads are written to the binary head file.  The settings for controlling head output are contained in the Output Control option.

\end{description}
\item \textbf{Block: DIMENSIONS}

\begin{description}
\item \texttt{nrhospecies}---number of species used in density equation of state.  This value must be one or greater.  The value must be one if concentrations are specified using the CONCENTRATION keyword in the PERIOD block below.

\end{description}
\item \textbf{Block: PACKAGEDATA}

\begin{description}
\item \texttt{irhospec}---integer value that defines the species number associated with the specified PACKAGEDATA data on the line. IRHOSPECIES must be greater than zero and less than or equal to NRHOSPECIES. Information must be specified for each of the NRHOSPECIES species or the program will terminate with an error.  The program will also terminate with an error if information for a species is specified more than once.

\item \texttt{drhodc}---real value that defines the slope of the density-concentration line for this species used in the density equation of state.

\item \texttt{crhoref}---real value that defines the reference concentration value used for this species in the density equation of state.

\item \texttt{modelname}---name of GWT model used to simulate a species that will be used in the density equation of state.  This name will have no affect if the simulation does not include a GWT model that corresponds to this GWF model.

\item \texttt{auxspeciesname}---name of an auxiliary variable in a GWF stress package that will be used for this species to calculate a density value.  If a density value is needed by the Buoyancy Package then it will use the concentration values in this AUXSPECIESNAME column in the density equation of state.  For advanced stress packages (LAK, SFR, MAW, and UZF) that have an associated advanced transport package (LKT, SFT, MWT, and UZT), the FLOW\_PACKAGE\_AUXILIARY\_NAME option in the advanced transport package can be used to transfer simulated concentrations into the flow package auxiliary variable.  In this manner, the Buoyancy Package can calculate density values for lakes, streams, multi-aquifer wells, and unsaturated zone flow cells using simulated concentrations.

\end{description}


\end{description}

\vspace{5mm}
\subsubsection{Example Input File}
\lstinputlisting[style=inputfile]{./mf6ivar/examples/gwf-buy-example.dat}



\newpage
\subsection{Constant-Head (CHD) Package}
\input{gwf/chd}

\newpage
\subsection{Well (WEL) Package}
\input{gwf/wel}

\newpage
\subsection{Drain (DRN) Package}
\input{gwf/drn}

\newpage
\subsection{River (RIV) Package}
\input{gwf/riv}

\newpage
\subsection{General-Head Boundary (GHB) Package}
\input{gwf/ghb}

\newpage
\subsection{Recharge (RCH) Package -- List-Based Input}
\input{gwf/rch}

\newpage
\subsection{Recharge (RCH) Package -- Array-Based Input}
\input{gwf/rcha}

\newpage
\subsection{Evapotranspiration (EVT) Package -- List-Based Input}
\input{gwf/evt}

\newpage
\subsection{Evapotranspiration (EVT) Package -- Array-Based Input}
\input{gwf/evta}

\newpage
\subsection{Multi-Aquifer Well (MAW) Package}
\input{gwf/maw}

\newpage
\subsection{Streamflow Routing (SFR) Package}
\input{gwf/sfr}

\newpage
\subsection{Lake (LAK) Package}
\input{gwf/lak}

\newpage
\subsection{Unsaturated Zone Flow (UZF) Package}
\input{gwf/uzf}

\newpage
\subsection{Water Mover (MVR) Package}
\input{gwf/mvr}

\newpage
\subsection{Ghost-Node Correction (GNC) Package}
\input{gwf/gnc}

\newpage
\subsection{Groundwater Flow (GWF) Exchange}
\input{gwf/gwf-gwf}

